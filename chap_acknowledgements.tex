\chapter*{Remerciements}

Une thèse est une quête de savoir, un voyage que l'on entreprend par curiosité et passion, sans toujours savoir exactement où il nous mènera.
C'est une aventure parsemée d'incertitudes, où il est facile de se perdre sans un entourage fiable sur lequel on peut compter, qui éclaire notre chemin et nous donne la force d'avancer à chaque étape.

C'est pourquoi je tiens avant tout à remercier Nic et Juan, mes directeurs de thèse, pour leur précieux encadrement, leurs conseils avisés et leur soutien tout au long de cette aventure.
Merci de m'avoir fait confiance, d'avoir compris mes attentes, de m'avoir consacré votre temps et de tout ce que vous avez pu faire d'autres pour moi.
Je suis heureux que nous ayons été sur la même longueur d’onde. 
Travailler avec vous fut un véritable plaisir et m’a permis de mener ce voyage à bon port avec enthousiasme.

Un merci tout particulier à Marco Cognetti, qui a rejoint ce projet en cours de route.
Merci pour ton engagement aussi bien dans le suivi de ma thèse que dans la rédaction.
Ton expertise sur le sujet m’a été d’une aide précieuse et, bien que tu n’aies pas été officiellement rattaché à cette thèse, j’espère que ces quelques mots mettront en valeur ta contribution essentielle à sa réussite.

Je tiens ensuite à remercier ma compagne Laure.
Au-delà de cette thèse, ma plus belle aventure est sans aucun doute celle que nous vivons tous les deux.
Merci pour tes attentions, ta petience, ton soutien, ou encore ton rire.
Merci de m'avoir remonté le moral quand il était au plus bas.
Merci d'avoir supporté mes interminables discussions sur la génération de mon dataset.
Sans toi, cette thèse aurait été bien différente et ma vie, en général, serait beaucoup plus terne.

Merci à ma mère Françoise et mon père Eric.
Merci d'avoir soutenu ma décision de me lancer dans cette thèse.
Merci d'avoir stressé à ma place tout au long de ces années, surtout toi, papa.
Je vous dois ce que je suis aujourd'hui.
J'espère vous avoir rendus fiers.

Merci à mes frères Clément et Paulin.
Même si la distance ne nous permet plus de nous voir aussi souvent, vous occuperez toujours une place de choix dans mon coeur.
Je suis ravi de vous voir vous épanouir ; cela me motive à me surpasser et m’a permis de terminer cette thèse et ce manuscrit sereinement.
Bien entendu, j'espère que vous mettrez cedit manuscrit sous vitrine, cela va de soi.
Je tiens également à remercier ma belle-soeur Mélanie pour l’intérêt que tu as porté à mes travaux et, au moins toi, de ne pas tricher à Elfenland.

Merci à mes cousins, cousines, oncles, tantes, parrain, marraine et grands-parents pour leur soutien et enthousiasme.

Je tiens ensuite à remercier mes amis: Corentin, Katerina, Erwan, Maelenn, Cyriaque, Seb, Franklin, Jeanne et JC, pour ces bons moments à incarner les meilleurs personnages d'Yskanov, ou encore pour ces échanges frauduleux d'une pierre contre trois argiles.
Les moments passés ensemble me sont précieux et m'ont offert des bouffées d'oxygène quand mon esprit ne pensait plus qu'à cette thèse.

Merci également à mes amis de la fameuse chambre 106.5, Bastien et Eugène.
Merci de vous être intéressés à mes travaux et d'avoir compris mes absences.
J'espère bientôt pouvoir rattraper le temps perdu et que nous nous remémorerons encore les bêtises de l'internat.

Cette thèse aurait eu une tout autre saveur sans les membres de la "Place des Swifties", avec qui j’ai passé d’excellents moments.
Merci à toi, Lou, pour ces franches rigolades, ces moments de malaise, ces concours de lancer de glands, et bien sûr, ce projet de tyrolienne entre Toulouse et les Alpes (en espérant que personne ne te vole l’idée en lisant ceci) !
Merci à toi, Bastien, d’être toujours partant pour engloutir plusieurs cafés avec moi à n’importe quelle heure, pour ces coupes de cheveux rocambolesques, pour ton aide précieuse dans le setup de mes expériences, pour ne pas avoir hésité un seul instant – mais alors pas du tout – à rejoindre ce carré, et bien sûr, pour ton sens de l’humour inégalable.
Je suis sûr que cette dernière qualité m’a permis de récupérer pas mal d’heures de vie perdues, consumées par la thèse.
Merci à toi, ô terrifiant Guillaume ! Nos joutes verbales (pour rester politiquement correct) me manqueront.
Merci pour ton expertise technique en programmation et pour avoir toujours répondu présent pour les impressions 3D.
Et bien sûr, merci de m’avoir conseillé d’excellents jeux, tels qu'Andor !
Finalement, merci à toi, Smail, mon acolyte de voyage depuis la première heure de cette thèse.
J'ai énormément apprécié les moments passés avec toi, que ce soit au travail, lors des conférences, ou même au McDo à 1h du matin après les deadlines.
Avoir travaillé aux côtés de quelqu'un d’aussi compétent que toi m’a poussé à donner le meilleur de moi-même pour rester à la hauteur.
Tu seras toujours le bienvenu pour "crash" chez moi, à condition que tu laisses des Chocobons, bien sûr.

Merci à toi, Philippe, pour tous ces bons moments passés ensemble.
Merci pour ta gentillesse et ton empathie.
Ton expertise photo en détourage de drone m'a sauvé bien des heures de sommeil.
Pas sûr de vouloir encore suivre tes "raccourcis" en randonnée, mais c’est avec plaisir que je viendrais prendre une bière au BSC ou regarder un match de rugby.

Merci infiniment Anthony pour toutes ce rigolades.
Merci pour ces discussions allant de l'algorithmique au jeu de rôle, en passant par l'élaboration de notre fameux monde gymnastique, où les personnes s'empileraient les unes sur les autres et où différents bâtiments et activités apparaîtraient en fonction des étages.
J'espère te recroiser, non pas dans ce monde gymnastique catastrophique, mais bien en enfer, Hellbanger.

Je tiens également à te remercier, William, pour ces discussions, ces après-midis jeux de société, et bien sûr pour ta maintenance sur Blender.
Merci à toi Stephy pour ces moments kool passé ensemble et ta gentillesse.
Merci, Illinka, pour ta prise de parole au sujet du sapin, pour l'instauration de ce jeudi Ghibli que je n'ai pas su tenir, et d'être toujours partante pour un concours de Glams.
Merci à vous Fadma et Maël pour les bons moments passés ensemble.

Merci également aux personnes que j'ai vu partir au court de cette aventure mais qui m'ont permis une intégration plus que chaleureuse, merci Amandine, Antoine, Yannick, Jeremy, Gianluca, Dario.
Merci à Arthur, Félix, Anthony, Aurélie, Rachid, Adrien, Roland, Virgile, Jonas, Alessia, Rebecca, Phani, James, et bien d’autres encore, constituant cette équipe RIS.
Merci de m’avoir si bien accueilli, et un grand merci à Simon Lacroix d'être à la tête de cette équipe formidable.

Merci aux membres de l’équipe Rainbow, et tout particulièrement à Paolo pour m’avoir donné cette opportunité et pour ton enthousiasme à propos de mes travaux. 
Merci également à Pascal et Ali pour leurs échanges enrichissants.

Je souhaite également exprimer ma gratitude envers les membres de mon jury.
Un grand merci à Daniel pour avoir présidé ce jury, et à Marilena, Pedro et Marco pour l’accueil réservé à mes travaux.

Merci au Sapin Éternel, ainsi qu’à ceux qui l’entretiennent, égayant le hall de notre bâtiment et y apportant joie et bonne humeur.

Enfin merci au LAAS-CNRS de m'avoir acceuilli et à l'ANR d'avoir financer mes travaux de recherche.