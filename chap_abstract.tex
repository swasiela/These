\chapter*{Abstract}

This thesis addresses the problem of generating robust and accurate trajectories taking into account uncertainties in the robot dynamic model. 
Based on the notion of \emph{closed-loop sensitivity}, which quantifies deviations in the closed-loop trajectories of any robot/controller pair against uncertainties in the robot model parameters, so-called 'uncertainty tubes' can be derived for bounded parameter variations.
Such tubes bound the system evolution both in the state and control input spaces.
Based on the 'feedback motion planning' paradigm, this work leverages these 'uncertainty tubes' to enforce robust constraints within sampling-based planners.

The first contribution of this thesis focuses on generating globally sensitivity-optimal trajectories while enforcing robust constraints thanks to uncertainty tubes.
However, results show that computing these uncertainty tubes at each iteration of a sampling-based planner is a bottleneck for the method.
Therefore, a lazy robust feasibility check is proposed to limit the frequency of computing uncertainty tubes, thus improving the computational efficiency of the framework.

Another contribution is to explore deep learning neural network that can be used to speed up closed-loop sensitivity dynamic and subsequent tubes computation.
By leveraging the structural similarity between ordinary differential equations and recurrent neural networks, a GRU-based architecture is proposed to directly link a planned trajectory with uncertainty tubes, achieving an order-of-magnitude improvement in computation time.

The thesis also shows how to integrate GRU-based predictions into a sampling-based planner, resulting in a more computationally efficient planning framework. 
Furthermore, while robust state-of-the-art methods primarily focus on satisfying robust constraints, this work leverages uncertainty tubes to define a cost function that enables the planning of task-specific, accurate trajectories.

Finally, the proposed sensitivity-based planning framework is experimentally validated on a 3D quadrotor in two challenging scenarios: a navigation through a narrow window, and an in-flight ``ring catching'' task that requires high accuracy. 