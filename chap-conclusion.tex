\chapter{Conclusion}\label{chap:concl}
\markboth{Conclusion}{}% To set left/right header

\glsresetall

In this thesis, 

\paragraph{Future Works} Several xxxx 


From the sensitivity point of view 

From the algorithmic point of view 


One can further improve the \myglsentry{lazysamp} variants to allow online robust replanning by investigating a more sofisticate robust reconnection procedure when not using sensitivity cost but only robust constraints as done in "R. Bohlin and L. E. Kavraki, "Path planning using lazy prm", Proceedings 2000 ICRA. Millennium Conference. IEEE International Conference on Robotics and Automation. Symposia Proceedings (Cat. No. 00CH37065), vol. 1, pp. 521-528, 2000.".
In brief, the colliding edge is removed from the tree and searching continues with adding samples near the discarded edge to connect the two sections of the tree. The lazy-collision checking alleviates one of the main computational bottlenecks of sampling-based planning by limiting collision checking to the edges of the solution. 
However, if number of collisions in the solution is large, then many efforts are required to repair the solution. 
Therefore, an efficient online replanning technic would consist on a simple low frequency "geometric" planning which robustly checks and reconnect in a lazy way and tacking advantage of the learning ! SUPER FAST BZZ BZZ 

In this thesis, the aim is to be as close as the reality at the global motion planning level such that computation resources are save at runtime to perform other stuffs.
However, in this thesis an extension is to take into account the sensors model to be better.

Furthermore, CBF can be used to take into account external disturbances.