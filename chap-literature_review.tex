\chapter{Related work}\label{chap:related_work}
\markboth{Related work}{}% To set left/right header
% \localtableofcontents

\section{Decoupled approaches}
\subsection{Motion planning}
In motion planning a distinction is usually made between the local planner and the global planner. 
The former is in charge of producing a valid trajectory between two configurations (or states) of the system without necessarily taking collisions into account. 
The later is the overall algorithmic process that is in charge of solving the motion planning problem by exploring the configuration space (or the state space) of the system. 
It relies on multiple calls to the local planner.
\subsubsection{Path finding}
\subsubsection{Trajectory generation}
\subsubsection{Kinodynamic motion planning}
\subsection{Robust control}
\subsubsection{DFL}
\subsubsection{Geometric controller}
\subsubsection{MPC}

\section{Robust and control-aware motion planning}
\subsection{Robust motion planning}
\subsubsection{Chance-constrained}
\subsubsection{Particle RRT}
\subsection{Control-aware motion planning}
\subsubsection{xxxx}
\subsection{Robust control-aware motion planning}
\subsubsection{Randomized uncertainty propagation}
\subsubsection{FaSTrack}
\subsubsection{LQR-trees}
\subsubsection{Contraction theory}

\todomarker{}