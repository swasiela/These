\chapter{Related work}\label{chap:related_work}
\markboth{Related work}{}% To set left/right header
% \localtableofcontents

This chapter provides an overview of the related work on the key concepts that this thesis builds upon and compares with.

\section{Motion planning}

\subsection{Path planning}

This subsection provides an overview of various path planning approaches. 
For a more detailed survey, please refer to~\cite{cLavalle, cKavraki, cFrazzoli}.
The path planning problem focuses on finding a collision-free and/or optimal path between two configurations: a starting configuration and a goal configuration. 
The work of~\cite{cConfigSpace} introduced the configuration space concept as a general framework for planning motions in arbitrary kinematic systems.
This idea provided a solid foundation for the field and, as noted in~\cite{cSpatialPlan}, formally established the motion planning problem as the task of finding a path within the configuration space.

However, planning motion in the configuration space poses significant challenges due to the problem computational complexity. 
A key stone in addressing this issue is the introduction of artificial potential fields~\cite{cAPF}, which create a repulsive force field around obstacles and an attractive force field toward the target.
However, although this strategy has proven to be efficient, it sacrifices guarantees such as algorithm completeness and optimality.

Another common approach to the path planning problem involves search-based techniques. 
These methods operate on a discrete representation of the configuration space, where vertices correspond to a finite set of robot configurations, and edges represent possible transitions between these configurations.
The desired path is found by performing a search for a minimum-cost path in such a graph using Dijkstra algorithm~\cite{cDijk}. 

Advances in computational techniques have enabled the development of a new class of algorithms, commonly referred to as sampling-based planners that allow to generate global motion plans.
A key contribution of the domain is the work of~\cite{cLatombe}, which improves the aforementioned potential fields approach by integrating a random walk mechanism, thus guaranteeing the probabilistic completeness of the algorithm.
Sampling-based planners are divided in two main categories: graph-based planners and tree-based planners.

Graph-based sampling-based planners, such as \gls{prm}~\cite{cPRM}, generate a roadmap by sampling configurations in the free space and connecting them with feasible paths. 
The technique involves a graph building phase and a query phase.
The graph is first built by randomly sampling configurations (denoted nodes) in the robot configuration space.
Each sampled configuration is checked for collisions with obstacles.
For each collision-free node, the planner attempts to connect it to nearby free configurations by creating edges between them.  
The edges are valid if the path connecting the two configurations is collision-free.
This procedure typically involves a notion of neighborhood that is space-dependent and generally increases as the dimensionality of the space grows.
Once the graph is built, a graph search query is performed to compute the desired path, using graph search techniques such as Dijkstra algorithm~\cite{cDijk} or A*~\cite{cA*}.

The other planner family among sampling-based planners are tree-based planners such as the well-known \gls{rrt}~\cite{cRRT}.
This algorithm generates a tree starting from an initial configuration.
At each iteration, a new configuration is sampled $q^{rand}$.
It's nearest neighbor among the existing tree nodes is found $q^{near}$, and then an extension is attempt starting from $q^{near}$ toward $q^{rand}$ leading to a new configuration $q^{new}$.
If the extension is collision free, the configuration is added to the tree as anew node with an edge connecting the two nodes (see Figure~\ref{fig:rrt}).

Therefore, sampling-based motion planners have become one of the most widely used strategy for generating global motion paths in the presence of obstacles, due to their efficiency in handling complex and high-dimensional configuration spaces, as well as their completeness guarantee.
Advances in sampling-based planners have then focused on generating feasible and cost-effective paths through various strategies, such as the transition-based approach~\cite{cTRRT}. 
These strategies often achieve asymptotic optimality by employing optimal connection procedures during the tree or graph construction process~\cite{cRRTstar, cTRRTstar, cFMT}.
Research on sampling-based planners has expanded into various areas, such as developing suitable sampling strategies~\cite{cSampling}, exploring lazy collision checking approaches~\cite{cLazy1}, extending to the molecular domain~\cite{cMolecular}, and more.

The aforementioned sampling-based methods are generally referred to as global planning methods, as they rely on local planning techniques to generate the edges that connect the sampled configurations.
Such local planning methods are not considered global, as they do not address the entire motion planning problem. 
A local planner is typically fast; however, the generated path may not fully satisfy the problem constraints (e.g., it could result in collisions).
Traditional local strategies typically involve path segments, but local planners often need to account for additional constraints.
For example, significant work has been done to enforce kinematic constraints on these local plans, ensuring the kinematic feasibility of the resulting global paths.
Such work includes, for example, Dubins path generation~\cite{cDubins}, which enables connecting two configurations under forward motion constraints and curvature constraints for car-like robots. 
The Reeds-Shepp method~\cite{cReeds} extends this local path generation procedure to car-like robots capable of moving both forward and backward.

\begin{figure} [htp]
    \centering
    \includegraphics[width=0.6\linewidth]{figures/models/rrt.png} 
    \caption{Illustration of the \myglsentry{rrt} tree extension procedure.}%
    \label{fig:rrt}%
  \end{figure}

\subsection{Kinodynamic planning}

While the aforementioned path finding methods focus on generating a route between two robot configurations, they do not address the problem on how the robot should move along this route over time.
The kinodynamic motion planning problem extends traditional motion planning by considering both the kinematic constraints (e.g., turning radius) and dynamic constraints (e.g., velocity, acceleration) of the robot, ensuring that the generated motion plans are physically executable.

% 2. Trajectory Generation

% While path finding determines the spatial route, trajectory generation focuses on how the robot should move along this route over time, considering kinematic and dynamic constraints to ensure feasibility.

%     Polynomial Splines:
%     Polynomial-based methods, such as cubic splines and quintic splines, are widely used for smooth trajectory interpolation between waypoints.
%         These methods ensure continuity in position, velocity, and acceleration, which is essential for smooth robotic motion.
%         They are computationally efficient and suitable for applications like robotic manipulators or UAV navigation.

%     Optimization-based Methods:
%     Optimization frameworks like CHOMP and STOMP formulate trajectory planning as an optimization problem, minimizing a cost function that accounts for obstacle avoidance and trajectory smoothness.
%         CHOMP uses gradient-based optimization to refine an initial trajectory, making it collision-free while maintaining smoothness.
%         STOMP, on the other hand, employs stochastic sampling to explore feasible trajectories, offering robustness in environments with complex obstacle configurations.
%         More advanced methods integrate convex optimization or leverage real-time solvers for handling dynamic constraints, enabling robots to adapt to changes in their environments.

%     Learning-based Approaches:
%     Data-driven methods leverage neural networks trained on large datasets of prior trajectories to predict feasible paths.
%         Imitation learning is a common technique where the robot learns from expert demonstrations to generate trajectories in similar scenarios.
%         These methods excel in reducing computation time, making them highly suitable for real-time applications, though they often rely on high-quality training data and can struggle with generalization.

% 3. Kinodynamic Motion Planning

% Kinodynamic motion planning extends traditional motion planning by considering both the kinematic constraints (e.g., turning radius) and dynamic constraints (e.g., velocity, acceleration) of the robot, ensuring that the generated motion plans are physically executable.

%     Time-parameterized RRTs:
%     Extensions of RRT, such as RRT-Connect and Kinodynamic RRT, incorporate dynamic constraints directly into the planning process.
%         These planners ensure that the generated paths are not only collision-free but also adhere to the robot's motion model, such as maximum acceleration or turning capabilities.
%         The addition of time-parameterization allows for the generation of time-efficient trajectories, suitable for systems requiring fast, dynamic responses.

%     Model Predictive Control (MPC):
%     MPC-based planners solve an optimization problem over a receding time horizon, generating dynamically feasible motion plans that account for constraints and obstacles.
%         MPC is particularly effective for systems with complex dynamics, such as mobile robots, autonomous vehicles, and drones.
%         Its ability to continuously update plans in real time makes it ideal for dynamic and uncertain environments.

%     Search-based Methods:
%     Algorithms like state-time A* or lattice-based planning search directly in the state-time space, incorporating kinodynamic constraints at the planning level.
%         These methods often use discretization to simplify the problem but can become computationally expensive in high-dimensional spaces.
%         They are well-suited for problems requiring precise adherence to constraints, such as robotic manipulators or autonomous cars navigating through dense traffic.

%     Nonlinear Dynamics Integration:
%     For robots with non-holonomic constraints (e.g., wheeled robots), planning methods often employ predefined motion primitives, such as Dubins paths or Reeds-Shepp curves, to generate feasible trajectories.
%         These techniques ensure that the paths are physically realistic while maintaining computational efficiency.

\section{Motion planning under uncertainty}

\subsection{Chance-constrained}

\subsection{POMDP}

\subsection{Chance-constrained}

\section{Control-aware motion planning}

\subsection{MPC}

\subsection{FaSTrack}

\subsection{LQR-trees}

\subsection{Contraction theory}

\subsection{Randomized uncertainty propagation}

\subsection{Sensitivity}