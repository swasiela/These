\chapter{Related work}\label{chap:related_work}
\markboth{Related work}{}% To set left/right header
% \localtableofcontents

This chapter provides an overview of the related work on the key concepts that this thesis builds upon and compares with.

\section{Motion planning}

\subsection{Path planning}

This subsection provides an overview of various path planning approaches. 
For a more detailed survey, please refer to~\cite{cLavalle, cKavraki, cFrazzoli}.
The path planning problem focuses on finding a collision-free and/or optimal path between two configurations: a starting configuration and a goal configuration. 
The work of~\cite{cConfigSpace} introduced the configuration space concept as a general framework for planning motions in arbitrary kinematic systems.
This idea provided a solid foundation for the field and, as noted in~\cite{cSpatialPlan}, formally established the motion planning problem as the task of finding a path within the configuration space.

However, planning motion in the configuration space poses significant challenges due to the problem computational complexity. 
A key stone in addressing this issue is the introduction of artificial potential fields~\cite{cAPF}, which create a repulsive force field around obstacles and an attractive force field toward the target.
However, although this strategy has proven to be efficient, it sacrifices guarantees such as algorithm completeness and optimality.

Another common approach to the path planning problem involves search-based techniques. 
These methods operate on a discrete representation of the configuration space, where vertices correspond to a finite set of robot configurations, and edges represent possible transitions between these configurations.
The desired path is found by performing a search for a minimum-cost path in such a graph using Dijkstra algorithm~\cite{cDijk}. 

Advances in computational techniques have enabled the development of a new class of algorithms, commonly referred to as sampling-based planners that allow to generate global motion plans.
A key contribution of the domain is the work of~\cite{cLatombe}, which improves the aforementioned potential fields approach by integrating a random walk mechanism, thus guaranteeing the probabilistic completeness of the algorithm.
Sampling-based planners are divided in two main categories: graph-based planners and tree-based planners.

Graph-based sampling-based planners, such as \gls{prm}~\cite{cPRM}, generate a roadmap by sampling configurations in the free space and connecting them with feasible paths. 
The technique involves a graph building phase and a query phase.
The graph is first built by randomly sampling configurations (denoted nodes) in the robot configuration space.
Each sampled configuration is checked for collisions with obstacles.
For each collision-free node, the planner attempts to connect it to nearby free configurations by creating edges between them.  
The edges are valid if the path connecting the two configurations is collision-free.
This procedure typically involves a notion of neighborhood that is space-dependent and generally increases as the dimensionality of the space grows.
Once the graph is built, a graph search query is performed to compute the desired path, using graph search techniques such as Dijkstra algorithm~\cite{cDijk} or A*~\cite{cA*}.

The other planner family among sampling-based planners are tree-based planners such as the well-known \gls{rrt}~\cite{cRRT}.
This algorithm generates a tree starting from an initial configuration.
At each iteration, a new configuration is sampled $q^{rand}$.
It's nearest neighbor among the existing tree nodes is found $q^{near}$, and then an extension is attempt starting from $q^{near}$ toward $q^{rand}$ leading to a new configuration $q^{new}$.
If the extension is collision free, the new configuration is added to the tree as a new node with an edge connecting the two nodes.

Therefore, sampling-based motion planners have become one of the most widely used strategy for generating global motion paths in the presence of obstacles, due to their efficiency in handling complex and high-dimensional configuration spaces, as well as their completeness guarantee.
Advances in sampling-based planners have then focused on generating feasible and cost-effective paths through various strategies, such as the transition-based approach~\cite{cTRRT}. 
These strategies often achieve asymptotic optimality by employing optimal connection procedures during the tree or graph construction process~\cite{cRRTstar, cTRRTstar, cFMT}.
Research on sampling-based planners has expanded into various areas, such as developing suitable sampling strategies~\cite{cSampling}, exploring lazy collision checking approaches~\cite{cLazy1}, extending to the molecular domain~\cite{cMolecular}, and more.

The aforementioned sampling-based methods are generally referred to as global planning methods, as they rely on local planning techniques to generate the edges that connect the sampled configurations.
Such local planning methods are not considered global, as they do not address the entire motion planning problem. 
A local planner is typically fast; however, the generated path may not fully satisfy the problem constraints (e.g., it could result in collisions).
Traditional local strategies typically involve path segments, but local planners often need to account for additional constraints.
For example, significant work has been done to enforce kinematic constraints on these local plans, ensuring the kinematic feasibility of the resulting global paths.
Such work includes, for example, Dubins path generation~\cite{cDubins}, which enables connecting two configurations under forward motion constraints and curvature constraints for car-like robots. 
The Reeds-Shepp method~\cite{cReeds} extends this local path generation procedure to car-like robots capable of moving both forward and backward.

\subsection{Kinodynamic planning}

While the aforementioned path finding methods focus on generating a route between two robot configurations, they do not address the problem on how the robot should move along this route over time.
Additionally, path planning does not account for the robot dynamic constraints, such as velocity and acceleration. 
The problem of generating motion plans that consider both kinematic constraints (e.g., turning radius) and dynamic constraints, ensuring the plans are physically executable, is known as kinodynamic motion planning.
Therefore, the kinodynamic problem extends beyond searching in the robot configuration space by considering its entire state space.

\subsubsection{Local planners}

Local planners, or steering methods, focus on generating feasible motion between two states while respecting kinematic and dynamic constraints. 
They are often used as building blocks in global kinodynamic planners. 

A common approach is to take advantage of polynomial-based methods, such as splines, that are widely used for smooth trajectory interpolation between waypoints.
For instance, the kino-splines local planner proposed in~\cite{cKino} generates smooth trajectories by solving a two-point \gls{bvp} between initial and final robot states. 
Based on the system differential flatness, this method ensures the creation of continuous and differentiable trajectories up to the 4th order by employing a bang-bang snap control strategy. 
The resulting trajectories respect kinodynamic constraints up to the jerk level and enable time-optimal connections between two given states.

Another example of a spline-based approach, which contrasts with the bang-bang snap strategy of kino-splines, is the minimum snap trajectory generation method~\cite{cMinimumSnap}. 
This approach, also based on system differential flatness, formulates an optimization problem that minimizes the integral of a weighted combination of the squared position snap and the squared yaw acceleration. 
It ensures continuity between trajectory segments, satisfies kinematic conditions at the start and end of the motion, and respects actuation limits.

While the local planners mentioned earlier primarily focus on enforcing kinodynamic constraints, Bézier curve-based planners~\cite{cBezier,cBezier2} offer additional benefits due to their inherent smoothness, and their convex hull property.
Their convex hull property ensures that the trajectory remains within the convex hull of its control points. 
This simplifies collision detection, as it guarantees that if the convex hull is free of obstacles, the entire trajectory will be too. 
Additionally, adjusting control points affects only a localized portion of the trajectory, allowing precise modifications without affecting the entire trajectory. 
This approach is particularly useful in environments where efficient collision detection is crucial.
Furthermore, its smoothness also naturally accommodates velocity, acceleration, and higher-order dynamic constraints, making it suitable for kinodynamic motion planning in various robotic systems. 

Another powerful method is \gls{chomp}~\cite{cCHOMP}, which formulates trajectory generation as an optimization problem starting from an initial trajectory as guess. 
The cost function in \myglsentry{chomp} typically balances smoothness and obstacle avoidance.
Smoothness is ensured by penalizing higher derivatives of the trajectory, while obstacle avoidance is achieved through the use of a signed distance field. 
\myglsentry{chomp} can incorporate dynamic and kinematic constraints, such as joint limits and velocity bounds, which are important for kinodynamic planning. 
Using gradient-based optimization, \myglsentry{chomp} iteratively refines the initial trajectory, ensuring that it satisfies all constraints while minimizing the cost function. 
While effective in high-dimensional or cluttered environments, \myglsentry{chomp} does require a good initial trajectory and can struggle in environments with non-convex obstacles.

While \myglsentry{chomp} can handle differentiable cost function due its gradient-based optimization, the \gls{stomp} algorithm~\cite{cSTOMP} is designed to optimize non-differentiable cost functions. 
\myglsentry{stomp} iteratively improves trajectories by introducing small perturbations to the initial guess, evaluating the resulting costs, and updating the trajectory based on the best-performing samples. 
This process also accounts for kinodynamic constraints and obstacle avoidance by adding state cost penalty in the cost function formulation.
While \myglsentry{stomp} is computationally more expensive than \myglsentry{chomp} as it relies on several sampling-based rollouts, its ability to handle more complex cost functions and find better solutions in challenging environments makes it highly versatile for kinodynamic motion planning.

\subsubsection{Forward propagation}\label{sec:forwardplanning}

Global sampling-based motion planners can generate globally kinodynamically feasible trajectories by leveraging the aforementioned local planners to connect their sampled state.
However, such local planners may not be available for every system.
Therefore, algorithms were developed to bypass this need by performing dynamic system forward propagation instead of solving a complex \myglsentry{bvp}.

Kinodynamic motion planners, such as Kinodynamic RRT~\cite{cKinoRRT} and \gls{sst}~\cite{cSST}, are designed to address the challenges of planning for robots with complex dynamics without relying on a local planner. 
These methods utilize system forward dynamic propagation by sampling control inputs and propagation time. 
Starting from an initial robot state, the system dynamics are integrated in an open-loop manner to propagate the state, allowing the planner to account for both the robot kinematic limitations and its dynamic behavior. 
This approach ensures that the generated paths are collision-free and physically executable while eliminating the need for local planners.

\section{Planning under uncertainty}

While the previous section focused on motion planning algorithms that can deal with kinodynamic constraints of various robot, none of them consider the unavoidable presence of uncertainties (e.g. external disturbances, sensor noise, model parameter mismatches, etc.).

A common approach to managing uncertainties is to compensate for them in real-time using robust controllers, such as H-infinity or \gls{lpv} methods.
However, these methods often encounter difficulties in maintaining robustness when applied to non-robust reference trajectories, due to their inherently local nature.
Additionally, they struggle to achieve global optimality and performance, which are typically better addressed by global planning approaches.
Therefore, considerable research has focused on designing motion planning algorithms that incorporate uncertainty into the trajectory generation process, with the goal of producing robust and feasible plans.

A seminal work in the domain is the \gls{lmt} framework~\cite{cLMT} that address the challenge of computing compliant-motion strategies

{Fine motion planning under uncertainty}

{POMDP}

{Chance-constrained}

{Particle RRT}



\section{Control-aware motion planning}

The algorithms presented so far have been designed to handle kinodynamic constraints and/or ensure robustness against various uncertainties.
However, none of them typically consider the inevitable presence of a feedback controller responsible for executing the generated trajectories. 
This controller might deviate from the planned trajectory to address uncertainties and disturbances, which can quickly compromise feasibility and optimality of the plan.
It is important to note that kinodynamic approaches relying on system dynamics forward propagation, as discussed in Section~\ref{sec:forwardplanning}, do not incorporate feedback actions.
Instead, they directly sample in the control input space, resulting in an open-loop propagation strategy.

MPC and control-aware from M Tognon

\section{Robust control-aware motion planning}

one should guarantee safe operation for all uncertainty realizations within these bounds taking into account for the presence of a feedback controller. This task is referred to as robust control-aware motion planning (or robust feedback motion planning).

\subsection{FaSTrack}

\subsection{LQR-trees}

\subsection{Contraction theory}

\subsection{Randomized uncertainty propagation}

\subsection{Sensitivity}

\section{Summary}