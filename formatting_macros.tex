% This file contains different settings pertianing to formatting, as well as the newcommand definitions

%%%%%%%%%%%%%%%%%%%%%%%%%%%%%
% #region COLOR DEFINITIONS %
%%%%%%%%%%%%%%%%%%%%%%%%%%%%%

\definecolor{IBMBlue}{HTML}{648FFF}
\definecolor{IBMPurple}{HTML}{785EF0}
\definecolor{IBMMagenta}{HTML}{DC267F}
\definecolor{IBMOrange}{HTML}{FE6100}
\definecolor{IBMYellow}{HTML}{FFB000}
\colorlet{Accent5}{IBMYellow}
\colorlet{Accent2}{IBMBlue}
\colorlet{Accent3}{IBMMagenta}
\colorlet{Accent4}{IBMPurple}
\colorlet{Accent1}{IBMOrange}

\definecolor{ThreeRed}{HTML}{D41159}
\definecolor{ThreeOrange}{HTML}{FFB000}
\definecolor{ThreeGreen}{HTML}{77AB02}

% #endregion

%%%%%%%%%%%%%%%%%%%%%
% #region Algorithm %
%%%%%%%%%%%%%%%%%%%%%

\algrenewcommand\algorithmicrequire{\textbf{Input:}}
\algrenewcommand\algorithmicensure{\textbf{Output:}}

% Define new blocks, see https://github.com/chrmatt/algpseudocodex/issues/3
\algnewcommand\algorithmicsoptional{\textbf{optional}}%

\makeatletter
\algdef{SE}[OPTIONAL]{Optional}{EndOptional}[1]{\algpx@startIndent\algpx@startCodeCommand\algorithmicsoptional}{\algpx@endIndent\algpx@startCodeCommand\algorithmicend\ \algorithmicsoptional}%

\ifbool{algpx@noEnd}{%
  \algtext*{EndOptional}%
  %
  % end indent line after (not before), to get correct y position for multiline text in last command
  \apptocmd{\EndOptional}{\algpx@endIndent}{}{}%
}{}%

\pretocmd{\Optional}{\algpx@endCodeCommand}{}{}

    % for end commands that may not be printed, tell endCodeCommand whether we are using noEnd
    \ifbool{algpx@noEnd}{%
    \pretocmd{\EndOptional}{\algpx@endCodeCommand[1]}{}{}%
}{%
\pretocmd{\EndOptional}{\algpx@endCodeCommand[0]}{}{}%
}%
\makeatother

% #endregion

%%%%%%%%%%%%%%%%%
% #region Fonts %
%%%%%%%%%%%%%%%%%

% Find a way to define a new family and use it twice
% \newfontfamily{\lmodernsc}{Latin Modern Roman}[
%   SmallCapsFont = Latin Modern Roman Caps,
%   UprightFeatures = {SmallCapsFeatures = {}}, % Suppresses warning about missing smcp feature.
%   BoldFeatures = {SmallCapsFont = {CMU Serif Bold}},
%   ItalicFeatures = {SmallCapsFont = {CMU Serif Roman Slanted}},
%   BoldItalicFeatures = {SmallCapsFont = {CMU Serif Bold Slanted}},
%   SmallCapsFeatures={Letters=SmallCaps}]
\setmainfont{Latin Modern Roman}[
  SmallCapsFont = Latin Modern Roman Caps,
  UprightFeatures = {SmallCapsFeatures = {}}, % Suppresses warning about missing smcp feature.
  BoldFeatures = {SmallCapsFont = {CMU Serif Bold}},
  ItalicFeatures = {SmallCapsFont = {CMU Serif Roman Slanted}},
  BoldItalicFeatures = {SmallCapsFont = {CMU Serif Bold Slanted}},
  SmallCapsFeatures={Letters=SmallCaps}]
% Loading font and renaming for template compat
\newfontfamily{\pbkcompat}{TeX Gyre Bonum}[NFSSFamily=pbk]
\newfontfamily{\cmrcompat}{Latin Modern Roman}[
  NFSSFamily=cmr,
  SmallCapsFont = Latin Modern Roman Caps,
  UprightFeatures = {SmallCapsFeatures = {}}, % Suppresses warning about missing smcp feature.
  BoldFeatures = {SmallCapsFont = {CMU Serif Bold}},
  ItalicFeatures = {SmallCapsFont = {CMU Serif Roman Slanted}},
  BoldItalicFeatures = {SmallCapsFont = {CMU Serif Bold Slanted}},
  SmallCapsFeatures={Letters=SmallCaps}]
\newfontfamily{\pcrcompat}{TeX Gyre Cursor}[NFSSFamily=pcr]
\setmonofont{Latin Modern Mono}

\newfontfamily\ttprop{Latin Modern Mono Prop}
\DeclareTextFontCommand{\textttpit}{\ttprop\itshape}
\DeclareTextFontCommand{\textttp}{\ttprop}

% #endregion

%%%%%%%%%%%%%%%%%%%%%%%%%%%%%%%%%%%%%
% #region Theorem-like environments %
%%%%%%%%%%%%%%%%%%%%%%%%%%%%%%%%%%%%%


% 'Plain' theorems
\newtheorem{theorem}{Theorem}[chapter]
\newtheorem{corollary}{Corollary}[theorem]
\newtheorem{lemma}[theorem]{Lemma}

\theoremstyle{definition}
\newtheorem{definition}{Definition}[chapter]

\theoremstyle{remark}
\newtheorem*{remark}{Remark}

% 'Boxed' Theorems

% Styling for examples
\newtheoremstyle{example}% Name
{}% space above
{}% space below
{\normalfont}% body font
{}% indent
{\bfseries}% head font
{}% punctuation after head
{\newline}% space after head (has to be space or dimension!)
{\thmname{#1}\thmnumber{ #2} \thmnote{{\normalfont\itshape #3}}}% head spec

% 
\newtheoremstyle{bground}% Name
{}% space above
{}% space below
{\normalfont}% body font
{}% indent
{\bfseries}% head font
{}% punctuation after head
{\newline}% space after head (has to be space or dimension!)
{\thmname{#1}\thmnumber{ #2}\thmnote{: {\normalfont\itshape\bfseries #3}}}% head spec


\theoremstyle{example}
\newtheorem{example}{Example}[chapter]

\theoremstyle{bground}
\newtheorem{bground}{Background}
\renewcommand{\thebground}{\Alph{bground}}

\usetikzlibrary{decorations.pathmorphing}
\pgfdeclaredecoration{complete zigzag}{initial}{
  \state{initial}[
    width=+0pt,
    next state=half up,
    persistent precomputation={\pgfmathsetmacro\matchinglength{
          \pgfdecoratedinputsegmentlength / int(\pgfdecoratedinputsegmentlength/\pgfdecorationsegmentlength)}
        \setlength{\pgfdecorationsegmentlength}{\matchinglength pt}
      }] {}
  \state{half up}[
    width=+.25\pgfdecorationsegmentlength,
    next state=big down]
  {\pgfpathlineto{\pgfqpoint{.25\pgfdecorationsegmentlength}{\pgfdecorationsegmentamplitude}}
  }
  \state{big down}[switch if less than=+.5\pgfdecorationsegmentlength to center finish,
    width=+.5\pgfdecorationsegmentlength,
    next state=big up]
  {
    \pgfpathlineto{\pgfqpoint{.5\pgfdecorationsegmentlength}{-\pgfdecorationsegmentamplitude}}
  }
  \state{big up}[switch if less than=+.5\pgfdecorationsegmentlength to center finish,
    width=+.5\pgfdecorationsegmentlength,
    next state=big down]
  {
    \pgfpathlineto{\pgfqpoint{.5\pgfdecorationsegmentlength}{\pgfdecorationsegmentamplitude}}
  }
  \state{center finish}[width=0pt, next state=final]{
  }
  \state{final}
  {
    \pgfpathlineto{\pgfpointdecoratedpathlast}
  }
}

\tcolorboxenvironment{example}{
  enhanced,
  breakable,
  height fixed for=middle,
  % overlay first={%
  %     \path[font=\small\itshape] (frame.south) node (cont) {(continued)};
  %     \begin{scope}[decoration={complete zigzag,amplitude=0.5mm}]
  %       \path[fill=Gray!10]  decorate {([xshift=1.2pt]frame.south west) -- (cont.west)} --++
  %       (0,0.5ex) -| cycle
  %       decorate {([xshift=-1.2pt]frame.south east) -- (cont.east)} --++
  %       (0,0.5ex) -| cycle;
  %       \path[fill=white]
  %       decorate {([xshift=1.2pt]frame.south west) -- (cont.west)} --++
  %       (0,-0.5ex) -| cycle
  %       decorate {([xshift=-1.2pt]frame.south east) -- (cont.east)} --++
  %       (0,-0.5ex) -| cycle;
  %       \draw[thick,Black,decorate] ([xshift=1.2pt]frame.south west) -- (cont.west);
  %       \draw[thick,Black,decorate] ([xshift=-1.2pt]frame.south east) -- (cont.east);
  %     \end{scope}
  %   },
  % overlay middle={%
  %     \path[font=\small\itshape] (frame.south) node (cont) {(continued)};
  %     \begin{scope}[decoration={complete zigzag,amplitude=0.5mm}]
  %       \path[fill=Gray!10]  decorate {([xshift=1.2pt]frame.south west) -- (cont.west)} --++
  %       (0,0.5ex) -| cycle
  %       decorate {([xshift=-1.2pt]frame.south east) -- (cont.east)} --++
  %       (0,0.5ex) -| cycle;
  %       \path[fill=white]
  %       decorate {([xshift=1.2pt]frame.south west) -- (cont.west)} --++
  %       (0,-0.5ex) -| cycle
  %       decorate {([xshift=-1.2pt]frame.south east) -- (cont.east)} --++
  %       (0,-0.5ex) -| cycle;
  %       \draw[thick,Black,decorate] ([xshift=1.2pt]frame.south west) -- (cont.west);
  %       \draw[thick,Black,decorate] ([xshift=-1.2pt]frame.south east) -- (cont.east);
  %     \end{scope}
  %     \path[font=\small\itshape] (frame.north) node (thm) {Example \theexample\ continued};
  %     \begin{scope}[decoration={complete zigzag,amplitude=0.5mm}]
  %       \path[fill=Gray!10]  decorate {([xshift=1.2pt]frame.north west) -- (thm.west)} --++
  %       (0,-0.5ex) -| cycle
  %       decorate {([xshift=-1.2pt]frame.north east) -- (thm.east)} --++
  %       (0,-0.5ex) -| cycle;
  %       \path[fill=white]
  %       decorate {([xshift=1.2pt]frame.north west) -- (thm.west)} --++
  %       (0,0.5ex) -| cycle
  %       decorate {([xshift=-1.2pt]frame.north east) -- (thm.east)} --++
  %       (0,0.5ex) -| cycle;
  %       \draw[thick,Black,decorate] ([xshift=1.2pt]frame.north west) -- (thm.west);
  %       \draw[thick,Black,decorate] ([xshift=-1.2pt]frame.north east) -- (thm.east);
  %     \end{scope}
  %   },
  % overlay last={%
  %     \path[font=\small\itshape] (frame.north) node (thm) {Example \theexample\ continued};
  %     \begin{scope}[decoration={complete zigzag,amplitude=0.5mm}]
  %       \path[fill=Gray!10]  decorate {([xshift=1.2pt]frame.north west) -- (thm.west)} --++
  %       (0,-0.5ex) -| cycle
  %       decorate {([xshift=-1.2pt]frame.north east) -- (thm.east)} --++
  %       (0,-0.5ex) -| cycle;
  %       \path[fill=white]
  %       decorate {([xshift=1.2pt]frame.north west) -- (thm.west)} --++
  %       (0,0.5ex) -| cycle
  %       decorate {([xshift=-1.2pt]frame.north east) -- (thm.east)} --++
  %       (0,0.5ex) -| cycle;
  %       \draw[thick,Black,decorate] ([xshift=1.2pt]frame.north west) -- (thm.west);
  %       \draw[thick,Black,decorate] ([xshift=-1.2pt]frame.north east) -- (thm.east);
  %     \end{scope}
  %   },
  colback=Gray!10,
  colframe={Black},
}

\tcolorboxenvironment{bground}{
  enhanced,
  breakable,
  % opacityback=0,
  colback=Gray!5,
  colframe={Gray},
}

% To have tikz nodes with the same color as the box background
\AtBeginEnvironment{example}{
  \tikzset{
    action/.append style = {fill=Gray!10},
    task/.append style = {fill=Gray!10},
  }
}
\AtBeginEnvironment{bground}{
  \tikzset{
    action/.append style = {fill=Gray!5},
    task/.append style = {fill=Gray!5},
  }
}



% #endregion

%%%%%%%%%%%%%%%%%%%%%%%%%%%%%%
% #region PDF/Links settings %
%%%%%%%%%%%%%%%%%%%%%%%%%%%%%%

% Links in pdf
\definecolor{linkcol}{rgb}{0,0,0.4}
\definecolor{citecol}{rgb}{0.5,0,0}
\definecolor{linkcol}{rgb}{0,0,0}
\definecolor{citecol}{rgb}{0,0,0}
% Change this to change the informations included in the pdf file
\hypersetup
{
  bookmarksopen=true,
  pdftitle="Learning by Interaction for Deliberate Acting in Robotics",
  pdfauthor="Philippe Herail", %auteur du document
  pdfsubject="These", %sujet du document
  %pdftoolbar=false, %barre d'outils non visible
  pdfmenubar=true, %barre de menu visible
  pdfhighlight=/O, %effet d'un clic sur un lien hypertexte
  colorlinks=true, %couleurs sur les liens hypertextes
  pdfpagemode=UseOutlines, %aucun mode de page
  pdfpagelayout=SinglePage, %ouverture en simple page
  pdffitwindow=true, %pages ouvertes entierement dans toute la fenetre
  linkcolor=linkcol, %couleur des liens hypertextes internes
  citecolor=citecol, %couleur des liens pour les citations
  urlcolor=linkcol %couleur des liens pour les url
}

% #endregion

%%%%%%%%%%%%%%%%%%%%%%%%%
% #region Header/Footer %
%%%%%%%%%%%%%%%%%%%%%%%%%


%%% Fancy Header %%%%%%%%%%%%%%%%%%%%%%%%%%%%%%%%%%%%%%%%%%%%%%%%%%%%%%%%%%%%%%%%%%
% Fancy Header Style Options

\fancypagestyle{fancy}{
  % Sets fancy header and footer
  \fancyfoot{}                            % Delete current footer settings

  %\renewcommand{\chaptermark}[1]{         % Lower Case Chapter marker style
  %  \markboth{\chaptername\ \thechapter.\ #1}}{}} %

  %\renewcommand{\sectionmark}[1]{         % Lower case Section marker style
  %  \markright{\thesection.\ #1}}         %

  \fancyhead[LE,RO]{\bfseries\thepage}    % Page number (boldface) in left on even
  % pages and right on odd pages
  \fancyhead[RE]{\bfseries\nouppercase{\leftmark}}      % Chapter in the right on even pages
  \fancyhead[LO]{\bfseries\nouppercase{\rightmark}}     % Section in the left on odd pages

  \let\headruleORIG\headrule
  \renewcommand{\headrule}{\color{black} \headruleORIG}
  \renewcommand{\headrulewidth}{1.0pt}
}

\fancypagestyle{plain}{
  \fancyhead{}
  % \fancyfoot[CO,CE]{\thepage}
  \renewcommand{\headrulewidth}{0pt}
}
%%% Clear Header %%%%%%%%%%%%%%%%%%%%%%%%%%%%%%%%%%%%%%%%%%%%%%%%%%%%%%%%%%%%%%%%%%
% Clear Header Style on the Last Empty Odd pages
\makeatletter

\def\cleardoublepage{\clearpage\if@twoside \ifodd\c@page\else%
      \hbox{}%
      \thispagestyle{empty}%              % Empty header styles
      \newpage%
      \if@twocolumn\hbox{}\newpage\fi\fi\fi}

\makeatother

% #endregion

\renewcommand{\baselinestretch}{1.05}

%%%%%%%%%%%%%%%%%%%%%%%%%%%%%%%%%%%%
% #region MISC FORMATTING COMMANDS %
%%%%%%%%%%%%%%%%%%%%%%%%%%%%%%%%%%%%

%%%%%%%%%%%%%%%%%%%%%%%%%%%%%%%%%%%%%%%%%%%%%%%%%%%%%%%%%%%%%%%%%%%%%%%%%%%%%%% 
% Prints your review date and 'Draft Version' (From Josullvn, CS, CMU)
\newcommand{\reviewtimetoday}[2]{\special{!userdict begin
/bop-hook{gsave 20 710 translate 45 rotate 0.8 setgray
/Times-Roman findfont 12 scalefont setfont 0 0   moveto (#1) show
0 -12 moveto (#2) show grestore}def end}}
% You can turn on or off this option.
% \reviewtimetoday{\today}{Draft Version}
%%%%%%%%%%%%%%%%%%%%%%%%%%%%%%%%%%%%%%%%%%%%%%%%%%%%%%%%%%%%%%%%%%%%%%%%%%%%%%% 



% #endregion

%%%%%%%%%%%%%%%%%%%%%%%%%%%%%%%%%%
% #region TOC/MINITOC Formatting %
%%%%%%%%%%%%%%%%%%%%%%%%%%%%%%%%%%

\etocbookstyle
% \etocstandardlines
\newlength\tocrulewidth
\setlength{\tocrulewidth}{1.5pt}
\setcounter{secnumdepth}{3}
\setcounter{tocdepth}{3}
\etocsetnexttocdepth{2}

% #endregion

% Commands to format MAX-SMT constants representing top or bottom args
\newcommand{\argtop}[1]{{\ensuremath{\bar{#1}}}}
\newcommand{\argbot}[1]{{\ensuremath{\ubar{#1}}}}

\newcommand{\ptype}[1]{\textttpit{#1}} % for planing params types
\newcommand{\pvar}[1]{\ensuremath{\operatorname{?\mathit{#1}}}} % for planning params vars (add subscript outside)
\newcommand{\pconst}[1]{\ensuremath{\operatorname{\mathit{#1}}}} % for planning params constants (add subscript outside)
\newcommand{\poper}[1]{{\textttp{#1}}} % for planning params operators (action, tasks)
\newcommand{\psf}[1]{{\textttp{#1}}} % for planning state variables

\newcommand{\htnmaker}{\textsc{HTN-Maker}}
\newcommand{\htnlearn}{\textsc{HTNLearn}}
\newcommand{\wordhtn}{\textsc{Word2HTN}}
\newcommand{\wordvec}{\textsc{Word2Vec}}
\newcommand{\bnfsym}[1]{\ensuremath{\langle\texttt{#1}\rangle}}

%%%%%%%%%%%%%%%%%%%%%%%%
%	  TODO commands    %
%%%%%%%%%%%%%%%%%%%%%%%%

\ifdefined\notodo
  \newcommand{\todomarker}[2][]{}
  \newcommand{\arthur}[1]{}
  \newcommand{\philippe}[1]{}
  \newcommand{\pcomment}[2][]{}
  \newcommand{\pinline}[1]{}
  \newcommand{\acomment}[1]{}
\else
  \setlength{\marginparwidth}{0.85in}
  \newcommand{\todomarker}[2][]{\todo[inline,size=\normalsize,#1]{TODO #2}}
  \newcommand{\arthur}[1]{{\color{blue} \textbf{#1}}}
  \newcommand{\philippe}[1]{{\color{ForestGreen} \textbf{#1}}}
  \newcommand{\pcomment}[2][]{\todo[backgroundcolor=ForestGreen!25,linecolor=ForestGreen,caption={P},prepend,size=\tiny,#1]{#2}}
  \newcommand{\pinline}[1]{\todo[backgroundcolor=ForestGreen!25,author={Philippe},inline,size=\normalsize]{#1}}
  \newcommand{\acomment}[1]{\todo[linecolor=blue,backgroundcolor=blue!25,bordercolor=blue,caption={A},prepend,size=\tiny]{#1}}
\fi


%%%%%%%%%%%%%%%%%%%%%%%%%%
% #region CUSTOM SYMBOLS %
%%%%%%%%%%%%%%%%%%%%%%%%%%

\DeclareMathSymbol{\mlq}{\mathord}{operators}{``}
\DeclareMathSymbol{\mrq}{\mathord}{operators}{`'}

\renewcommand{\epsilon}{\varepsilon}

\newcommand{\red}[1]{{\color{red}#1}} 
\newcommand{\blue}[1]{{\color{blue}#1}}
\newcommand{\magen}[1]{{\color{magenta}#1}}
\newcommand{\bxi}{\boldsymbol{\xi}}
\newcommand{\x}{\boldsymbol{x}}
\newcommand{\X}{\boldsymbol{X}}
\newcommand{\q}{\boldsymbol{q}}
\newcommand{\e}{\boldsymbol{e}}
\newcommand{\f}{\boldsymbol{f}}
\newcommand{\g}{\boldsymbol{g}}
\renewcommand{\k}{\boldsymbol{k}}
\newcommand{\h}{\boldsymbol{h}}
\renewcommand{\r}{\boldsymbol{r}}
\renewcommand{\u}{\boldsymbol{u}}
\renewcommand{\a}{\boldsymbol{a}}
\newcommand{\p}{\boldsymbol{p}}
\newcommand{\bPhi}{\boldsymbol{\Phi}}
\newcommand{\bPi}{\boldsymbol{\Pi}}
\newcommand{\bPixi}{\boldsymbol{\Pi_{\xi}}}
\newcommand{\bTheta}{\boldsymbol{\Theta}}
\newcommand{\W}{\boldsymbol{W}}
\newcommand{\Rq}{\boldsymbol{r_q}}
\newcommand{\Ru}{\boldsymbol{r_u}}
\newcommand{\U}{\boldsymbol{U}}
\newcommand{\bW}{\boldsymbol{W}}
\newcommand{\boldeta}{\boldsymbol{\eta}}
\newcommand{\boldmu}{\boldsymbol{\mu}}

% #endregion

%%%%%%%%%%%%%%%%%%%%%%%%%%%%%%%%%
% #region CUSTOM MATH OPERATORS %
%%%%%%%%%%%%%%%%%%%%%%%%%%%%%%%%%

\DeclareMathOperator*{\argmax}{arg\,max}%
\DeclareMathOperator*{\argmin}{arg\,min}%
\DeclareMathOperator{\cost}{cost}%
\DeclareMathOperator{\head}{head}
\DeclareMathOperator{\name}{name}
\DeclareMathOperator{\pre}{pre}
\DeclareMathOperator{\post}{post}
\DeclareMathOperator{\eff}{eff}

\DeclareMathOperator{\decs}{Decs}
\DeclareMathOperator{\vdecs}{{VDecs}}

\DeclareMathOperator*{\args}{args}%
\DeclareMathOperator{\argt}{arg}%
\DeclareMathOperator{\subs}{subtasks}
\DeclareMathOperator*{\tsym}{sym}%
\DeclareMathOperator{\prims}{Prims}%
\DeclareMathOperator{\win}{win}%
\DeclareMathOperator{\gnd}{gnd}%
\DeclareMathOperator{\inst}{inst}%
\DeclareMathOperator*{\soft}{\textit{\textsc{Soft}}}%
\DeclareMathOperator*{\hard}{\textit{\textsc{Hard}}}%

\DeclareMathOperator{\zip}{zip}

% Functions used in MAX-SMT formulations
\DeclareMathOperator*{\pgroup}{\textsc{PGroup}}%
\DeclareMathOperator*{\isgroupcount}{\textsc{IsGroupCount}}%
\DeclareMathOperator*{\grounding}{\textsc{Gnd}}%

\DeclareMathOperator*{\gndparam}{\textsc{GndParam}}%
\DeclareMathOperator*{\gndgroup}{\textsc{GndGroup}}%

% Two dots instead of three
\newcommand{\slice}{\mathinner{{\ldotp}{\ldotp}}}
% Two dots instead of three
\newcommand{\sliceeq}{\mathinner{{\ldotp}{\ldotp}{=}}}

% Big eq symbol, like bigcup
\DeclareMathOperator*{\bigeq}{\scalerel*[15pt]{=}{\sum}}


% #endregion

