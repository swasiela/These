% Choose the language of your thesis passing 'french' or 'english' as
% \documentclass option.
% Note1: The 'page de garde' will always be written in French.
% Note2: You will have an error if you change the language of the document and
%        compile it without cleaning the auxiliary files. Compiling it again
%        should solve the problem.
\PassOptionsToPackage{main=UKenglish}{babel}
\documentclass[UKenglish,a4paper,12pt,twoside]{StyleThese}

\ifdefined\isfinal
  \def\noexternal{}
  \def\notodo{}
\else
\fi

\ifdefined\watermark
  \def\notodo{}
\else
\fi

% This file contains all the included packages, in order to statisfy ordering and similar contraints.
\usepackage{shellesc} % Solve some issues with shell escaping and luatex
\usepackage{transparent}

\usepackage[dvipsnames]{xcolor}  % Colors for text, and new color definitions

\usepackage[french]{babel} % French as second language, used if main language is not this one
\usepackage{csquotes}

\usepackage{datetime}
\usepackage{subcaption}
\usepackage{graphicx}
\usepackage{svg}

\usepackage{tikz}

\usepackage[edges]{forest} % Nice & easy tree structures using tikz

\usepackage[figuresright]{rotating} % Sideways for figures & tables
\usepackage{placeins}


\usepackage{standalone} % Useful for working on specific tikz pictures

%%%%%%%%%%%%%%%%%%%%%%%%%%%%%%%%%%%%%%
% #region ALGORITHM RELATED PACKAGES %
%%%%%%%%%%%%%%%%%%%%%%%%%%%%%%%%%%%%%%


\usepackage{minted}
\usepackage[chapter]{algorithm}
\usepackage[indLines=true,commentColor=black]{algpseudocodex}




% #endregion

\usepackage{fontspec} % For specific font setup


%%%%%%%%%%%%%%%%%%%%%%%%%%%%%%%%%%%%%
% #region THEOREM-LIKE ENVIRONMENTS %
%%%%%%%%%%%%%%%%%%%%%%%%%%%%%%%%%%%%%

\usepackage{amsthm}
\usepackage[breakable,skins,theorems]{tcolorbox}

% #endregion


%%%%%%%%%%%%%%%%%%%%%%%%%%
% #region TABLES & RULES %
%%%%%%%%%%%%%%%%%%%%%%%%%%

\usepackage{afterpage}

\usepackage{longtable}
\usepackage{booktabs}
\usepackage{tablefootnote}

\usepackage{tabularx}
\usepackage{makecell}
\usepackage{multirow}
\usepackage{float}
\usepackage{hhline}


% #endregion



%%%%%%%%%%%%%%%%%%%%%%%%%%%%%%%%%%%%
% #region HYPERLINK & PDF SETTINGS %
%%%%%%%%%%%%%%%%%%%%%%%%%%%%%%%%%%%%

\usepackage{pdflscape}
\usepackage{pdfpages}
\usepackage[hyperindex=true]{hyperref}
\usepackage{eso-pic}

\ifdefined\annotateprint% Used to have easy make rules to build the file
  \usepackage[left=2in,right=2in,top=1.5in,bottom=1.5in,includehead,headheight=26pt]{geometry}
\else
  \ifdefined\finalprint
    \usepackage[inner=1.1in,right=0.9in,top=1.3in,bottom=1in,headheight=26pt]{geometry}
  \else
    \usepackage[left=1in,right=1in,top=1.3in,bottom=1in,headheight=26pt]{geometry}
  \fi
\fi


% #endregion

\usepackage{etoc} % Mini toc

\usepackage[style=ext-alphabetic,backref=true,sortcites=true,
  doi=false,isbn=false,url=false,eprint=false, maxbibnames=99]{biblatex} %Bib formatting

\usepackage{fancyhdr} % Fancy Header and Footer

\ifdefined\notodo
  \usepackage[disable]{todonotes}
\else
  \usepackage[colorinlistoftodos,prependcaption,textsize=tiny]{todonotes}
\fi
\usepackage{xargs}   % Use more than one optional parameter in a new commands

%%%%%%%%%%%%%%%%%%%%%%%%%
% #region Maths/Symbols %
%%%%%%%%%%%%%%%%%%%%%%%%%

\usepackage{pifont}
\usepackage{mathtools}
\usepackage{blkarray, bigstrut}
\usepackage{amssymb}
\usepackage{accents}
\usepackage{scalerel}
\usepackage{delarray}
\usepackage{array}

% #endregion

\usepackage[bottom,perpage]{footmisc}

\usepackage{listings}
\usepackage[normalem]{ulem}

% \ifdefined\noxindy% Used to have easy make rules to build the file
%   \usepackage[acronyms,sort=standard,toc]{glossaries} % Do not use xindy to limite dependencies
% \else
%   \usepackage[acronyms,xindy,sort=standard,toc]{glossaries}
% \fi

% Need to be loaded after many other packages
\usepackage{cleveref} % For nice multiple refs in a single command
% This file contains different settings pertianing to formatting, as well as the newcommand definitions

%%%%%%%%%%%%%%%%%%%%%%%%%%%%%
% #region COLOR DEFINITIONS %
%%%%%%%%%%%%%%%%%%%%%%%%%%%%%

\definecolor{IBMBlue}{HTML}{648FFF}
\definecolor{IBMPurple}{HTML}{785EF0}
\definecolor{IBMMagenta}{HTML}{DC267F}
\definecolor{IBMOrange}{HTML}{FE6100}
\definecolor{IBMYellow}{HTML}{FFB000}
\colorlet{Accent5}{IBMYellow}
\colorlet{Accent2}{IBMBlue}
\colorlet{Accent3}{IBMMagenta}
\colorlet{Accent4}{IBMPurple}
\colorlet{Accent1}{IBMOrange}

\definecolor{ThreeRed}{HTML}{D41159}
\definecolor{ThreeOrange}{HTML}{FFB000}
\definecolor{ThreeGreen}{HTML}{77AB02}

% #endregion

%%%%%%%%%%%%%%%%%%%%%
% #region Algorithm %
%%%%%%%%%%%%%%%%%%%%%

\algrenewcommand\algorithmicrequire{\textbf{Input:}}
\algrenewcommand\algorithmicensure{\textbf{Output:}}

% Define new blocks, see https://github.com/chrmatt/algpseudocodex/issues/3
\algnewcommand\algorithmicsoptional{\textbf{optional}}%

\makeatletter
\algdef{SE}[OPTIONAL]{Optional}{EndOptional}[1]{\algpx@startIndent\algpx@startCodeCommand\algorithmicsoptional}{\algpx@endIndent\algpx@startCodeCommand\algorithmicend\ \algorithmicsoptional}%

\ifbool{algpx@noEnd}{%
  \algtext*{EndOptional}%
  %
  % end indent line after (not before), to get correct y position for multiline text in last command
  \apptocmd{\EndOptional}{\algpx@endIndent}{}{}%
}{}%

\pretocmd{\Optional}{\algpx@endCodeCommand}{}{}

    % for end commands that may not be printed, tell endCodeCommand whether we are using noEnd
    \ifbool{algpx@noEnd}{%
    \pretocmd{\EndOptional}{\algpx@endCodeCommand[1]}{}{}%
}{%
\pretocmd{\EndOptional}{\algpx@endCodeCommand[0]}{}{}%
}%
\makeatother

% #endregion

%%%%%%%%%%%%%%%%%
% #region Fonts %
%%%%%%%%%%%%%%%%%

% Find a way to define a new family and use it twice
% \newfontfamily{\lmodernsc}{Latin Modern Roman}[
%   SmallCapsFont = Latin Modern Roman Caps,
%   UprightFeatures = {SmallCapsFeatures = {}}, % Suppresses warning about missing smcp feature.
%   BoldFeatures = {SmallCapsFont = {CMU Serif Bold}},
%   ItalicFeatures = {SmallCapsFont = {CMU Serif Roman Slanted}},
%   BoldItalicFeatures = {SmallCapsFont = {CMU Serif Bold Slanted}},
%   SmallCapsFeatures={Letters=SmallCaps}]
\setmainfont{Latin Modern Roman}[
  SmallCapsFont = Latin Modern Roman Caps,
  UprightFeatures = {SmallCapsFeatures = {}}, % Suppresses warning about missing smcp feature.
  BoldFeatures = {SmallCapsFont = {CMU Serif Bold}},
  ItalicFeatures = {SmallCapsFont = {CMU Serif Roman Slanted}},
  BoldItalicFeatures = {SmallCapsFont = {CMU Serif Bold Slanted}},
  SmallCapsFeatures={Letters=SmallCaps}]
% Loading font and renaming for template compat
\newfontfamily{\pbkcompat}{TeX Gyre Bonum}[NFSSFamily=pbk]
\newfontfamily{\cmrcompat}{Latin Modern Roman}[
  NFSSFamily=cmr,
  SmallCapsFont = Latin Modern Roman Caps,
  UprightFeatures = {SmallCapsFeatures = {}}, % Suppresses warning about missing smcp feature.
  BoldFeatures = {SmallCapsFont = {CMU Serif Bold}},
  ItalicFeatures = {SmallCapsFont = {CMU Serif Roman Slanted}},
  BoldItalicFeatures = {SmallCapsFont = {CMU Serif Bold Slanted}},
  SmallCapsFeatures={Letters=SmallCaps}]
\newfontfamily{\pcrcompat}{TeX Gyre Cursor}[NFSSFamily=pcr]
\setmonofont{Latin Modern Mono}

\newfontfamily\ttprop{Latin Modern Mono Prop}
\DeclareTextFontCommand{\textttpit}{\ttprop\itshape}
\DeclareTextFontCommand{\textttp}{\ttprop}

% #endregion

%%%%%%%%%%%%%%%%%%%%%%%%%%%%%%%%%%%%%
% #region Theorem-like environments %
%%%%%%%%%%%%%%%%%%%%%%%%%%%%%%%%%%%%%


% 'Plain' theorems
\newtheorem{theorem}{Theorem}[chapter]
\newtheorem{corollary}{Corollary}[theorem]
\newtheorem{lemma}[theorem]{Lemma}

\theoremstyle{definition}
\newtheorem{definition}{Definition}[chapter]

\theoremstyle{remark}
\newtheorem*{remark}{Remark}

% 'Boxed' Theorems

% Styling for examples
\newtheoremstyle{example}% Name
{}% space above
{}% space below
{\normalfont}% body font
{}% indent
{\bfseries}% head font
{}% punctuation after head
{\newline}% space after head (has to be space or dimension!)
{\thmname{#1}\thmnumber{ #2} \thmnote{{\normalfont\itshape #3}}}% head spec

% 
\newtheoremstyle{bground}% Name
{}% space above
{}% space below
{\normalfont}% body font
{}% indent
{\bfseries}% head font
{}% punctuation after head
{\newline}% space after head (has to be space or dimension!)
{\thmname{#1}\thmnumber{ #2}\thmnote{: {\normalfont\itshape\bfseries #3}}}% head spec


\theoremstyle{example}
\newtheorem{example}{Example}[chapter]

\theoremstyle{bground}
\newtheorem{bground}{Background}
\renewcommand{\thebground}{\Alph{bground}}

\usetikzlibrary{decorations.pathmorphing}
\pgfdeclaredecoration{complete zigzag}{initial}{
  \state{initial}[
    width=+0pt,
    next state=half up,
    persistent precomputation={\pgfmathsetmacro\matchinglength{
          \pgfdecoratedinputsegmentlength / int(\pgfdecoratedinputsegmentlength/\pgfdecorationsegmentlength)}
        \setlength{\pgfdecorationsegmentlength}{\matchinglength pt}
      }] {}
  \state{half up}[
    width=+.25\pgfdecorationsegmentlength,
    next state=big down]
  {\pgfpathlineto{\pgfqpoint{.25\pgfdecorationsegmentlength}{\pgfdecorationsegmentamplitude}}
  }
  \state{big down}[switch if less than=+.5\pgfdecorationsegmentlength to center finish,
    width=+.5\pgfdecorationsegmentlength,
    next state=big up]
  {
    \pgfpathlineto{\pgfqpoint{.5\pgfdecorationsegmentlength}{-\pgfdecorationsegmentamplitude}}
  }
  \state{big up}[switch if less than=+.5\pgfdecorationsegmentlength to center finish,
    width=+.5\pgfdecorationsegmentlength,
    next state=big down]
  {
    \pgfpathlineto{\pgfqpoint{.5\pgfdecorationsegmentlength}{\pgfdecorationsegmentamplitude}}
  }
  \state{center finish}[width=0pt, next state=final]{
  }
  \state{final}
  {
    \pgfpathlineto{\pgfpointdecoratedpathlast}
  }
}

\tcolorboxenvironment{example}{
  enhanced,
  breakable,
  height fixed for=middle,
  % overlay first={%
  %     \path[font=\small\itshape] (frame.south) node (cont) {(continued)};
  %     \begin{scope}[decoration={complete zigzag,amplitude=0.5mm}]
  %       \path[fill=Gray!10]  decorate {([xshift=1.2pt]frame.south west) -- (cont.west)} --++
  %       (0,0.5ex) -| cycle
  %       decorate {([xshift=-1.2pt]frame.south east) -- (cont.east)} --++
  %       (0,0.5ex) -| cycle;
  %       \path[fill=white]
  %       decorate {([xshift=1.2pt]frame.south west) -- (cont.west)} --++
  %       (0,-0.5ex) -| cycle
  %       decorate {([xshift=-1.2pt]frame.south east) -- (cont.east)} --++
  %       (0,-0.5ex) -| cycle;
  %       \draw[thick,Black,decorate] ([xshift=1.2pt]frame.south west) -- (cont.west);
  %       \draw[thick,Black,decorate] ([xshift=-1.2pt]frame.south east) -- (cont.east);
  %     \end{scope}
  %   },
  % overlay middle={%
  %     \path[font=\small\itshape] (frame.south) node (cont) {(continued)};
  %     \begin{scope}[decoration={complete zigzag,amplitude=0.5mm}]
  %       \path[fill=Gray!10]  decorate {([xshift=1.2pt]frame.south west) -- (cont.west)} --++
  %       (0,0.5ex) -| cycle
  %       decorate {([xshift=-1.2pt]frame.south east) -- (cont.east)} --++
  %       (0,0.5ex) -| cycle;
  %       \path[fill=white]
  %       decorate {([xshift=1.2pt]frame.south west) -- (cont.west)} --++
  %       (0,-0.5ex) -| cycle
  %       decorate {([xshift=-1.2pt]frame.south east) -- (cont.east)} --++
  %       (0,-0.5ex) -| cycle;
  %       \draw[thick,Black,decorate] ([xshift=1.2pt]frame.south west) -- (cont.west);
  %       \draw[thick,Black,decorate] ([xshift=-1.2pt]frame.south east) -- (cont.east);
  %     \end{scope}
  %     \path[font=\small\itshape] (frame.north) node (thm) {Example \theexample\ continued};
  %     \begin{scope}[decoration={complete zigzag,amplitude=0.5mm}]
  %       \path[fill=Gray!10]  decorate {([xshift=1.2pt]frame.north west) -- (thm.west)} --++
  %       (0,-0.5ex) -| cycle
  %       decorate {([xshift=-1.2pt]frame.north east) -- (thm.east)} --++
  %       (0,-0.5ex) -| cycle;
  %       \path[fill=white]
  %       decorate {([xshift=1.2pt]frame.north west) -- (thm.west)} --++
  %       (0,0.5ex) -| cycle
  %       decorate {([xshift=-1.2pt]frame.north east) -- (thm.east)} --++
  %       (0,0.5ex) -| cycle;
  %       \draw[thick,Black,decorate] ([xshift=1.2pt]frame.north west) -- (thm.west);
  %       \draw[thick,Black,decorate] ([xshift=-1.2pt]frame.north east) -- (thm.east);
  %     \end{scope}
  %   },
  % overlay last={%
  %     \path[font=\small\itshape] (frame.north) node (thm) {Example \theexample\ continued};
  %     \begin{scope}[decoration={complete zigzag,amplitude=0.5mm}]
  %       \path[fill=Gray!10]  decorate {([xshift=1.2pt]frame.north west) -- (thm.west)} --++
  %       (0,-0.5ex) -| cycle
  %       decorate {([xshift=-1.2pt]frame.north east) -- (thm.east)} --++
  %       (0,-0.5ex) -| cycle;
  %       \path[fill=white]
  %       decorate {([xshift=1.2pt]frame.north west) -- (thm.west)} --++
  %       (0,0.5ex) -| cycle
  %       decorate {([xshift=-1.2pt]frame.north east) -- (thm.east)} --++
  %       (0,0.5ex) -| cycle;
  %       \draw[thick,Black,decorate] ([xshift=1.2pt]frame.north west) -- (thm.west);
  %       \draw[thick,Black,decorate] ([xshift=-1.2pt]frame.north east) -- (thm.east);
  %     \end{scope}
  %   },
  colback=Gray!10,
  colframe={Black},
}

\tcolorboxenvironment{bground}{
  enhanced,
  breakable,
  % opacityback=0,
  colback=Gray!5,
  colframe={Gray},
}

% To have tikz nodes with the same color as the box background
\AtBeginEnvironment{example}{
  \tikzset{
    action/.append style = {fill=Gray!10},
    task/.append style = {fill=Gray!10},
  }
}
\AtBeginEnvironment{bground}{
  \tikzset{
    action/.append style = {fill=Gray!5},
    task/.append style = {fill=Gray!5},
  }
}



% #endregion

%%%%%%%%%%%%%%%%%%%%%%%%%%%%%%
% #region PDF/Links settings %
%%%%%%%%%%%%%%%%%%%%%%%%%%%%%%

% Links in pdf
\definecolor{linkcol}{rgb}{0,0,0.4}
\definecolor{citecol}{rgb}{0.5,0,0}
\definecolor{linkcol}{rgb}{0,0,0}
\definecolor{citecol}{rgb}{0,0,0}
% Change this to change the informations included in the pdf file
\hypersetup
{
  bookmarksopen=true,
  pdftitle="Learning by Interaction for Deliberate Acting in Robotics",
  pdfauthor="Philippe Herail", %auteur du document
  pdfsubject="These", %sujet du document
  %pdftoolbar=false, %barre d'outils non visible
  pdfmenubar=true, %barre de menu visible
  pdfhighlight=/O, %effet d'un clic sur un lien hypertexte
  colorlinks=true, %couleurs sur les liens hypertextes
  pdfpagemode=UseOutlines, %aucun mode de page
  pdfpagelayout=SinglePage, %ouverture en simple page
  pdffitwindow=true, %pages ouvertes entierement dans toute la fenetre
  linkcolor=linkcol, %couleur des liens hypertextes internes
  citecolor=citecol, %couleur des liens pour les citations
  urlcolor=linkcol %couleur des liens pour les url
}

% #endregion

%%%%%%%%%%%%%%%%%%%%%%%%%
% #region Header/Footer %
%%%%%%%%%%%%%%%%%%%%%%%%%


%%% Fancy Header %%%%%%%%%%%%%%%%%%%%%%%%%%%%%%%%%%%%%%%%%%%%%%%%%%%%%%%%%%%%%%%%%%
% Fancy Header Style Options

\fancypagestyle{fancy}{
  % Sets fancy header and footer
  \fancyfoot{}                            % Delete current footer settings

  %\renewcommand{\chaptermark}[1]{         % Lower Case Chapter marker style
  %  \markboth{\chaptername\ \thechapter.\ #1}}{}} %

  %\renewcommand{\sectionmark}[1]{         % Lower case Section marker style
  %  \markright{\thesection.\ #1}}         %

  \fancyhead[LE,RO]{\bfseries\thepage}    % Page number (boldface) in left on even
  % pages and right on odd pages
  \fancyhead[RE]{\bfseries\nouppercase{\leftmark}}      % Chapter in the right on even pages
  \fancyhead[LO]{\bfseries\nouppercase{\rightmark}}     % Section in the left on odd pages

  \let\headruleORIG\headrule
  \renewcommand{\headrule}{\color{black} \headruleORIG}
  \renewcommand{\headrulewidth}{1.0pt}
}

\fancypagestyle{plain}{
  \fancyhead{}
  % \fancyfoot[CO,CE]{\thepage}
  \renewcommand{\headrulewidth}{0pt}
}
%%% Clear Header %%%%%%%%%%%%%%%%%%%%%%%%%%%%%%%%%%%%%%%%%%%%%%%%%%%%%%%%%%%%%%%%%%
% Clear Header Style on the Last Empty Odd pages
\makeatletter

\def\cleardoublepage{\clearpage\if@twoside \ifodd\c@page\else%
      \hbox{}%
      \thispagestyle{empty}%              % Empty header styles
      \newpage%
      \if@twocolumn\hbox{}\newpage\fi\fi\fi}

\makeatother

% #endregion

\renewcommand{\baselinestretch}{1.05}

%%%%%%%%%%%%%%%%%%%%%%%%%%%%%%%%%%%%
% #region MISC FORMATTING COMMANDS %
%%%%%%%%%%%%%%%%%%%%%%%%%%%%%%%%%%%%

%%%%%%%%%%%%%%%%%%%%%%%%%%%%%%%%%%%%%%%%%%%%%%%%%%%%%%%%%%%%%%%%%%%%%%%%%%%%%%% 
% Prints your review date and 'Draft Version' (From Josullvn, CS, CMU)
\newcommand{\reviewtimetoday}[2]{\special{!userdict begin
/bop-hook{gsave 20 710 translate 45 rotate 0.8 setgray
/Times-Roman findfont 12 scalefont setfont 0 0   moveto (#1) show
0 -12 moveto (#2) show grestore}def end}}
% You can turn on or off this option.
% \reviewtimetoday{\today}{Draft Version}
%%%%%%%%%%%%%%%%%%%%%%%%%%%%%%%%%%%%%%%%%%%%%%%%%%%%%%%%%%%%%%%%%%%%%%%%%%%%%%% 



% #endregion

%%%%%%%%%%%%%%%%%%%%%%%%%%%%%%%%%%
% #region TOC/MINITOC Formatting %
%%%%%%%%%%%%%%%%%%%%%%%%%%%%%%%%%%

\etocbookstyle
% \etocstandardlines
\newlength\tocrulewidth
\setlength{\tocrulewidth}{1.5pt}
\setcounter{secnumdepth}{3}
\setcounter{tocdepth}{3}
\etocsetnexttocdepth{2}

% #endregion

% Commands to format MAX-SMT constants representing top or bottom args
\newcommand{\argtop}[1]{{\ensuremath{\bar{#1}}}}
\newcommand{\argbot}[1]{{\ensuremath{\ubar{#1}}}}

\newcommand{\ptype}[1]{\textttpit{#1}} % for planing params types
\newcommand{\pvar}[1]{\ensuremath{\operatorname{?\mathit{#1}}}} % for planning params vars (add subscript outside)
\newcommand{\pconst}[1]{\ensuremath{\operatorname{\mathit{#1}}}} % for planning params constants (add subscript outside)
\newcommand{\poper}[1]{{\textttp{#1}}} % for planning params operators (action, tasks)
\newcommand{\psf}[1]{{\textttp{#1}}} % for planning state variables

\newcommand{\htnmaker}{\textsc{HTN-Maker}}
\newcommand{\htnlearn}{\textsc{HTNLearn}}
\newcommand{\wordhtn}{\textsc{Word2HTN}}
\newcommand{\wordvec}{\textsc{Word2Vec}}
\newcommand{\bnfsym}[1]{\ensuremath{\langle\texttt{#1}\rangle}}

%%%%%%%%%%%%%%%%%%%%%%%%
%	  TODO commands    %
%%%%%%%%%%%%%%%%%%%%%%%%

\ifdefined\notodo
  \newcommand{\todomarker}[2][]{}
  \newcommand{\arthur}[1]{}
  \newcommand{\philippe}[1]{}
  \newcommand{\pcomment}[2][]{}
  \newcommand{\pinline}[1]{}
  \newcommand{\acomment}[1]{}
\else
  \setlength{\marginparwidth}{0.85in}
  \newcommand{\todomarker}[2][]{\todo[inline,size=\normalsize,#1]{TODO #2}}
  \newcommand{\arthur}[1]{{\color{blue} \textbf{#1}}}
  \newcommand{\philippe}[1]{{\color{ForestGreen} \textbf{#1}}}
  \newcommand{\pcomment}[2][]{\todo[backgroundcolor=ForestGreen!25,linecolor=ForestGreen,caption={P},prepend,size=\tiny,#1]{#2}}
  \newcommand{\pinline}[1]{\todo[backgroundcolor=ForestGreen!25,author={Philippe},inline,size=\normalsize]{#1}}
  \newcommand{\acomment}[1]{\todo[linecolor=blue,backgroundcolor=blue!25,bordercolor=blue,caption={A},prepend,size=\tiny]{#1}}
\fi


%%%%%%%%%%%%%%%%%%%%%%%%%%
% #region CUSTOM SYMBOLS %
%%%%%%%%%%%%%%%%%%%%%%%%%%

\DeclareMathSymbol{\mlq}{\mathord}{operators}{``}
\DeclareMathSymbol{\mrq}{\mathord}{operators}{`'}

\renewcommand{\epsilon}{\varepsilon}

\newcommand{\red}[1]{{\color{red}#1}} 
\newcommand{\blue}[1]{{\color{blue}#1}}
\newcommand{\magen}[1]{{\color{magenta}#1}}
\newcommand{\bxi}{\boldsymbol{\xi}}
\newcommand{\x}{\boldsymbol{x}}
\newcommand{\X}{\boldsymbol{X}}
\newcommand{\q}{\boldsymbol{q}}
\newcommand{\e}{\boldsymbol{e}}
\newcommand{\f}{\boldsymbol{f}}
\newcommand{\g}{\boldsymbol{g}}
\renewcommand{\k}{\boldsymbol{k}}
\newcommand{\h}{\boldsymbol{h}}
\renewcommand{\r}{\boldsymbol{r}}
\renewcommand{\u}{\boldsymbol{u}}
\renewcommand{\a}{\boldsymbol{a}}
\newcommand{\p}{\boldsymbol{p}}
\newcommand{\bPhi}{\boldsymbol{\Phi}}
\newcommand{\bPi}{\boldsymbol{\Pi}}
\newcommand{\bPixi}{\boldsymbol{\Pi_{\xi}}}
\newcommand{\bTheta}{\boldsymbol{\Theta}}
\newcommand{\W}{\boldsymbol{W}}
\newcommand{\Rq}{\boldsymbol{r_q}}
\newcommand{\Ru}{\boldsymbol{r_u}}
\newcommand{\U}{\boldsymbol{U}}
\newcommand{\bW}{\boldsymbol{W}}
\newcommand{\boldeta}{\boldsymbol{\eta}}
\newcommand{\boldmu}{\boldsymbol{\mu}}

% #endregion

%%%%%%%%%%%%%%%%%%%%%%%%%%%%%%%%%
% #region CUSTOM MATH OPERATORS %
%%%%%%%%%%%%%%%%%%%%%%%%%%%%%%%%%

\DeclareMathOperator*{\argmax}{arg\,max}%
\DeclareMathOperator*{\argmin}{arg\,min}%
\DeclareMathOperator{\cost}{cost}%
\DeclareMathOperator{\head}{head}
\DeclareMathOperator{\name}{name}
\DeclareMathOperator{\pre}{pre}
\DeclareMathOperator{\post}{post}
\DeclareMathOperator{\eff}{eff}

\DeclareMathOperator{\decs}{Decs}
\DeclareMathOperator{\vdecs}{{VDecs}}

\DeclareMathOperator*{\args}{args}%
\DeclareMathOperator{\argt}{arg}%
\DeclareMathOperator{\subs}{subtasks}
\DeclareMathOperator*{\tsym}{sym}%
\DeclareMathOperator{\prims}{Prims}%
\DeclareMathOperator{\win}{win}%
\DeclareMathOperator{\gnd}{gnd}%
\DeclareMathOperator{\inst}{inst}%
\DeclareMathOperator*{\soft}{\textit{\textsc{Soft}}}%
\DeclareMathOperator*{\hard}{\textit{\textsc{Hard}}}%

\DeclareMathOperator{\zip}{zip}

% Functions used in MAX-SMT formulations
\DeclareMathOperator*{\pgroup}{\textsc{PGroup}}%
\DeclareMathOperator*{\isgroupcount}{\textsc{IsGroupCount}}%
\DeclareMathOperator*{\grounding}{\textsc{Gnd}}%

\DeclareMathOperator*{\gndparam}{\textsc{GndParam}}%
\DeclareMathOperator*{\gndgroup}{\textsc{GndGroup}}%

% Two dots instead of three
\newcommand{\slice}{\mathinner{{\ldotp}{\ldotp}}}
% Two dots instead of three
\newcommand{\sliceeq}{\mathinner{{\ldotp}{\ldotp}{=}}}

% Big eq symbol, like bigcup
\DeclareMathOperator*{\bigeq}{\scalerel*[15pt]{=}{\sum}}


% #endregion



\ifdefined\watermark
\AddToShipoutPictureFG{%
\AtPageCenter{
% \rotatebox[origin=c]{90}{% Angle
  \scalebox{10}{% Size (scaled 10x)
      \makebox[0pt]{% Centered horizontally
            \color{Gray!50}% Colour
            \transparent{0.4}%
            \vphantom{j}DRAFT%
      }%
    }%
  % }%}
}
}

\else
\fi



\newcommand*\myglsentry[1]{%
  \protect\ifglsused{#1}{%
    \glsentryshort{#1}%
  }{%
    \glsentrylong{#1} (\glsentryshort{#1})%
  }%
}

\makeglossaries

\newacronym{rrt}{RRT}{Rapidly-exploring Random Trees}
\newacronym{rrtstar}{RRT*}{Assymptotically Optimal Rapidly-exploring Random Trees}
\newacronym{odes}{ODEs}{ordinary differential equations}
\newacronym{com}{CoM}{center of mass}
\newacronym{dfl}{DFL}{dynamic feedback linearization}
\newacronym{AABBs}{AABBs}{Axis-Aligned Bounding Boxes}
\newacronym{samp}{SAMP}{Sensitivity-Aware Motion Planner}
\newacronym{sarrt*}{SARRT*}{Sensitivity-Aware RRT*}
\newacronym{sst*}{SST*}{Stable Sparse RRT*}
\newacronym{sasst*}{SASST*}{Sensitivity-Aware SST*}
\newacronym{ompl}{OMPL}{Open Motion Planning Library}
\newacronym{lazysamp}{LazySAMP}{Lazy Sensitivity-Aware Motion Planner}
\newacronym{lazysarrt*}{LazySARRT*}{Lazy Sensitivity-Aware RRT*}
\newacronym{SAshortcut}{SAShortcut}{Sensitivity-Aware Shortcut}
\bibliography{references}

%% This is file `example.tex',
%% Copyright 2013 Tristan GREGOIRE
%% Copyright 2015 Yann BACHY
%
% This work may be distributed and/or modified under the
% conditions of the LaTeX Project Public License, either version 1.3
% of this license or (at your option) any later version.
% The latest version of this license is in
%   http://www.latex-project.org/lppl.txt
% and version 1.3 or later is part of all distributions of LaTeX
% version 2005/12/01 or later.
%
%
% This work has the LPPL maintenance status `maintained'.
% 
% The Current Maintainer of this work is T. GREGOIRE
%

% Loading the tlsflyleaf.sty package require some option to define the
% establishment name, the doctoral school and the PhD speciality.
% In that aim you have 2 key-value option:
%   - Ets=<value> : define the establishment name
%   - ED=<value>  : define the doctoral school and speciality
%   - ED2=<value> : define the second speciality ("double mention"). OPTIONAL.
% The full list of accepted values for each option could be find either
% in the documentation or in ED-list.txt and Ets-list.txt files provide with the package.
\usepackage[ED=EDSYS-InfoRob, Ets=INSA]{tlsflyleaf}

% ==================
% Setup basic string
% - PhD Title
% - author
% - defence date
% - laboratory
% - cotutelle
\title{Robust Control-Aware Motion Planning Against Parametric Uncertainties}
\author{Simon WASIELA}
\defencedate{XXXXX}
\lab{LAAS-CNRS}
%\cotutelle{}

% ==================
% Setup people like your boss, the jury team and the referees
% - First you need to define how number they will be in each category
%   It is done with the commands \nboss{n}, \nreferee{n} and \njudge{n}.
%   You can define more people in each category than the number given 
%   but only the first "\npeople" will be print.
% - Then use the command \makesomeone{<category>}{<number>}{<name>}{<status>}{<other>}
%   where:
%     <category> should be select in ['boss', 'referee', 'judge']
%     <number>   is the rank for printing the person. 
%                Only number <= "\npeople" will be printed
%     <name>     First name and las name of the people
%     <status>   Is (s)he a "charg\'e de recher" ou un "professeur d'universit\'e"...
%     <other>    What ever string you want to add (laboratory, jury member place...).
%% Boss
\nboss{2}
\makesomeone{boss}{1}{Thierry SIMÉON}{}{} % Sera afiche en premier
\makesomeone{boss}{2}{Juan CORTÉS}{}{}  % Sera affiche en second
%% Referee
\nreferee{2}
\makesomeone{referee}{1}{Marilena VENDITTELLI}{}{}
\makesomeone{referee}{2}{Pedro CASTILLO}{}{}
%% Jury
\njudge{1}
\makesomeone{judge}{1}{XXXX}{XXXX}{XXX}

% ============================================================
% DOCUMENT
%\begin{document}
%   \makeflyleaf
%\end{document}

\sloppy

\begin{document}

% \includepdf{couverture_these.pdf}

\begin{otherlanguage}{french}
  \makeflyleaf
\end{otherlanguage}

\cleardoublepage


\frontmatter
\pagestyle{plain}

\chapter*{Abstract}

This thesis addresses the problem of generating robust and accurate trajectories taking into account uncertainties in the robot dynamic model. 
Based on the notion of \emph{closed-loop sensitivity}, which quantifies deviations in the closed-loop trajectories of any robot/controller pair against uncertainties in the robot model parameters, so-called ``uncertainty tubes'' can be derived for bounded parameter variations.
Such tubes bound the system evolution both in the state and control input spaces.
Based on the ``control-aware motion planning'' paradigm, this work leverages these ``uncertainty tubes'' to enforce robust constraints within sampling-based planners.

The first contribution of this thesis focuses on generating globally sensitivity-optimal trajectories while enforcing robust constraints thanks to uncertainty tubes.
However, results show that computing these uncertainty tubes at each iteration of a sampling-based planner is a bottleneck for the method.
Therefore, a lazy robust feasibility check is proposed to limit the frequency of computing uncertainty tubes, thus improving the computational efficiency of the framework.

Another contribution is to explore deep learning neural network that can be used to speed up closed-loop sensitivity dynamic and subsequent tubes computation.
By leveraging the structural similarity between ordinary differential equations and recurrent neural networks, a GRU-based architecture is proposed to directly link a planned trajectory with uncertainty tubes, achieving an order-of-magnitude improvement in computation time.

The thesis also shows how to integrate GRU-based predictions into a sampling-based planner, resulting in a more computationally efficient planning framework. 
Furthermore, while robust state-of-the-art methods primarily focus on satisfying robust constraints, this work leverages uncertainty tubes to define a cost function that enables the planning of task-specific, accurate trajectories.

Finally, the proposed sensitivity-based planning framework is experimentally validated on a 3D quadrotor in two challenging scenarios: a navigation through a narrow window, and an in-flight ``ring catching'' task that requires high accuracy. 
\chapter*{Résumé}

Cette thèse aborde le problème de la génération de trajectoires robustes et précises en tenant compte des incertitudes dans le modèle dynamique du robot.
Basée sur la notion de \emph{sensibilité en boucle fermée}, qui quantifie les déviations des trajectoires en boucle fermée de toute paire robot/contrôleur face aux incertitudes des paramètres du modèle du robot, des ``tubes d'incertitude'' peuvent être calculés pour des variations bornées de paramètres. 
Ces tubes bornent l'évolution du système à la fois dans les espaces d'état et d'entrée de commande. 
En s'appuyant sur le paradigme de la ``planification de mouvement sensible au contrôle'', ce travail exploite ces ``tubes d'incertitude'' pour imposer des contraintes robustes au sein des planificateurs basés sur l'échantillonnage.

La première contribution de cette thèse porte sur la génération de trajectoires globalement optimales en sensibilité tout en imposant des contraintes robustes grâce aux tubes d'incertitude.
Cependant, les résultats montrent que le calcul de ces tubes d'incertitude à chaque itération d'un planificateur basé sur l'échantillonnage constitue un frein pour la méthode.
Ainsi, une vérification de faisabilité robuste paresseuse est proposée afin de limiter la fréquence de calcul des tubes d'incertitude, améliorant ainsi l'efficacité computationnelle des algorithmes.

Une autre contribution consiste à explorer l'utilisation de réseaux de neurones profonds pour accélérer l'intégration de la dynamique de la sensibilité en boucle fermée et le calcul subséquent des tubes.
En tirant parti de la similarité structurelle entre les équations différentielles ordinaires et les réseaux neuronaux récurrents, une architecture basée sur les GRU est proposée pour corréler directement une trajectoire planifiée aux tubes d'incertitude, atteignant une amélioration d'un ordre de grandeur en termes de temps de calcul.

La thèse montre également comment intégrer les prédictions basées sur les GRU dans un planificateur basé sur l'échantillonnage, aboutissant à des approches de planification plus efficace en temps de calcul.
De plus, alors que les méthodes robustes de l'état de l'art se concentrent principalement sur la satisfaction des contraintes robustes, ce travail exploite les tubes d'incertitude pour définir une fonction de coût permettant la planification de trajectoires précises et spécifiques à la tâche.

Enfin, la méthode de planification basé sur la sensibilité proposé est validé expérimentalement sur un quadrirotor 3D dans deux scénarios exigeants : une navigation à travers une fenêtre étroite, et une tâche de ``capture d'anneau'' en vol nécessitant une grande précision. 

\chapter*{Remerciements}

Une thèse est une quête de savoir, un voyage que l'on entreprend par curiosité et passion, sans toujours savoir exactement où il nous mènera.
C'est une aventure parsemée d'incertitudes, où il est facile de se perdre sans un entourage fiable sur lequel on peut compter, qui éclaire notre chemin et nous donne la force d'avancer à chaque étape.

C'est pourquoi je tiens avant tout à remercier Nic et Juan, mes directeurs de thèse, pour leur précieux encadrement, leurs conseils avisés et leur soutien tout au long de cette aventure.
Merci de m'avoir fait confiance, d'avoir compris mes attentes, de m'avoir consacré votre temps et de tout ce que vous avez pu faire d'autres pour moi.
Je suis heureux que nous ayons été sur la même longueur d’onde. 
Travailler avec vous fut un véritable plaisir et m’a permis de mener ce voyage à bon port avec enthousiasme.

Un merci tout particulier à Marco Cognetti, qui a rejoint ce projet en cours de route.
Merci pour ton engagement aussi bien dans le suivi de ma thèse que dans la rédaction.
Ton expertise sur le sujet m’a été d’une aide précieuse et, bien que tu n’aies pas été officiellement rattaché à cette thèse, j’espère que ces quelques mots mettront en valeur ta contribution essentielle à sa réussite.

Je tiens ensuite à remercier ma compagne Laure.
Au-delà de cette thèse, ma plus belle aventure est sans aucun doute celle que nous vivons tous les deux.
Merci pour tes attentions, ta petience, ton soutien, ou encore ton rire.
Merci de m'avoir remonté le moral quand il était au plus bas.
Merci d'avoir supporté mes interminables discussions sur la génération de mon dataset.
Sans toi, cette thèse aurait été bien différente et ma vie, en général, serait beaucoup plus terne.

Merci à ma mère Françoise et mon père Eric.
Merci d'avoir soutenu ma décision de me lancer dans cette thèse.
Merci d'avoir stressé à ma place tout au long de ces années, surtout toi, papa.
Je vous dois ce que je suis aujourd'hui.
J'espère vous avoir rendus fiers.

Merci à mes frères Clément et Paulin.
Même si la distance ne nous permet plus de nous voir aussi souvent, vous occuperez toujours une place de choix dans mon coeur.
Je suis ravi de vous voir vous épanouir ; cela me motive à me surpasser et m’a permis de terminer cette thèse et ce manuscrit sereinement.
Bien entendu, j'espère que vous mettrez cedit manuscrit sous vitrine, cela va de soi.
Je tiens également à remercier ma belle-soeur Mélanie pour l’intérêt que tu as porté à mes travaux et, au moins toi, de ne pas tricher à Elfenland.

Merci à mes cousins, cousines, oncles, tantes, parrain, marraine et grands-parents pour leur soutien et enthousiasme.

Je tiens ensuite à remercier mes amis: Corentin, Katerina, Erwan, Maelenn, Cyriaque, Seb, Franklin, Jeanne et JC, pour ces bons moments à incarner les meilleurs personnages d'Yskanov, ou encore pour ces échanges frauduleux d'une pierre contre trois argiles.
Les moments passés ensemble me sont précieux et m'ont offert des bouffées d'oxygène quand mon esprit ne pensait plus qu'à cette thèse.

Merci également à mes amis de la fameuse chambre 106.5, Bastien et Eugène.
Merci de vous être intéressés à mes travaux et d'avoir compris mes absences.
J'espère bientôt pouvoir rattraper le temps perdu et que nous nous remémorerons encore les bêtises de l'internat.

Cette thèse aurait eu une tout autre saveur sans les membres de la "Place des Swifties", avec qui j’ai passé d’excellents moments.
Merci à toi, Lou, pour ces franches rigolades, ces moments de malaise, ces concours de lancer de glands, et bien sûr, ce projet de tyrolienne entre Toulouse et les Alpes (en espérant que personne ne te vole l’idée en lisant ceci) !
Merci à toi, Bastien, d’être toujours partant pour engloutir plusieurs cafés avec moi à n’importe quelle heure, pour ces coupes de cheveux rocambolesques, pour ton aide précieuse dans le setup de mes expériences, pour ne pas avoir hésité un seul instant – mais alors pas du tout – à rejoindre ce carré, et bien sûr, pour ton sens de l’humour inégalable.
Je suis sûr que cette dernière qualité m’a permis de récupérer pas mal d’heures de vie perdues, consumées par la thèse.
Merci à toi, ô terrifiant Guillaume ! Nos joutes verbales (pour rester politiquement correct) me manqueront.
Merci pour ton expertise technique en programmation et pour avoir toujours répondu présent pour les impressions 3D.
Et bien sûr, merci de m’avoir conseillé d’excellents jeux, tels qu'Andor !
Finalement, merci à toi, Smail, mon acolyte de voyage depuis la première heure de cette thèse.
J'ai énormément apprécié les moments passés avec toi, que ce soit au travail, lors des conférences, ou même au McDo à 1h du matin après les deadlines.
Avoir travaillé aux côtés de quelqu'un d’aussi compétent que toi m’a poussé à donner le meilleur de moi-même pour rester à la hauteur.
Tu seras toujours le bienvenu pour "crash" chez moi, à condition que tu laisses des Chocobons, bien sûr.

Merci à toi, Philippe, pour tous ces bons moments passés ensemble.
Merci pour ta gentillesse et ton empathie.
Ton expertise photo en détourage de drone m'a sauvé bien des heures de sommeil.
Pas sûr de vouloir encore suivre tes "raccourcis" en randonnée, mais c’est avec plaisir que je viendrais prendre une bière au BSC ou regarder un match de rugby.

Merci infiniment Anthony pour toutes ce rigolades.
Merci pour ces discussions allant de l'algorithmique au jeu de rôle, en passant par l'élaboration de notre fameux monde gymnastique, où les personnes s'empileraient les unes sur les autres et où différents bâtiments et activités apparaîtraient en fonction des étages.
J'espère te recroiser, non pas dans ce monde gymnastique catastrophique, mais bien en enfer, Hellbanger.

Je tiens également à te remercier, William, pour ces discussions, ces après-midis jeux de société, et bien sûr pour ta maintenance sur Blender.
Merci à toi Stephy pour ces moments kool passé ensemble et ta gentillesse.
Merci, Illinka, pour ta prise de parole au sujet du sapin, pour l'instauration de ce jeudi Ghibli que je n'ai pas su tenir, et d'être toujours partante pour un concours de Glams.
Merci à vous Fadma et Maël pour les bons moments passés ensemble.

Merci également aux personnes que j'ai vu partir au court de cette aventure mais qui m'ont permis une intégration plus que chaleureuse, merci Amandine, Antoine, Yannick, Jeremy, Gianluca, Dario.
Merci à Arthur, Félix, Anthony, Aurélie, Rachid, Adrien, Roland, Virgile, Jonas, Alessia, Rebecca, Phani, James, et bien d’autres encore, constituant cette équipe RIS.
Merci de m’avoir si bien accueilli, et un grand merci à Simon Lacroix d'être à la tête de cette équipe formidable.

Merci aux membres de l’équipe Rainbow, et tout particulièrement à Paolo pour m’avoir donné cette opportunité et pour ton enthousiasme à propos de mes travaux. 
Merci également à Pascal et Ali pour leurs échanges enrichissants.

Je souhaite également exprimer ma gratitude envers les membres de mon jury.
Un grand merci à Daniel pour avoir présidé ce jury, et à Marilena, Pedro et Marco pour l’accueil réservé à mes travaux.

Merci au Sapin Éternel, ainsi qu’à ceux qui l’entretiennent, égayant le hall de notre bâtiment et y apportant joie et bonne humeur.

Enfin merci au LAAS-CNRS de m'avoir acceuilli et à l'ANR d'avoir financer mes travaux de recherche.

\cleardoublepage

\pagestyle{fancy}
\tableofcontents
\cleardoublepage

% Need to be set here to have the right formatting for the main toc and the local tocs
\etocsetlevel{chapter}{-1}
\etocsetlevel{section}{0}
\etocsetlevel{subsection}{1}
\etocsetlevel{subsubsection}{2}

\etocsettocstyle{{\large \textbf{Contents}}\vspace{-0.5\baselineskip}\\\rule{\linewidth}{\tocrulewidth}\vskip0.5\baselineskip}{{\parindent=0em\rule{\linewidth}{\tocrulewidth}}}

% Here you can see an example of how to create text conditioned by the language
% variable. The \iftoggle command:
%
%   \iftoggle{ThesisInEnglish}{%
%   <your-text-in-english>
%   }{%
%   <your-text-in-french>
%   }
%
% will compile only one of the two blocks, depending on the variable you set at
% the beginning of this document. Language selection is managed this way in the
% formatAndDefs.tex file. You too can create sections of your thesis that is
% language dependend this way, although you probably won't need it.

% \printacronyms
% % \printglossaries
% \cleardoublepage

% \cleardoublepage
% \addcontentsline{toc}{chapter}{\listfigurename}
% \listoffigures
% \cleardoublepage
% \addcontentsline{toc}{chapter}{\listtablename}
% \listoftables
% \cleardoublepage
% \addcontentsline{toc}{chapter}{\listalgorithmname}
% \listofalgorithms

% \newpage 
% \ % The empty page
% \newpage

\addcontentsline{toc}{chapter}{Acronyms}
\printnoidxglossary[type=\acronymtype]
\cleardoublepage

% \newpage 
% \ % The empty page
% \newpage
% \cleardoublepage

\mainmatter
\glsresetall
\chapter{Introduction}
\markboth{Introduction}{}% To set left/right header
% \localtableofcontents

\section{Robust Motion Planning}

\section{Context and Objectives}

\section{Thesis Outline}

\section{Thesis Contributions}

\todomarker{}
% \cleardoublepage
% \newpage 
% \ % The empty page
% \newpage
\chapter{Related work}\label{chap:related_work}
\markboth{Related work}{}% To set left/right header
% \localtableofcontents

This chapter provides an overview of the related work on the key concepts that this thesis builds upon and compares with.

\section{Motion planning}

\subsection{Path planning}

This subsection provides an overview of various path planning approaches. 
For a more detailed survey, please refer to~\cite{cLavalle, cKavraki, cFrazzoli}.
The path planning problem focuses on finding a collision-free and/or optimal path between two configurations: a starting configuration and a goal configuration. 
The work of~\cite{cConfigSpace} introduced the configuration space concept as a general framework for planning motions in arbitrary kinematic systems.
This idea provided a solid foundation for the field and, as noted in~\cite{cSpatialPlan}, formally established the motion planning problem as the task of finding a path within the configuration space.

However, planning motion in the configuration space poses significant challenges due to the problem computational complexity. 
A key stone in addressing this issue is the introduction of artificial potential fields~\cite{cAPF}, which create a repulsive force field around obstacles and an attractive force field toward the target.
However, although this strategy has proven to be efficient, it sacrifices guarantees such as algorithm completeness and optimality.

Another common approach to the path planning problem involves search-based techniques. 
These methods operate on a discrete representation of the configuration space, where vertices correspond to a finite set of robot configurations, and edges represent possible transitions between these configurations.
The desired path is found by performing a search for a minimum-cost path in such a graph using Dijkstra algorithm~\cite{cDijk}. 

Advances in computational techniques have enabled the development of a new class of algorithms, commonly referred to as sampling-based planners that allow to generate global motion plans.
A key contribution of the domain is the work of~\cite{cLatombe}, which improves the aforementioned potential fields approach by integrating a random walk mechanism, thus guaranteeing the probabilistic completeness of the algorithm.
Sampling-based planners are divided in two main categories: graph-based planners and tree-based planners.

Graph-based sampling-based planners, such as \gls{prm}~\cite{cPRM}, generate a roadmap by sampling configurations in the free space and connecting them with feasible paths. 
The technique involves a graph building phase and a query phase.
The graph is first built by randomly sampling configurations (denoted nodes) in the robot configuration space.
Each sampled configuration is checked for collisions with obstacles.
For each collision-free node, the planner attempts to connect it to nearby free configurations by creating edges between them.  
The edges are valid if the path connecting the two configurations is collision-free.
This procedure typically involves a notion of neighborhood that is space-dependent and generally increases as the dimensionality of the space grows.
Once the graph is built, a graph search query is performed to compute the desired path, using graph search techniques such as Dijkstra algorithm~\cite{cDijk} or A*~\cite{cA*}.

The other planner family among sampling-based planners are tree-based planners such as the well-known \gls{rrt}~\cite{cRRT}.
This algorithm generates a tree starting from an initial configuration.
At each iteration, a new configuration is sampled $q^{rand}$.
It's nearest neighbor among the existing tree nodes is found $q^{near}$, and then an extension is attempt starting from $q^{near}$ toward $q^{rand}$ leading to a new configuration $q^{new}$.
If the extension is collision free, the configuration is added to the tree as anew node with an edge connecting the two nodes (see Figure~\ref{fig:rrt}).

Therefore, sampling-based motion planners have become one of the most widely used strategy for generating global motion paths in the presence of obstacles, due to their efficiency in handling complex and high-dimensional configuration spaces, as well as their completeness guarantee.
Advances in sampling-based planners have then focused on generating feasible and cost-effective paths through various strategies, such as the transition-based approach~\cite{cTRRT}. 
These strategies often achieve asymptotic optimality by employing optimal connection procedures during the tree or graph construction process~\cite{cRRTstar, cTRRTstar, cFMT}.
Research on sampling-based planners has expanded into various areas, such as developing suitable sampling strategies~\cite{cSampling}, exploring lazy collision checking approaches~\cite{cLazy1}, extending to the molecular domain~\cite{cMolecular}, and more.

The aforementioned sampling-based methods are generally referred to as global planning methods, as they rely on local planning techniques to generate the edges that connect the sampled configurations.
Such local planning methods are not considered global, as they do not address the entire motion planning problem. 
A local planner is typically fast; however, the generated path may not fully satisfy the problem constraints (e.g., it could result in collisions).
Traditional local strategies typically involve path segments, but local planners often need to account for additional constraints.
For example, significant work has been done to enforce kinematic constraints on these local plans, ensuring the kinematic feasibility of the resulting global paths.
Such work includes, for example, Dubins path generation~\cite{cDubins}, which enables connecting two configurations under forward motion constraints and curvature constraints for car-like robots. 
The Reeds-Shepp method~\cite{cReeds} extends this local path generation procedure to car-like robots capable of moving both forward and backward.

\begin{figure} [htp]
    \centering
    \includegraphics[width=0.6\linewidth]{figures/models/rrt.png} 
    \caption{Illustration of the \myglsentry{rrt} tree extension procedure.}%
    \label{fig:rrt}%
  \end{figure}

\subsection{Kinodynamic planning}

While the aforementioned path finding methods focus on generating a route between two robot configurations, they do not address the problem on how the robot should move along this route over time.
The kinodynamic motion planning problem extends traditional motion planning by considering both the kinematic constraints (e.g., turning radius) and dynamic constraints (e.g., velocity, acceleration) of the robot, ensuring that the generated motion plans are physically executable.

% 2. Trajectory Generation

% While path finding determines the spatial route, trajectory generation focuses on how the robot should move along this route over time, considering kinematic and dynamic constraints to ensure feasibility.

%     Polynomial Splines:
%     Polynomial-based methods, such as cubic splines and quintic splines, are widely used for smooth trajectory interpolation between waypoints.
%         These methods ensure continuity in position, velocity, and acceleration, which is essential for smooth robotic motion.
%         They are computationally efficient and suitable for applications like robotic manipulators or UAV navigation.

%     Optimization-based Methods:
%     Optimization frameworks like CHOMP and STOMP formulate trajectory planning as an optimization problem, minimizing a cost function that accounts for obstacle avoidance and trajectory smoothness.
%         CHOMP uses gradient-based optimization to refine an initial trajectory, making it collision-free while maintaining smoothness.
%         STOMP, on the other hand, employs stochastic sampling to explore feasible trajectories, offering robustness in environments with complex obstacle configurations.
%         More advanced methods integrate convex optimization or leverage real-time solvers for handling dynamic constraints, enabling robots to adapt to changes in their environments.

%     Learning-based Approaches:
%     Data-driven methods leverage neural networks trained on large datasets of prior trajectories to predict feasible paths.
%         Imitation learning is a common technique where the robot learns from expert demonstrations to generate trajectories in similar scenarios.
%         These methods excel in reducing computation time, making them highly suitable for real-time applications, though they often rely on high-quality training data and can struggle with generalization.

% 3. Kinodynamic Motion Planning

% Kinodynamic motion planning extends traditional motion planning by considering both the kinematic constraints (e.g., turning radius) and dynamic constraints (e.g., velocity, acceleration) of the robot, ensuring that the generated motion plans are physically executable.

%     Time-parameterized RRTs:
%     Extensions of RRT, such as RRT-Connect and Kinodynamic RRT, incorporate dynamic constraints directly into the planning process.
%         These planners ensure that the generated paths are not only collision-free but also adhere to the robot's motion model, such as maximum acceleration or turning capabilities.
%         The addition of time-parameterization allows for the generation of time-efficient trajectories, suitable for systems requiring fast, dynamic responses.

%     Model Predictive Control (MPC):
%     MPC-based planners solve an optimization problem over a receding time horizon, generating dynamically feasible motion plans that account for constraints and obstacles.
%         MPC is particularly effective for systems with complex dynamics, such as mobile robots, autonomous vehicles, and drones.
%         Its ability to continuously update plans in real time makes it ideal for dynamic and uncertain environments.

%     Search-based Methods:
%     Algorithms like state-time A* or lattice-based planning search directly in the state-time space, incorporating kinodynamic constraints at the planning level.
%         These methods often use discretization to simplify the problem but can become computationally expensive in high-dimensional spaces.
%         They are well-suited for problems requiring precise adherence to constraints, such as robotic manipulators or autonomous cars navigating through dense traffic.

%     Nonlinear Dynamics Integration:
%     For robots with non-holonomic constraints (e.g., wheeled robots), planning methods often employ predefined motion primitives, such as Dubins paths or Reeds-Shepp curves, to generate feasible trajectories.
%         These techniques ensure that the paths are physically realistic while maintaining computational efficiency.

\section{Motion planning under uncertainty}

\subsection{Chance-constrained}

\subsection{POMDP}

\subsection{Chance-constrained}

\section{Control-aware motion planning}

\subsection{MPC}

\subsection{FaSTrack}

\subsection{LQR-trees}

\subsection{Contraction theory}

\subsection{Randomized uncertainty propagation}

\subsection{Sensitivity}
% \cleardoublepage
% \newpage 
% \ % The empty page
% \newpage
\chapter{Preliminaries}\label{chap:models}
\markboth{Preliminaries}{}% To set left/right header
% \localtableofcontents

This chapter introduces the notion of \emph{closed-loop sensitivity}, which forms a key foundation of this thesis.
This concept quantifies how variations of some model parameters (supposed to be uncertain) affect the evolution of the system in closed-loop, i.e., by also taking into account any controller chosen for executing the task.
Then, it is presented how this sensitivity notion can be leveraged to derive \emph{uncertainty tubes} that bounds the system evolution both in the \emph{input} and \emph{state} spaces.
These tubes are subsequently used to enforce robust constraints within the several motion planning algorithms presented in this thesis.
Finally, the quadrotor and differential drive robot models used in this manuscript are introduced.

\section{Closed-loop sensitivity}\label{sec:sensi_and_tubes}

\subsection{Definition}\label{sec:sensi}

Consider an arbitrary dynamic system with a set of uncertain parameters $\p \in \mathbb{R}^{n_p}$ (i.e. parameters that are difficult to model).
The system dynamic can be described using the following set of \gls{odes}:
\begin{equation}\label{eq:dyna}
    \dot{\q}=\f(\q,\,\u,\,\p), \quad \q(t_0)=\q_{0},
\end{equation}
where $\q\in \mathbb{R}^{n_{q}}$ is the system state vector and $\u\in \mathbb{R}^{n_{u}}$ is the control input vector.
Also, assume the presence of a controller $\boldmu$ of any form whose aim is to track a \emph{desired trajectory} $\q_d(t)$ such that:
\begin{equation}\label{eq:ctrl}
     \left \{
     \begin{array}{l l}
          \dot{\bxi} = \g(\bxi,\,\q,\,\q_d,\,\p_n,\,\k_c,\,t), \quad \bxi(t_0)=\bxi_{0}, \\
          \u=\boldmu(\bxi,\,\q,\,\q_d,\,\p_n,\,\k_c,\,t), 
   \end{array} 
   \right .
\end{equation}
where $\bxi\in \mathbb{R}^{n_{\xi}}$ are the internal states of the controller (e.g., an integral action), $\k_{c}\in \mathbb{R}^{n_{k}}$ the controller gains, and $\p_{n}\in{\mathbb{R}^{n_{p}}}$ is the vector of "nominal" system parameters used in the control loop, i.e. the estimated nominal values of $\p$.

In line with the previous definitions, the following sections of this thesis will then differentiate between three types of state vectors of key importance:
\begin{enumerate}
  \item $\q_d$: The \emph{desired} system state vector which refers to the desired values of the controllable system states. 
  This state vector is typically the output of a motion planner. Note that the dimension of this vector may differ from that of the real system because the vector represents a simplified or abstracted model, which might omit certain physical aspects or constraints that are present in the actual system, particularly in the case of under-actuated systems, where not all degrees of freedom are controlled.
  \item $\q_n$: The \emph{nominal} system state vector which represents the real state values of the system during the execution under nominal parameters (i.e. when the uncertain system parameters $\p$ perfectly match the parameter values used in the control loop $\p_n$). 
  A distinction is made between the nominal states and the desired states, as they are generally not equal due to factors such as controller settings (e.g. overshooting behavior).
  \item $\q$: The \emph{uncertain} system state vector which refers to the behavior of the system when the values of real parameters are taken into account.
\end{enumerate}
Note that the same notations apply to the control input vector as well (e.g., $\u_n$ represents the "nominal" control input values, i.e., when $\p=\p_n$).

It is possible to quantify how the presence of uncertain parameters (i.e. when the real system parameters $\p$ deviate from the nominal value $\p_n$ used in the control loop) affects the evolution of $\q(t)$ and $\u(t)$ according to the following matrices:
\begin{equation}\label{eq:sensi}
  \bPi(t)=\left.\frac{\partial \q(t)}{\partial \p}\right|_{\p=\p_n} \quad\quad \bTheta(t)=\left.\frac{\partial \u(t)}{\partial \p}\right|_{\p=\p_n}
\end{equation}
where $\bPi(t)\in \mathbb{R}^{n_q \times n_p}$ and $\bTheta(t)\in \mathbb{R}^{n_u \times n_p}$ are respectively defined in~\cite{cPi,cTh} as the \emph{state-sensitivity matrix} and the \emph{input-sensitivity matrix}.
A closed-form expression for Equation~(\ref{eq:sensi}) is, in general, not available. 
However, as shown in~\cite{cPi,cTh}, their evolution in time can be computed by differentiating Equation~(\ref{eq:sensi}) according to the following set of \myglsentry{odes}:
\begin{equation}\label{eq:dyna_sensi}
  \left \{
  \begin{array}{l l l}
       \dot{\bPi}(t) = \boldsymbol{\frac{\partial{f}}{\partial{q}}}\bPi+ \boldsymbol{\frac{\partial{f}}{\partial{u}}}\bTheta+ \boldsymbol{\frac{\partial{f}}{\partial{p}}}, \quad \bPi(t_0)=\bPi_0, \\
       \dot{\bPi}_{\xi}(t) = \boldsymbol{\frac{\partial{g}}{\partial{q}}}\bPi+ \boldsymbol{\frac{\partial{g}}{\partial{\xi}}}\bPi_{\xi}, \quad \bPi_{\xi}(t_0)=\bPi_{\xi0}, \\
       \bTheta(t) = \boldsymbol{\frac{\partial{h}}{\partial{q}}}\bPi+ \boldsymbol{\frac{\partial{h}}{\partial{\xi}}}\bPi_{\xi} 
  \end{array}
  \right .
\end{equation}
where $\bPixi(t)\in \mathbb{R}^{n_{\xi} \times n_p}$ represents the \emph{internal state sensitivity} matrix.

Now that the sensitivity matrices are defined, one can optimize the desired trajectory s.t. a norm of these matrices is minimized (see~\cite{cPi,cTh}). 
This optimization produces a trajectory s.t. the closed-loop evolution of $\q(t)$/$\u(t)$ closely matches its evolution in the nominal case $\q_n(t)$/$\u_n(t)$.

\subsection{Tube computation}\label{sec:tubes}

\begin{figure} [t]
  \centering
  \includegraphics[width=0.8\linewidth]{figures/models/tubes.png} 
  \caption{Illustration of the uncertainty tube (red) and a perturbed trajectory $\q(t)$ (blue), centered around the nominal trajectory $\q_n(t)$ (green), which results from following the reference trajectory $\q_d(t)$ (dashed black).}%
  \label{fig:tubes}%
\end{figure}

\begin{figure} [t]
  \centering
  \includegraphics[width=0.6\linewidth]{figures/models/radius.png} 
  \caption{2D representation of an uncertainty ellipse (green) in the x-y state space centered at $\q_n$, along with the tube radius (red) that illustrates the worst-case deviations along each state space components.}%
  \label{fig:ellips_radius}%
\end{figure}

Another important feature of the sensitivity matrices is that it is possible to derive the so-called \emph{uncertainty tubes}, that bounds the closed-loop system trajectory $\q(t)$/$\u(t)$ around its nominal trajectory $\q_n(t)$/$\u_n(t)$ as shown in~\cite{cTube} and illustrated in Figure~\ref{fig:tubes}.
The rest of this section shows how to establish such bounds around the nominal state trajectory $\q_n(t)$ by focusing solely on the state-sensitivity matrix $\bPi(t)$ for clarity. 
However, it is important to note that the same procedure can also be applied to compute bounds around the nominal input trajectory $\u_n(t)$ by leveraging the input-sensitivity matrix $\bTheta(t)$.

The uncertainty tube for each component of the state is characterized by a \emph{radius} which bounds the state component evolution from its nominal value over time, i.e. for the $i$-th component of the state ($q_i(t)$) the tube radius $r_{q,i}(t)$ is defined s.t.:
\begin{equation}\label{eq:bounds_q}
  q_{n,i}(t) - r_{q,i}(t) \leq q_i(t) \leq q_{n,i}(t) + r_{q,i}(t).
\end{equation}

Let $\Delta\q(t) = \q(t) - \q_n(t)$, representing the deviation of the perturbed trajectory from the nominal one, which we seek to bound.
Assume that for each uncertain parameter $p_{i}, i \in [1, n_p]$ in $\p$, we have a bounded parameter deviation $\delta p_i \in \mathbb{R}$ s.t. 
\begin{equation*}
  \forall i \in [1, n_p] ,p_i \in [p_{n_i}-\delta p_i, p_{n_i}+\delta p_i].
\end{equation*}
Such deviations can be mapped into the parameter space by mean of the following ellipsoid
\begin{equation}\label{eq:p_ellipsoid}
  \Delta\p^T \bW^{-1} \Delta\p \leq 1,
\end{equation}
where $\bW$ is the following diagonal weight matrix
\begin{equation*}
  \bW = \begin{bmatrix}
    \delta p_1^2 & 0 & \cdots & 0 \\
    0 & \delta p_2^2 & \cdots & 0 \\
    \vdots & \vdots & \ddots & \vdots \\
    0 & 0 & \cdots & \delta p_{n_p}^2
    \end{bmatrix} \in \mathbb{R}^{n_p \times n_p}.
\end{equation*}

Assuming small parameters variations (i.e. small $\delta \p$) it is possible to perform a first-order approximation around the nominal trajectory $\q_n(t)$ to obtain 
\begin{equation}\label{eq:approx}
  \Delta\q(t) =  \q(t) - \q_n(t) \approx \bPi(t) \Delta\p.
\end{equation}

Without loss of generality, for a well-chosen $\delta \p$ s.t. Equation~(\ref{eq:approx}) holds, ~\cite{cTube} has shown how to map the parameters ellipsoid from Equation~(\ref{eq:p_ellipsoid}) in the system state space in order to obtain the corresponding uncertainty ellipsoid
\begin{equation}\label{eq:q_ellipsoid}
  \Delta\q(t)^T (\bPi(t) \bW \bPi(t)^T)^\dag \Delta\q(t) \leq 1.
\end{equation}
However, it is important to note that the ellipsoid axes are in general not aligned with the canonical basis of the state space.
This implies that computing the deviation of each state component $q_i(t)$ is not straightforward, as direct use of the ellipsoid semi-axes length is not possible and the direction of interest may not belong to the range of the ellipsoid.

Nevertheless, it has been shown in ~\cite{cTube} how to obtain the tube radius along the $i$-th component of the state $q_i(t)$ by mean of the following projection:
\begin{equation}\label{eq:radius}
  r_{q,i}(t) =  \sqrt{\boldsymbol{n_i}^{T} \boldsymbol{K_{\Pi}}(t) \boldsymbol{n_i}},
\end{equation}
where $\boldsymbol{n}_i$ is the unit-norm vector of the $i$-th state space component, and $\boldsymbol{K_{\Pi}}(t) = \boldsymbol{\Pi}(t)\boldsymbol{W}\boldsymbol{\Pi}(t)^T$ is the kernel matrix of Equation~(\ref{eq:q_ellipsoid}).
It is important to note that the radius along the $i$-th component represents the maximum possible deviation in that direction and does not directly correspond to the semi-axes of the uncertainty ellipsoid as depicted in Figure~\ref{fig:ellips_radius}.

Finally, it is important to underline that since the uncertainty tube radii depend on the state sensitivity $\boldsymbol{\Pi}(t)$, it is necessary to solve the set of \myglsentry{odes} represented by Equation~(\ref{eq:dyna}), Equation~(\ref{eq:ctrl}), and Equation~(\ref{eq:dyna_sensi}) beforehand.
Additionally, note how, in general, the tube radius does not necessarily increase monotonically due to the influence of feedback action as illustrated in Figure~\ref{fig:tubes}, despite the cumulative effect of uncertainties over time. 

\section{Models considered in this thesis}

To demonstrate the versatility of the approaches proposed in this thesis across various robots and controllers, a quadrotor and a differential drive robot are considered in this manuscript, along with two distinct controllers and interpolation methods, which are detailed in the following sections.

\subsection{Quadrotor robot} \label{sec:quad_model}

\subsubsection{Dynamic model}

\begin{figure} [t]
    \centering
    \includegraphics[width=0.8\linewidth]{figures/models/drone.png} 
    \caption{Quadrotor robot representation with a shift in the center of mass.}%
    \label{fig:quad}%
\end{figure}

Let the ENU (East North Up) world frame be defined as $F_W \allowbreak = \allowbreak \{O_W, \allowbreak X_W, \allowbreak Y_W, \allowbreak Z_W\}$ and $F_B = \allowbreak\{O_B, \allowbreak X_B, \allowbreak Y_B, \allowbreak Z_B\}$ be the quadrotor body frame attached to its geometric center ($O_B$).
The state of the quadrotor is defined as $\boldsymbol{q} = [\boldsymbol{x}  \, \boldsymbol{v} \, \boldsymbol{\rho} \, \boldsymbol{\omega}]$ where $\boldsymbol{x} = [x \, y \,z] \in \mathbb{R}^{3}$ and $\boldsymbol{v} = [v_x \, v_y \,v_z] \in \mathbb{R}^{3}$ are respectively the position and linear velocity vector of $O_B$ expressed in $F_W$. The body orientation w.r.t. $F_W$ is represented by the unitary quaternion $\boldsymbol{\rho}$ and its angular velocity as $\boldsymbol{\omega} = [\omega_x \, \omega_y \, \omega_z] \in \mathbb{R}^{3}$. 
Finally, let $\boldsymbol{R(\rho)}$ be the rotation matrix associated to $\boldsymbol{\rho}$.

Let the control input vector $\u = [\omega_{1}^2 \, \omega_{2}^2 \, \omega_{3}^2 \, \omega_{4}^2]^T$ represents the squared rotor speeds. 
The vector $\u$ is related to the total propeller thrust $f$ and torques $\boldsymbol{\tau}$ by mean of a standard allocation matrix $\boldsymbol{T}$ s.t.
\begin{equation}\label{eq:alloc_mat}
  \begin{bmatrix}
    f \\
    \boldsymbol{\tau}
    \end{bmatrix} = \boldsymbol{T}(l, k_f, k_{\tau}) \u,
\end{equation}
where $k_f$ and $k_{\tau}$ refer to the thrust and drag coefficients of the propellers respectively, and $l$ stands for the quadrotor arm length.
Also, consider that the \gls{com} is displaced from the robot geometric center of an offset $\boldsymbol{x_{c}} = [x_{cx} \, x_{cy} \, x_{cz}]$ expressed in $F_B$ as depicted in Figure~\ref{fig:quad}. 
This displacement can occur due to onboard sensors, the presence of a payload, or other factors.
Under this consideration and according to Newton's second law, the total force ($\boldsymbol{f_{tot}}$) and torque ($\boldsymbol{\tau_{tot}}$) acting on the quadrotor can be expressed by taking an additional fictitious force due to the displaced \myglsentry{com} in $F_B$ s.t. 
\begin{equation}\label{eq:quad_newton}
    \begin{array}{@{}l@{}l@{}}
        \boldsymbol{f_{tot}} &= f Z_W - mg\boldsymbol{R(\rho)}^TZ_W-m[\boldsymbol{\omega}]_{\times}[\boldsymbol{\omega}]_{\times}\boldsymbol{x_c} 
          
          \\
      
       \boldsymbol{\tau_{tot}} &= \boldsymbol{\tau}-mg[\boldsymbol{x_c}]_{\times}\boldsymbol{R(\rho)}^{T}Z_W - [\boldsymbol{\omega}]_{\times}\boldsymbol{J}\boldsymbol{\omega}
  \end{array}
\end{equation}
where $m$ is the mass and $\boldsymbol{J}$ is the inertia matrix of the system. 

By considering the spatial inertia matrix
\[
\boldsymbol{S} =   \left( {\begin{array}{cc}
    m\boldsymbol{I_3} & -m[\boldsymbol{x_c}]_{\times} \\
    m[\boldsymbol{x_c}]_{\times} & \boldsymbol{J} \\
  \end{array} } \right)
\]
one finally gets the body frame linear acceleration $\boldsymbol{\alpha}$ and angular acceleration $\boldsymbol{\eta}$
as: 
$
\left( 
    \boldsymbol{\alpha}^T \; \boldsymbol{\eta}^T \right)^T 
  =
  \boldsymbol{S}^{-1}
  \left( \boldsymbol{f_{tot}}^T \;
    \boldsymbol{\tau_{tot}}^T \right)^T. 
$
The dynamic model is then defined as follows:
\begin{equation}\label{eq:quad_dynamic}
    \Dot{\q}
    =
     \left \{
     \begin{array}{l l}
           \dot{\boldsymbol{x}} = \boldsymbol{v} \\
           
           \dot{\boldsymbol{v}}= \boldsymbol{\alpha} \\

           \dot{\boldsymbol{\rho}}=\frac{1}{2}
               \boldsymbol{\rho} \otimes \begin{bmatrix}
                                          0 \\
                                          \boldsymbol{\omega}
                                          \end{bmatrix} \\
           
           \dot{\boldsymbol{\omega}}=\boldsymbol{\eta}
   \end{array}
   \right . 
\end{equation}
In line with Section~\ref{sec:sensi}, let the vector $\p = [m \, x_{cx} \, x_{cy} \, x_{cz} \, J_{x} \, J_{y} \,J_{z} \, k_f \, k_{\tau}]^T \in \mathbb{R}^{9}$ represent all the parameters used in the robot model outlined above that are subject to uncertainty. 
This vector can be adjusted based on the various scenarios presented in this thesis.

Finally, as tracking controller, we consider the widely used Lee (or geometric) controller~\cite{cLee} to computes the control input vector $\u$.
This controller takes advantage of the well-known quadrotor flat outputs ~\cite{cFlat} to track a desired trajectory defined as $\q_d(t) = [\boldsymbol{x}_d(t) \, \boldsymbol{v}_d(t) \, \boldsymbol{a}_d(t) \, \psi_d(t) \, \Dot{\psi}_d(t)]^T \in \mathbb{R}^{11}$ respectively composed of the desired linear positions, velocities, accelerations, yaw orientation angle, and yaw angular velocity.
Note that, since the quadrotor is an under-actuated system, not all states can be controlled. This is why only the desired yaw angle is incorporated into the desired trajectory.

Let the linear position error and linear velocity error be defined as follows:
\begin{equation}\label{eq:pos_error}
  \boldsymbol{e_x}(t) = \boldsymbol{x}(t) - \boldsymbol{x}_d(t) \in \mathbb{R}^3, \,
  \boldsymbol{e_v}(t) = \boldsymbol{v}(t) - \boldsymbol{v}_d(t) \in \mathbb{R}^3,
\end{equation}
and the controller gains vector \(\boldsymbol{k}_c = [\boldsymbol{k}_{x}^T \, \boldsymbol{k}_{v}^T \, \boldsymbol{k}_{R}^T \, \boldsymbol{k}_{\omega}^T]^T \in \mathbb{R}^{12}\) with their associated diagonal representation $(\boldsymbol{K_x}, \boldsymbol{K_v}, \boldsymbol{K_R}, \boldsymbol{K_\omega}) \in (\mathbb{R}^{3\times3})^4$.
This control strategy starts by computing the desired third basis vector of the body frame:
\begin{equation}
  \boldsymbol{r}_{3d} = \frac{-\boldsymbol{K_x} \cdot \boldsymbol{e_x} - \boldsymbol{K_v} \cdot \boldsymbol{e_v} + m(g Z_W + \boldsymbol{a}_d) }{\left\| -\boldsymbol{K_x} \cdot \boldsymbol{e_x} - \boldsymbol{K_v} \cdot \boldsymbol{e_v} + m(g Z_W + \boldsymbol{a}_d) \right\|} \in \mathbb{R}^3.
\end{equation}
Then, using the desired yaw angle $\psi_d$, the desired heading vector can be defined as $\boldsymbol{\Omega_{\psi_d}} = [cos(\psi_d) \, sin(\psi_d) \, 0]^T$.
This heading vector is then used to compute the desired first and second basis vectors of the body frame:
\begin{equation}
  \boldsymbol{r}_{2d} = \frac{\left[\boldsymbol{r}_{3d}\right]_{\times} \cdot \boldsymbol{\Omega}_{\psi_d}}{\left\|\left[\boldsymbol{r}_{3d}\right]_{\times} \cdot \boldsymbol{\Omega}_{\psi_d}\right\|} \in \mathbb{R}^3, \, \boldsymbol{r}_{1d} = \left[\boldsymbol{r}_{2d}\right]_{\times} \cdot \boldsymbol{r}_{3d} \in \mathbb{R}^3.
\end{equation}
From the three body frame basis vectors it is then possible to define the desired rotation matrix $\boldsymbol{R}_d(\psi_d, t) = [\boldsymbol{r}_{1d} \, \boldsymbol{r}_{2d} \, \boldsymbol{r}_{3d}]$, and to compute the following attitude error:
\begin{equation}\label{eq:att_error}
  \boldsymbol{e_R}(t) = \frac{1}{2}[\boldsymbol{R}_d(\psi_d, t)^T \cdot \boldsymbol{R}(\boldsymbol{\rho}, t) - \boldsymbol{R}(\boldsymbol{\rho}, t)^T \cdot \boldsymbol{R}_d(\psi_d, t)^T]^\vee \in \mathbb{R}^3.
\end{equation}
Finally, to simplify the overall controller design, the desired yaw rate $\Dot{\psi}_d$ is always set to zero in this thesis, allowing the angular velocity error to be expressed as:
\begin{equation}\label{eq:rate_error}
  \boldsymbol{e_\omega}(t) = \omega(t) \in \mathbb{R}^3.
\end{equation}

According to the aforementioned errors, this control strategy computes the desired propeller thrust $f_d(t)$ and desired propeller torques $\boldsymbol{\tau}_d(t)$ that allow the tracking of the desired trajectory $\q_d(t)$:
\begin{equation}\label{eq:desftau}
  \left \{
    \begin{array}{l l}
      f_d(t) &= \left( -\boldsymbol{K_x} \cdot \boldsymbol{e_x}(t) - \boldsymbol{K_v} \cdot \boldsymbol{e_v}(t) + m(g Z_W + \boldsymbol{a}_d(t)) \right)^T \cdot (\boldsymbol{R}(\boldsymbol{\rho}, t) \cdot Z_W) \\
      \boldsymbol{\tau}_d(t) &= -\boldsymbol{K_R} \cdot \boldsymbol{e_R}(t) - \boldsymbol{K_\omega} \cdot \boldsymbol{e_\omega}(t)
    \end{array}
  \right .
\end{equation}
Note that the choice of setting the desired yaw rate to zero cancels terms in the expression of $\boldsymbol{\tau}_d(t)$ compared to the original expression in ~\cite{cLee}.

Finally, the control input vector $\u$ can be computed using the same standard allocation matrix $\boldsymbol{T}$ from Equation~\ref{eq:alloc_mat} using nominal parameters denoted $\boldsymbol{T}_n$ s.t.:
\begin{equation}\label{eq:control_lee}
  \u = \boldsymbol{T}_n^{-1}\begin{bmatrix}
        f_d \\
        \boldsymbol{\tau}_d
      \end{bmatrix}.
\end{equation}
Note that under nominal case (i.e. when $\p = \p_n$), $\boldsymbol{T} = \boldsymbol{T}_n$; thus, the propeller thrust and torques applied to the real system, as for $f$ and $\boldsymbol{\tau}$ in Equation~\ref{eq:quad_newton}, perfectly match the desired thrust and torques computed by the control law.
Note also that no internal controller states are considered in this controller; therefore, Equation~\ref{eq:dyna_sensi} simplifies to:
\begin{equation}\label{eq:dyna_sensi_simp}
  \left \{
  \begin{array}{l l}
       \dot{\bPi}(t) = \boldsymbol{\frac{\partial{f}}{\partial{q}}}\bPi+ \boldsymbol{\frac{\partial{f}}{\partial{u}}}\bTheta+ \boldsymbol{\frac{\partial{f}}{\partial{p}}}, \quad \bPi(t_0)=\bPi_0, \\\\
       \bTheta(t) = \boldsymbol{\frac{\partial{\eta}}{\partial{q}}}\bPi 
\end{array}
\right .
\end{equation}

\subsubsection{State interpolation}\label{sec:kinosplines}

With the quadrotor model and controller now defined, this section presents the interpolation method (hereafter referred to as the \emph{local} planner) for computing a desired trajectory that the geometric controller will track.

Given an initial desired state $\q_d^0 = [\boldsymbol{\gamma}^0 \, \boldsymbol{\Dot{\gamma}}^0 \, \boldsymbol{\Ddot{\gamma}}^0] \in \mathbb{R}^{12}$ and a goal state $\q_d^F = [\boldsymbol{\gamma}^F \, \boldsymbol{\Dot{\gamma}}^F \, \boldsymbol{\Ddot{\gamma}}^F] \in \mathbb{R}^{12}$ for the quadrotor, where $\boldsymbol{\gamma} = [x_{d} \, y_{d} \, z_{d} \, \Psi_{d}]^T \in \mathbb{R}^{4}$ represents the desired position along the x, y, and z axes and the desired yaw angle, the \emph{kino-spline} method from~\cite{cKino} is employed to plan a time optimal and continuous desired trajectory $\q_d(t)$ connecting $\q_d^0$ and $\q_d^F$. 

The trajectory generation problem is solved independently for each component of $\boldsymbol{\gamma}$.
Furthermore, in order to generate smooth trajectories in a global context, the local planner ensures the continuity of their derivatives up to the jerk (acceleration derivative). 
Note that even if the yaw angular acceleration is not required by the control law of Section~\ref{sec:quad_model}, the initial and goal yaw angular accelerations are used to ensure the smoothness of the generated trajectory. 
As for both initial and goal jerk, they are set to zero to facilitate a smooth transition between trajectories, and are therefore not considered in $\boldsymbol{\gamma}$.
Finally, the kino-spline method also consider bounds on the several derivatives up to the snap (i.e. $v_{max}, a_{max}$, etc.) and plan the trajectory accordingly.

As previously mentioned, this local planner focuses on generating local trajectories that are time optimal, i.e. that minimize the total time needed to reach $\q_d^F$.
This is achieved by reaching the full speed as soon as possible and by maintaining it as long as possible for each component of $\boldsymbol{\gamma}$.
This also implies that the time spent reaching this velocity must be minimized, which means that the time spent at maximum acceleration during transient phases must be maximized.
The same principle applies to jerk and snap; during variations in acceleration, the maximum achievable jerk should be maintained for as long as possible, while the duration of jerk variation phases is minimized.
This is achieved through a straightforward bang-bang snap method, which aims to maximize the time spent at maximum snap during the jerk transient phases.
By doing so, the duration of transient phases is minimized, and gradually, the duration of the maximum speed phase is maximized.
The principle of this trajectory generation is illustrated by Figure~\ref{fig:kino}.

\begin{figure} [t]
  \centering
  \includegraphics[width=0.99\linewidth]{figures/models/kino.png} 
  \caption{Example of a trajectory generated by the local planner for a single coordinate. The pink dashed lines indicate the bounds for each derivative, while the red and green dashed lines represent the initial and final values, respectively. 
  The letters A, B, C, D, E, G, and H represent the different phases that need to be minimized or maximized during the trajectory planning process.(extracted from~\cite{cKino})}%
  \label{fig:kino}%
\end{figure}

\subsection{Differential drive robot}\label{sec:unic_model}

\subsubsection{Dynamic model}

\begin{figure} [t]
  \centering
  \includegraphics[width=0.6\linewidth]{figures/models/unicycle.png} 
  \caption{Illustration of the main quantities characterizing the differential drive robot.}%
  \label{fig:unicycle}%
\end{figure}

Let the world frame be defined as $F_W \allowbreak = \allowbreak \{O_W, \allowbreak X_W, \allowbreak Y_W\}$ and $F_B = \allowbreak\{O_B, \allowbreak X_B, \allowbreak Y_B\}$ be the differential drive robot body frame attached to its geometric center ($O_B$).
The robot state is defined as $\boldsymbol{q} = [\boldsymbol{x} \, \theta] \in \mathbb{R}^3$ where $\boldsymbol{x} = [x, \, y] \in \mathbb{R}^2$ are the linear positions of $O_B$ in $F_W$ and $\theta$ is the body heading (see Figure~\ref{fig:unicycle}).

Define the control input vector $\u = [\omega_r \, \omega_l]^T \in \mathbb{R}^2$, where $(\omega_r, \, \omega_l)$ are respectively the right and left wheel angular velocity.
Let also the robot linear and angular velocities be denoted by $v$ and $\omega$, respectively.
The aforementioned linear and angular velocities are related to the control input vector by mean of an allocation matrix $\boldsymbol{T}$ s.t.:
\begin{equation}\label{eq:unic_alloc}
  \begin{bmatrix}
    v \\
    \omega
  \end{bmatrix}
  =
  \begin{bmatrix}
    \frac{r}{2} & \frac{r}{2}\\
    \frac{r}{2d} & \frac{-r}{2d}
  \end{bmatrix}
  \begin{bmatrix}
    \omega_r \\
    \omega_l
  \end{bmatrix}
  =
  \boldsymbol{T} \u,
\end{equation}
where $d$ stands for the length between the two wheels, and $r$ for wheel radius.

The differential drive robot dynamic is then expressed as:
\begin{equation}\label{eq:unic_dyna}
  \Dot{\q} = 
  \begin{bmatrix}
    cos(\theta) & 0 \\
    sin(\theta) & 0 \\
    0 & 1
  \end{bmatrix} \boldsymbol{T} \u.
\end{equation}
In line with Section~\ref{sec:sensi}, the vector of parameters used in the robot model outlined above that are subject to uncertainty is defined as $\p = [r \, d]^T \in \mathbb{R}^{2}$.

The control task is to let the robot positions $\boldsymbol{x}$ track desired positions $\boldsymbol{x}_d = [x_d, \, y_d] \in \mathbb{R}^2$.
The differential drive robot is an under-actuated system, which is why the heading is excluded from the tracking process; it is induced by the robot dynamic instead.

To perform the tracking task, a \myglsentry{dfl} controller is used (e.g. see~\cite{cDFL}).
This control strategy considers an extended desired trajectory $\q_d(t) = [\boldsymbol{x}_d(t) \, \dot{\boldsymbol{x}}_d(t) \, \ddot{\boldsymbol{x}}_d(t)]^T \in \mathbb{R}^6$, where $\Dot{\boldsymbol{x}}_d(t) \in \mathbb{R}^2$ denote the desired linear velocities, and $\ddot{\boldsymbol{x}}_d(t) \in \mathbb{R}^2$ are the desired linear accelerations.
This trajectory is tracked by mean of the following controller internal states $\boldsymbol{\xi} = [\xi_v \, \xi_x \, \xi_y]^T \in \mathbb{R}^3$, where $\xi_v$ stands for the dynamic extension of the differential drive robot linear velocity $v$ (see Figure~\ref{fig:unicycle}), and $(\xi_x, \, \xi_y)$ are integral actions. 

By differentiating the robot positions twice, one obtains the following equations for the robot linear accelerations:
\begin{equation*}
  \ddot{\boldsymbol{x}}
  = 
  \begin{bmatrix}
    cos(\theta) & -\xi_v sin(\theta) \\
    sin(\theta) & \xi_v cos(\theta) 
  \end{bmatrix} 
  \begin{bmatrix}
    \dot{v} \\
    \omega
  \end{bmatrix}
  = \boldsymbol{A} \begin{bmatrix}
    \dot{v} \\
    \omega
  \end{bmatrix}
\end{equation*}
This differentiation is essential for formulating the kinematic relationships that enable efficient tracking of the desired positions.

Let the following vectors:
\begin{equation}\label{eq:unic_control_law}
  \left \{
  \begin{array}{l l l}
       \dot{\boldsymbol{x}}_{\xi} = [cos(\theta)\xi_v \, sin(\theta)\xi_v]^T \\
       \boldsymbol{\xi}_{xy} = [\xi_x \, \xi_y]^T \\
       \boldsymbol{\varrho} = \boldsymbol{\ddot{x}}_d + k_v(\boldsymbol{\dot{x}}_d - \dot{\boldsymbol{x}}_{\xi}) + k_p(\boldsymbol{x}_{d} - \boldsymbol{x}) + k_i\boldsymbol{\xi}_{xy}
  \end{array}
  \right .
\end{equation}
where $k_v$, $k_p$ and $k_i$ are the controller gains, s.t. in the following chapters of this thesis, the controller gain vector is defined as $\boldsymbol{k}_c = [k_p, \, k_v \, k_i]^T \in \mathbb{R}^3$.

Without loss of generality, the dynamics of the internal control states can be expressed (refer to~\cite{cDFL} for further details) as follows:
\begin{equation}
  \dot{\boldsymbol{\xi}} = 
  \begin{bmatrix}
    \begin{bmatrix}1 & 0\end{bmatrix}\boldsymbol{A}^{-1}\boldsymbol{\varrho} \\
    \boldsymbol{x}_d - \boldsymbol{x}
  \end{bmatrix}, 
\end{equation}
and the control inputs as:
\begin{equation}
  \u = \boldsymbol{T}_n^{-1} 
  \begin{bmatrix}
    \xi_v \\
    \begin{bmatrix}0 & 1\end{bmatrix}\boldsymbol{A}^{-1}\boldsymbol{\varrho}
  \end{bmatrix},
\end{equation}
where $\boldsymbol{T}_n$ represent the same allocation matrix as in Equation~\ref{eq:unic_dyna} but evaluated on the nominal parameters $\p_n$.

\subsubsection{State interpolation}\label{sec:dubins}

Now that the dynamic of the differential drive robot and its controller are defined, this section presents the local planner used to generate the desired trajectory $\q_d(t) = [\boldsymbol{x}_d(t) \, \dot{\boldsymbol{x}}_d(t) \, \ddot{\boldsymbol{x}}_d(t)]^T \in \mathbb{R}^6$ that is tracked by the \myglsentry{dfl} controller.
The trajectory generation problem is addressed using Dubins curves~\cite{cDubins}, which are differentiated twice. 

This approach ensures smooth and continuous trajectories between two position $(x^0, y^0)$ and $(x^F, y^F)$ in the x-y plane for systems with constraints on turning radius and forward motion such as the differential drive robot.
The method takes advantage of specified initial and final tangents to the curve (e.g. the differential drive robot heading angle $\theta$, in this case) to create smooth curves that respect the turning constraints.
A Dubins curve is composed of three possible segment types: left turn, right turn, and straight line as depicted in Figure~\ref{fig:dubins}. 
Each segment is designed with an optimal distance, arranged to create the shortest path between the two points $(x^0, y^0)$ and $(x^F, y^F)$. 
This configuration results in one of six possible combinations of segments (e.g., left-straight-right), producing a smooth, continuous path that meets the system turning and heading requirements.

In this thesis, the robot state used to compute such path is defined as $\boldsymbol{\gamma} = [x \, y \, \theta] \in \mathbb{R}^3$.
Once the path is computed, the x and y components are discretized using a given time step and then differentiated to obtain the corresponding desired velocities and accelerations required by Equation~\ref{eq:unic_control_law}.
Note that, although the heading is not directly controlled, a desired heading is specified to satisfy the tangent constraints.

\begin{figure} [t]
  \centering
  \includegraphics[width=0.4\linewidth]{figures/models/dubins.png} 
  \caption{Illustration of two possible Dubins paths: (top) a right turn, followed by a straight segment, then a left turn; (bottom) a right turn, followed by a left turn, then another right turn. (extracted from~\cite{cFigDubins})}%
  \label{fig:dubins}%
\end{figure}
% \cleardoublepage
% \newpage 
% \ % The empty page
% \newpage
\chapter{Sensitivity-aware motion planner}
\markboth{Sensitivity-aware motion planner}{}% To set left/right header
% \localtableofcontents

Now that the sensitivity matrices are defined, one can compute and minimize a chosen norm of $\bPi(t)$ and $\bTheta(t)$ w.r.t. the desired trajectory $\q_d(t)$. 
This optimization generates a desired trajectory with minimal sensitivity, ensuring that the closed-loop evolution of $\q(t)$/$\u(t)$ closely follows its evolution in the nominal case $\q_n(t)$/$\u_n(t)$.

\section{Unified approach}
\subsection{Algorithm}
\subsubsection{Robust collision checking}
The tubes can leverage to perform robust collision check with the environments but also in the control inputs space to avoid actuators saturation. 
Collision checks with the environments are only performed by considering uncertainty in the x y z subspace.
Several strategies to perform robust collision with the environments:
\begin{enumerate}
    \item One can simply considered a convex bounding shape of the robot and enlarge it by the maximum radius (i.e. the maximum of the worst case deviations). See appendinx for proof.
    \item Compute the bonding box that contains the ellipsoid. Note that even if in this thesis this is the method employed to perform robust collision check, by mean of smooth visualization we depicted the tubes using ellipsoid representation in the various figures.
    \item Compute the ellipsoid semi-axes, and it's orientation w.r.t. the canonical basis of the subspace. However, such orinetation and semi-axis lenght can be obtain by decomposing the kernel into eigenvalues and eigenvectors representation that leads to higher computation times that solely computing the worst deviations. For the quadrotor case, 200 ellipsoids where computed and the time ratio for the bounding box method over the true ellipsoid computation shows 18\% times and the volume shows 172\%, meaning that bounding box faster but a bit conservative as it's approx 1.72 times bigger. Faster because computing the semi axes lenght requires to pseudo invert K.
\end{enumerate}
\subsection{Simulation results}
\subsubsection{Differential drive robot}
\subsubsection{Quadrotor robot}

\section{Decoupled approach}
\subsection{Algorithm}
\subsection{Simulation results}
\subsubsection{Differential drive robot}
\subsubsection{Quadrotor robot}

\section{Conclusion}

\todomarker{}
% \cleardoublepage
% \newpage 
% \ % The empty page
% \newpage
\chapter{Learning uncertainty tubes via recurrent neural networks} \label{chap:NN}
\markboth{Learning uncertainty tubes via recurrent neural networks}{}% To set left/right header
% \localtableofcontents

% Parallelization of ODE (Ordinary Differential Equations) solvers for systems like a quadrotor is indeed possible, but it comes with certain challenges due to the inherent sequential nature of ODE solvers. Each time step generally depends on the previous one, making it difficult to parallelize directly. However, various techniques and strategies can help in parallelizing ODEs, even for small time steps. Let's explore this in more detail.
% 1. Parallelization in ODE Solvers

% ODE solvers like Runge-Kutta, Euler, or Adams-Bashforth typically compute solutions in a time-stepping manner, where each step depends on the solution from the previous step. This sequential dependency makes it hard to directly parallelize time integration.

% However, some strategies can parallelize parts of the problem:
% 1.1 Parallelizing the Evaluation of Derivatives

% The right-hand side (RHS) of the ODE system (i.e., the function that describes the system's dynamics) can often be parallelized if it involves multiple independent computations. For example, in a quadrotor, if the dynamics involve complex matrix operations, vector algebra, or sensor models, these computations can be parallelized across multiple cores or even on a GPU.
% 1.2 Domain Decomposition for Parallelization

% This approach involves splitting the physical domain or the variables of the system and solving them in parallel. For a quadrotor, domain decomposition may not apply as easily as it would for a large physical domain problem (like fluid dynamics), but it could still be used if the quadrotor's dynamics are complex and involve interacting subsystems. For instance, rotor dynamics, attitude control, and position dynamics could potentially be solved in parallel if they are weakly coupled.
% 1.3 Multiple Shooting or Parareal Algorithms

% These methods introduce parallelization by splitting the time domain into multiple intervals and solving the ODE in parallel on each subinterval.

%     Parareal Algorithm: This method allows for parallel time integration by solving coarse approximations over large time intervals in parallel, followed by corrections using fine time steps. While the fine steps remain sequential, the coarse steps can be parallelized. The Parareal method is well-suited for large-scale systems but might be overkill for small systems like a quadrotor, especially with small time steps.

%     Multiple Shooting: In this technique, the time domain is divided into smaller intervals, and the ODE is solved independently in each interval. The solutions are then stitched together using boundary conditions. This method is typically used for stiff ODE problems or optimization-based control (like Model Predictive Control).

% 1.4 Parallel Integration of Different States

% In some cases, the state variables of a system might evolve independently or with weak coupling, allowing for parallel integration of different states. For a quadrotor, some aspects of the dynamics (e.g., attitude and position) could potentially be integrated separately in parallel, especially if the time scales are slightly different.
% 2. Time-Stepping Constraints and Relaxation

% You mentioned time-step constraints. In ODE solvers, especially explicit solvers like Runge-Kutta or Euler, the time step size is usually limited by stability constraints (e.g., Courant-Friedrichs-Lewy (CFL) condition for PDEs or step size for stiff ODEs).

%     Relaxing Time-Step Constraints: Parallelization techniques like Parareal introduce time-step relaxation by allowing coarse-grain parallel computations over larger time intervals, then correcting with smaller steps sequentially. This is useful for accelerating convergence in large-scale simulations with longer time horizons.

%     Using Small Time Steps in Parallelization: If your goal is to maintain small time steps (e.g., for accuracy, stability, or real-time control in a quadrotor), relaxation-based methods might not be as helpful. However, parallelization can still help in other ways:
%         Task-based Parallelism: Use parallel computing resources to speed up the evaluation of forces, torques, control inputs, sensor models, or optimization procedures (if using real-time control).
%         Predictor-Corrector Methods: These methods can use a fast, coarse-grained predictor step, followed by parallel corrections that refine the solution. The corrections can be done in parallel with small time steps, making it more suitable for small time steps.
%         Parallelization of Subsystems: Parallelizing distinct components of the quadrotor's dynamics (e.g., thrust dynamics, rotational dynamics, or environmental interactions) can improve computational efficiency while keeping the same small time steps.

% 3. Maintaining Small Time Steps in Parallelization

% To directly answer your question, you can use parallelization techniques while keeping small time steps, but the degree of parallelization might be more limited compared to approaches where time steps are relaxed. The most promising approaches in this scenario would be:

%     Parallel evaluation of the system's dynamics: Evaluating the dynamics (i.e., the forces and moments) in parallel could reduce computation time significantly, especially if your quadrotor model involves complex dynamics (e.g., aerodynamic drag, rotor dynamics, etc.).

%     Task-level parallelism: In real-time control applications, parallelizing sensor fusion, state estimation, and control computations (while maintaining small time steps) can improve performance.

%     Multiple shooting methods: Even with small time steps, the multiple shooting method can potentially parallelize certain segments of the time domain, but you'd need to stitch together the results carefully, ensuring stability.

% Conclusion

% While parallelization techniques like Parareal or Multiple Shooting often involve relaxing time-step constraints, it is still possible to parallelize certain parts of the quadrotor's ODE solver without relaxing the small time-step constraints. You can achieve this by focusing on parallelizing the evaluation of the system's dynamics, using task-level parallelism, or exploiting weakly coupled subsystems. However, the benefit of parallelization will likely be more constrained compared to systems where time-step constraints are relaxed.

% 3. Challenges of Using Parareal for Control Stability

% The Parareal method works by first running a coarse-grained simulation with larger time steps, followed by corrections using finer time steps. If the coarse solver uses large time steps, it could violate the control stability requirements during the initial stages of the Parareal iterations. This would make it unsuitable for real-time control if the initial approximation (coarse solver) is far from accurate, as the control system relies on frequent updates.



% Your GRU (Gated Recurrent Unit) neural network might be faster than other methods for solving an ODE on a CPU for several reasons related to its structure and the way neural networks handle data. The efficiency of GRU in this context largely depends on the number of operations it performs and how well it's optimized for CPU computations. Let's break it down:
% Why GRU is Faster for ODE Solving on CPU:

%     Fewer Parameters: GRUs are known for their simplicity compared to other recurrent neural networks (RNNs) like LSTMs. GRUs have fewer gates and parameters, making them computationally cheaper.
%         GRUs have only two gates (reset and update), while LSTMs have three gates (input, forget, output), which translates to fewer matrix multiplications per time step.
%         Fewer parameters also mean less memory consumption, which reduces the computational load when training or running the model.

%     Parallelism: Neural networks, including GRUs, can be structured in such a way that they process multiple data points (or time steps) in parallel, especially during training or inference. This can make GRUs more efficient than traditional methods for solving ODEs, such as Runge-Kutta, which might be more sequential in nature.

%     Reduced Complexity of Time-Stepping:
%         Numerical methods for solving ODEs, such as Runge-Kutta or implicit methods, often involve multiple function evaluations at each time step. GRUs, however, learn a representation of the system's dynamics, effectively "learning" a function approximation, which can be faster at runtime.
%         GRUs can use a single pass per time step (or batch of time steps), while traditional solvers may require many steps or iterations at each point.


% Model Type	Inference Time	Remarks
% RNN (GRU, LSTM)	Faster	Efficient for discrete, regular time steps; fixed computation per time step
% Neural ODE	Slower	Inference depends on ODE solver complexity and system dynamics

% In most practical scenarios, RNNs (GRUs, LSTMs) will offer faster inference compared to Neural ODEs due to the absence of ODE solvers and the simpler, more direct nature of their computations. Neural ODEs are more suited for scenarios where continuous dynamics are the priority, but this often comes at the cost of slower inference times.

% Time Steps vs. ODE Solver Integration:

%     In RNNs, the number of time steps is fixed, and you perform one forward pass per time step.
%     In Neural ODEs, the time steps are determined by the ODE solver, which can be adaptive. The solver may take many small steps or fewer large steps, depending on the system's complexity.

%     RNNs (GRU, LSTM, etc.):

%     RNNs are designed to handle discrete time sequences. They take input in the form of a time series with fixed time steps (e.g., t1,t2,t3,...) and learn to map these discrete states over time.
%     RNNs operate by passing hidden states from one time step to the next, effectively modeling the temporal correlations in the data.
%     LSTMs and GRUs are specific variants of RNNs that improve on the basic RNN structure by addressing the problem of vanishing/exploding gradients in long sequences. They do this using gating mechanisms that control the flow of information through time.
%     Since RNNs work with fixed time steps, they model time as discrete, and handling continuous dynamics requires fine-tuning, such as choosing a time step size.

% Neural ODEs:

%     Neural ODEs, on the other hand, model continuous-time dynamics. They replace the discrete sequence of states (as in RNNs) with a differential equation for the hidden states:
%     dh(t)dt=f(h(t),t,θ)
%     dtdh(t)=f(h(t),t,θ) where ff is a neural network and θθ are its parameters. The dynamics evolve continuously over time, and you can evaluate the state at any arbitrary time tt, not just at predefined discrete steps.
%     Neural ODEs use ODE solvers to numerically integrate the system over time, allowing for flexible handling of varying time steps and even irregular or non-uniform time intervals.


% The use of a simple GRU architecture instead of Neural ODEs or Physics-Informed Neural Networks (PINNs) to approximate ODE solutions in the context of the paper by Wasiela et al. likely stems from several practical advantages that GRUs offer, particularly in the domain of real-time robot motion planning and control. Here are the key reasons and advantages of this choice:
%     1. Efficiency and Simplicity in Real-Time Applications
    
%     GRUs are known for their computational efficiency, especially in real-time applications like robotics where fast decision-making is crucial. A GRU-based model can be trained relatively easily and deployed quickly for inference, allowing the robot to plan and adjust its motions on the fly.
    
%         Fewer Parameters: Compared to models like Neural ODEs or more complex recurrent architectures, GRUs have a simpler structure and fewer parameters, making them faster and more efficient during inference. This makes GRUs particularly well-suited for tasks where real-time performance is required, such as robot motion planning.
    
%     2. Handling Sequential Data with Temporal Dependencies
    
%     GRUs are designed to handle sequential data and are efficient at modeling temporal dependencies, which are common in dynamical systems. In the context of approximating ODEs for robot motion, where states evolve over time, GRUs can track the evolving state of the robot and make predictions about future states efficiently.
    
%         Unlike Neural ODEs, which require solving differential equations iteratively through a solver, GRUs can directly approximate the state transitions (i.e., how the system evolves in time) in a single forward pass per time step, making them faster for inference.
    
%     3. Avoiding the Complexity of ODE Solvers
    
%     Neural ODEs, while powerful, often rely on the use of ODE solvers during training and inference, which can be computationally expensive and require adaptive step sizes. These solvers often need to balance accuracy with computational speed, which can become a bottleneck in real-time applications.
    
%         GRUs, in contrast, do not require iterative solvers to propagate forward in time. Once trained, a GRU can generate the next state in the sequence in a single step, leading to much faster performance, especially in resource-constrained environments like robotics.
    
%     4. Flexibility and Generalization
    
%     GRUs are flexible in terms of what kind of sequential data they can handle. Unlike Physics-Informed Neural Networks (PINNs), which integrate physical constraints directly into the learning process (requiring knowledge of the system’s physics and explicit boundary conditions), GRUs can approximate ODEs without needing explicit knowledge of the underlying physical laws.
    
%         This makes GRUs a more flexible and practical choice when the goal is to approximate the system’s behavior based on empirical data, rather than being strictly bound to physical laws or differential equations. This is especially useful in real-world scenarios where the system dynamics may be complex or partially unknown.
    
%     5. Data-Driven Approach with Learned Dynamics
    
%     In applications where data-driven approaches are preferred (such as when the robot's environment is highly dynamic or uncertain), GRUs can learn to model the system’s dynamics based purely on input-output data without needing a detailed mathematical model.
    
%         Neural ODEs and PINNs require knowledge of the system’s governing equations (or their differential form), which may not always be available or easy to define in the context of robot motion planning with uncertainties. A GRU can capture complex non-linear dynamics without needing to explicitly model the differential equations.
    
%     6. Handling Uncertainty and Robustness
    
%     In the paper, the GRU is used to model Learned Uncertainty Tubes, which suggests that the authors are dealing with uncertain environments and need a way to account for this uncertainty in planning robust robot motions. GRUs are good at capturing uncertainty in sequential predictions because of their inherent ability to model and propagate information over time.
    
%         GRUs can be trained to model not just the system's trajectory, but also the potential deviations or uncertainties in future states. This allows the GRU-based model to handle uncertain environments more robustly and efficiently, which may be harder to achieve with Neural ODEs or PINNs due to the complexity of integrating uncertainty into those frameworks.
    
%     7. Empirical Success and Ease of Implementation
    
%     Another reason for choosing GRUs could simply be their empirical success in time-series forecasting and modeling dynamic systems. They are widely used and have proven to work well in practice. Additionally, GRUs are easier to implement and train compared to Neural ODEs or PINNs, which often require more sophisticated training setups, custom loss functions, and solvers.
%     Summary of Advantages of Using GRUs over Neural ODEs or PINNs:
    
%         Faster inference: GRUs don't need iterative ODE solvers, making them more efficient for real-time applications.
%         Simplicity: GRUs are simpler to implement and require fewer resources compared to Neural ODEs or PINNs.
%         Good at handling sequential data: GRUs excel at modeling time-series data and can approximate temporal dependencies effectively.
%         Data-driven and flexible: GRUs can approximate the dynamics of a system without needing a detailed mathematical model of the underlying physics.
%         Handling uncertainty: GRUs are well-suited for modeling uncertainty in robot motion planning, as they can track evolving dynamics and uncertainties over time.
    
%     By using GRUs, the authors likely balance between achieving real-time performance, modeling complex dynamics, and managing uncertainties—all essential for robust robot motion planning.

%     You're absolutely right. Since the authors in the mentioned paper want to bypass the ODEs and avoid calling traditional solvers (like Euler, Runge-Kutta, etc.), Neural ODEs (NODEs) are not suitable because NODEs inherently rely on ODE solvers during inference. Let's now consider whether Physics-Informed Neural Networks (PINNs) could be an alternative, given the situation.
%     Why PINNs Might Not Be the Right Fit in This Context
    
%     Even though PINNs integrate known ODEs into the learning process, there are several reasons why they might not be the best choice for this particular case where the authors want to completely bypass the ODEs:
    
%     2. PINNs May Not Be as Efficient in Real-Time
    
%         PINNs often require more computational resources and longer training times because they solve ODEs in a way that enforces the governing equations (through residuals and constraints). This can lead to slower inference times compared to using something like a GRU, which is purely data-driven and bypasses both the explicit and implicit use of ODEs.
    
%         In the context of robot motion planning—which requires fast, real-time predictions—PINNs might be overkill or too computationally demanding for the purpose, especially when the focus is on efficient approximation of motion without solving differential equations explicitly.
    
%     3. PINNs Are Better Suited When Physics Constraints Are Critical
    
%         PINNs excel when you want to incorporate physical constraints or laws explicitly during learning, particularly when data is sparse or noisy. However, in the case where the authors want to approximate the ODEs using a simple data-driven approach (like a GRU), it suggests that they are prioritizing speed and simplicity over strict adherence to the physics.
    
%         If the authors are confident that the GRU can capture the dynamics without relying on the exact physical laws during inference, then the use of PINNs—which are designed to enforce those physical laws—becomes unnecessary.

\section{Reccurent neural networks: overview}

\section{Method} \label{sec:method}

This section presents a multi-task learning neural network based on a GRU architecture, which takes as input the desired states of the dynamic system and outputs the nominal control inputs $\u_n$, as well as the uncertainty tubes around the states $\Rq$ and the control inputs $\Ru$. 

\subsection{Problem statement}

The goal is to train a neural network to approximate sensitivity-based uncertainty tubes, hence avoiding the computational cost of solving many \myglsentry{odes}. 
Moreover, the model aims at predicting the control inputs that the system will exert to successfully track a desired trajectory, in addition to the uncertainty tubes on these control inputs. 
Given a sequence of desired robot state vectors $\textbf{M} = \{\q_{d}^0, \q_{d}^1, ..., \q_{d}^n\}$ representing the desired robot's motion (i.e. a desired trajectory to follow), the task at hand is to learn a function $\boldsymbol{g}$ that estimates the radii of uncertainty tubes for each state $\q_{d}^k$ as well as the robot control inputs at this state such that:
\begin{equation}\label{eq:prob_statement}
\{\Rq^{k}, \Ru^{k},\u^{k}\} = \boldsymbol{g}(\q_{d}^0, \q_{d}^1, ... \q_{d}^{k-1})
\end{equation}
Since evaluating the function $\boldsymbol{g}$ on $\q_{d}^k$ depends on all the previous states of the robot in $\textbf{M}$, recurrent neural network architectures are a good fit given their ability to encode and accumulate temporal information while keeping inference time low.

\subsection{Neural network architecture}\label{sec:architecture}

\begin{figure} [t]
    \centering
    \includegraphics[width=0.8\linewidth]{figures/learning_quadrotor/SensiNN_GRU.png}%
    \caption{Representation of the proposed neural network architecture.
    }%
    \label{fig: NN}%
\end{figure}

A representation of the neural network architecture is presented in Figure~\ref{fig: NN}.
Blue blocks refer to the inputs of the model which are composed of an initial hidden state $h_{0}$ and of a sequence of desired robot states. 
For the quadrotor case, in order to make the learned model independent of workspace boundaries used during planning (i.e. robot position and orientation bounds) and initial robot orientations, only the desired linear velocities, accelerations, and yaw angular velocities ($\Dot{\Psi}_d$), are kept as the network input components.
Hence, the input to the neural network is a vector $\boldsymbol{\q_{in}^k} = [\boldsymbol{v_d} ,\, \boldsymbol{a_d} ,\, \Dot{\Psi}_d]^T$, where $k$ refers to the $k$-th state of the desired trajectory.

The outputs of the neural network correspond to the green blocks.
In the case of a quadrotor, it is composed of the predicted uncertainty tubes radii along the $\{x,y,z\}$-axis of the state $\Rq=$[$r_{x},\,r_{y},\,r_{z}]^T$, and the uncertainty tubes radii associated with the control inputs of the system $\Ru = [r_{u1},\,r_{u2},\,r_{u3},\,r_{u4}]^T$.
Moreover, the control input values $\u = [u_1 ,\, u_2 ,\, u_3 ,\, u_4]^T$ are required for the motion planning algorithm, and their computation depends on ODE forward integration (as shown in (Eq.~\ref{eq:dyna_ctrl})). Since our approach aims at eliminating the need for solving ODEs, we need to train the neural network to also predict the control inputs.
Finally, $h_{N}$ corresponds to the hidden state at the last point of our sequence (i.e., the last trajectory state).

At $k=0$, the first state in the input sequence $\boldsymbol{\q_{in}^0}$ and the initial hidden state $h_0$ are given to a single-layer GRU block with a hidden state size of 512, which outputs an updated hidden state $h_1$. 
The latter is then fed to a 3-layer multi-layer perceptron (MLP), each layer followed by a ReLU activation function except the final one, to obtain the predicted control inputs $\u^0$, the state uncertainty tubes $\Rq^0$ as well as the control uncertainty tubes $\Ru^0$. 
The updated hidden state $h_1$ is then given back to the GRU block along with the second element of the input sequence. This process is repeated for each element $\boldsymbol{\q_{in}^k}$ until all predictions are obtained.

The network is intended to be used in a sampling-based tree planner, where local trajectories are concatenated to form a global one. 
Thus, since the hidden state encodes and accumulates temporal information about the input sequence, the final hidden state $h_{N}$ of a local trajectory obtained after an iteration of a sampling-based tree planner can be saved and then given back as the initial hidden state $h_{0}$ for future local trajectories considered during the next iterations. 
Therefore, in practice, this initial hidden state is generally not null.

\subsection{Dataset}

\begin{figure} [t]
    \centering
    \includegraphics[width=0.9\linewidth]{figures/learning_quadrotor/dataset_generation.png}%
    \caption{Dataset generation process. 
    Starting from the hovering state $\q_{init}$, at each iteration a new state $\boldsymbol{q_{rand}}$ is randomly sampled and connected to the previous state $\boldsymbol{q_{prev}}$ until a total trajectory length ($T_F$) of 15s is reached.
    Data annotation is performed by simulating the tracking of the generated trajectory under nominal parameters.
    }%
    \label{fig: data_generation}%
\end{figure}

In order to train the proposed neural network, we generated a dataset of trajectories computed in an obstacle-free environment 
as depicted in Figure.\ref{fig: data_generation}.
This ensures that the resulting learned model is totally independent of the environment and only depends on the system.
Each global trajectory starts from the same initial hovering state ($\q_{init}$) initialized at zero velocities and accelerations\footnote{Note that this initialization does not hurt the generalizability of the model to different initial conditions and environments with obstacles (cf. Section~\ref{sec:qualitative results}).}.
Based on the same principle as a sampling-based planner, the global trajectory is made up of local sub-trajectories until a total execution time length ($T_F$) of 15s is reached.
Each local sub-trajectory is generated by uniformly sampling an arrival state ($\q_{rand}$) and connecting it to the previous sampled state ($\q_{prev}$) using a \emph{Kinosplines} steering method \cite{cKino}.
These kinosplines have the advantage of enforcing the following kinodynamic constraints on the generated splines $[v_{max}, a_{max}, j_{max}, s_{max}]$, where $v_{max},a_{max},j_{max}$ and $s_{max}$ represent the maximum allowed velocity, acceleration, jerk and snap respectively.
If the local trajectory (between $\boldsymbol{q_{rand}}$ and $\boldsymbol{q_{prev}}$) expected execution time is greater than a maximum local duration $T_l$, it is truncated to $T_l$. Similarly, when the total trajectory duration $T_F$ exceeds 15s after adding a new state, it is truncate at 15s.

To generate the outputs, i.e. to annotate the data with uncertainty tubes and control inputs, the closed-loop dynamic and the sensitivity matrices are computed by simulating the tracking of the global trajectories using an integration time step $\Delta T$ which corresponds to the same time step used for collision checking in a sampling-based motion planner.

\begin{figure} [t]
    \centering
    \includegraphics[width=0.49\linewidth]{figures/learning_quadrotor/vnorm_val2.png}
    \includegraphics[width=0.49\linewidth]{figures/learning_quadrotor/vnorm_test.png}
    \caption{Velocity norm distribution in the training (left) and test sets (right).}%
    \label{fig: valvstest}%
\end{figure}
Using this mechanism, a training set composed of 8.000 trajectories and a validation set composed of 2.000 trajectories were generated, making sure that every trajectory in the dataset is different.
These trajectories were generated considering a maximum local duration of $T_l = 1s$ and an integration time step $\Delta T = 0.05s$.

The kinodynamic constraints enforced on the generated splines are $[v_{max}, a_{max}, j_{max}, s_{max}] = [5.0 \, m.s^{-1}, \allowbreak 1.5 \, m.s^{-2}, \allowbreak 15.0 \, m.s^{-3}, \allowbreak 30.0 \, m.s^{-4}]$. 
The nominal values of the uncertain parameters presented in Sect.~\ref{sec: sensi} are $\p_{c}$ = $[1.113, \allowbreak0.0, \allowbreak0.0, \allowbreak0.015, \allowbreak0.015, \allowbreak0.007]^T$ and their associated uncertainty range used for the tubes computation are $\delta\p = [7\%, \allowbreak3cm, \allowbreak3cm, \allowbreak10\%, \allowbreak10\%, \allowbreak10\%]^T$, which represents the variation of the parameters w.r.t. their associated nominal value.
The controller gains used are $\boldsymbol{k_x}=[20.0 \,, \allowbreak 20.0 \,, \allowbreak 25.0]$, $\boldsymbol{k_v}=[9.0 \,, \allowbreak 9.0 \,, \allowbreak 12.0]$, $\boldsymbol{k}_{R}=[4.6 \,, 4.6 \,, 0.8]$, $\boldsymbol{k_{\omega}}=[0.5 \,, \allowbreak 0.5 \,, \allowbreak 0.08]$.

In order to show the reliability and generalizability of the learned model, a test set composed of 1.000 trajectories was generated in the same way, but considering a maximum local duration $T_l = 2s$.
As a result, trajectories with higher velocities are encountered in the test set compared to the validation set, as depicted in Figure~\ref{fig: valvstest} where velocity norms can reach up to 7 m.s$^{-1}$ in the test set, compared with only 3 m.s$^{-1}$ in the validation set\footnote{Note that the velocity norms exceed the velocity limit $v_{max}$, this is expected since this limit applies to the components of the velocity vector rather than the norm.}.

The data annotation was performed by computing and integrating (Equation~\ref{eq:dyna_ctrl}) and (Equation~\ref{eq:dyna_sensi}) by mean of the dopri5~\cite{cdopri5} \myglsentry{odes} solver along a desired trajectory. 
Once $\bPi$ and $\bTheta$ had been computed, a simple projection was performed to recover the tubes thanks to (Eq.~\ref{eq:projection}).
Note that the control inputs $\u$ are computed during the \myglsentry{odes} resolution.
The mean and standard deviation values of the various components of the output vector for the validation and test sets generated by this setup are provided in Table~\ref{tab:datas_stats}.

\begin{table}[h]
\centering
\begin{tabular}{ | c | c || c |}
\hline
  \textbf{Output}  & \textbf{Validation set}  & \textbf{Test set} \\ \hline
$\Rq$ & $1.0e^{-1} \pm 2.0e^{-2}$ & $1.1e^{-2} \pm 2.1e^{-2}$ \\ \hline
$\u$ & $12469.3 \pm 861.6$ & $12476.8 \pm 1016.5$ \\ \hline
$\Ru$ & $7782.6 \pm 3172.3$ & $7828.6 \pm 2845.7$ \\ \hline
    
\end{tabular}
\caption{
Mean and standard deviation of the output vector components norm after data annotation for the validation and test sets.
$\Rq$ is expressed in \emph{m}, and $(\u, \Ru)$, are squared propeller angular velocities [(rad/s)²].}
 \label{tab:datas_stats}
\end{table}

\section{Evaluation}
\subsection{Metrics}

The performance of the model is evaluated using the Mean Absolute Error (MAE) over the different outputs. Rather than comparing the results on each sub-component of the input (e.g $\{x, y, z\}$ for $r_q$), we combine them into a single metric in order to obtain simpler and more general comparisons, as follows:
\begin{itemize}
    \item $\mathbf{MAE_{\Rq}}$ : represents the mean absolute error on the norm of $\Rq$. 
    For a given datapoint, given the ground truth and predicted state uncertainty tubes $\Rq$ and $\mathbf{\hat{r}_q}$, their respective norms $\|\Rq\|$ and $\|\mathbf{\hat{r}_q}\|$, are computed. 
    The mean absolute error, $\mathbf{MAE_{\Rq}}$, is then defined as:
    \begin{equation}\label{eq:MAE rq}
        \mathbf{MAE_{\Rq}} = MAE(\|\Rq\|, \|\mathbf{\hat{r}_q}\|)
    \end{equation}
    This metric is expressed in meters (m).
    \item $\mathbf{MAE_{u}}$ : represents the mean absolute error on the norm of $\u$. 
    The norms of the ground truth and predicted control inputs, $|\u|$ and $|\hat{\u}|$, are computed. 
    The mean absolute error, $\mathbf{MAE_{u}}$, is then calculated as the mean absolute error between these two norms. It is expressed in [(rad/s)²].
    \item $\mathbf{MAE_{\Ru}}$ : represents the mean absolute error on the norm of $\Ru$. As for the two previous metrics, $\mathbf{MAE_{\Ru}}$ is defined as the mean absolute error between the norms $\|\Ru\|$ and $\|\mathbf{\hat{r}_u}\|$. This metric is expressed in [(rad/s)²].
\end{itemize}

These metrics are averaged over all elements of a sequence, then averaged over all sequences in the validation and test sets. In addition, we use inference time as metric in order to evaluate the computational cost of our method compared to the defined baselines, as well as traditional uncertainty tubes computation methods involving an ODE solver. Time is denoted $T_{solver/NN}$ according to the model or solver used.

\subsection{Implementation details}

\section{Simulation results}
\subsection{Training}
\subsection{Model comparison}
\subsection{Ablation study}
\subsection{Qualitative results}

\section{Conclusion} \label{sec:NN_concl}

\todomarker{}
% \cleardoublepage
% \newpage 
% \ % The empty page
% \newpage
\chapter{Robust motion planning with accuracy optimization via learned sensitivity metrics}\label{chap:sampNN}
\markboth{Robust MP with accuracy optimization via learned sensitivity metrics}{}% To set left/right header
\localtableofcontents \newpage

This last contribution introduces an enhanced version of the sensitivity-aware variants from Chapter~\ref{chap:samp}, incorporating the GRU-based architecture from Chapter~\ref{chap:NN} to mitigate the computational cost of solving thousands of \myglsentry{odes} in sampling-based contexts. 
Moreover, methods from Chapter~\ref{chap:samp} and the ones in the literature focus on computing robust trajectories but do not consider the problem of \emph{also} minimizing uncertainty tubes at a desired location for increasing the accuracy of specific tasks (e.g., insertion, grasping).
Therefore, in this chapter, a novel sensitivity-based cost function is introduced, differing from the one in Chapter~\ref{chap:samp}. 
Instead of minimizing sensitivity along the entire trajectory, it focuses on minimizing the radius of uncertainty tubes in relevant directions at desired states for the task at hand. 
Additionally, unlike Chapter~\ref{chap:samp}, where trajectory states were the sole optimization variables, this chapter also considers controller gains as optimization variables during the planning process.

With respect to these considerations, the contribution of this chapter is twofold:
\begin{enumerate}
    \item A computationally efficient version of the \myglsentry{samp} variants (see Chapter~\ref{chap:samp}) is proposed, relying on a Gated Recurrent Unit neural network (GRU)\cite{cGRU} (see Chapter~\ref{chap:NN}), which quickly and accurately estimates time-varying uncertainty tubes and control input profiles along trajectories.
    A general method for incorporating this network into a sampling-based tree planners is presented to predict uncertainty tubes and control inputs along trajectories.
    \item Based on this new planner variants, a comprehensive framework is proposed that plans robust trajectories and locally optimizes them, along with the controller gains, to maximize accuracy at desired states of the planned trajectory for any system/controller.
\end{enumerate}

This chapter is structured as follows: Section~\ref{sec:RASAMP} introduces the integration of the GRU-based architecture within sampling-based planners and details the process of accuracy optimization.
Then, Section~\ref{sec:SimuResults} demonstrates the approach efficiency through simulation results.
Finally, these contributions are experimentally validated in Section~\ref{sec:Experimental} through two challenging scenarios illustrated in Figure~\ref{fig: Missed and succes}, which demonstrate the application to a quadrotor model in experimental settings involving uncertain parameters. 
The scenarios include: (i) robust navigation through a narrow window, and (ii) an in-flight ring-catching task. 
These experiments highlight the robustness and accuracy of the proposed framework under parameter uncertainties in the robot model.

This chapter is associated to the following publication in RA-L 2024~\cite{cRAL}: Wasiela, S., Cognetti, M., Giordano, P. R., Cortés, J., and Simeon, T. (2024). "Robust Motion Planning with Accuracy Optimization based on Learned Sensitivity Metrics." In IEEE Robotics and Automation Letters (RA-L).\footnote{The results presented in this chapter differ from those in the article due to improvements in implementation.}

\section{Robust and accurate planning framework}\label{sec:RASAMP}

THe learnin has to be mitigate: first we use Libtorch C implementation that does not support GPU computation but only CPU, thanks to small model size it is still good, secondly, the learning bypass the 2 final equation, sensi matrices and tubes, but we still need to compute the nominal state and solve small odes

The method consists of two stages: $(i)$ first, it generates a robust trajectory based on a Robust Sensitivity-Aware Motion Planner (R-SAMP) -- explained in Sect.~\ref{sec:RSAMP} -- that utilizes the GRU-based computation of the uncertainty tubes; $(ii)$ second, it optimizes the accuracy at some given states along this trajectory by minimizing the size of the uncertainty tubes at these locations.
The motivation behind the second stage is to improve the accuracy of the planned robust trajectory for tasks -- e.g., pick-and-place or insertion tasks -- where minimizing the deviation from the nominal trajectory is important only at specific designed locations, as for picking the ring in Fig.~\ref{fig: Missed and succes}.

The pseudo-code of RA-SAMP (Robust and Accurate Sensitivity-Aware Motion Planner) is presented in Alg.\ref{alg:RA-SAMP}. 
It takes as input (line 1) a list of desired states $list_{d} = (q_{d}^0, \dots, q_{d}^n)$ for which the accuracy should be optimized, and the initial controller gains vector $k_{c}^{init}$ considered constant all along the trajectory.

\begin{algorithm}[h]
\caption{RA-SAMP [$list_{d}, k_{c}^{init}$]}\label{alg:RA-SAMP}
\begin{algorithmic}[1]
\State $\pi_d^{tot} \gets \emptyset;$
\For {($i=1;\ i< len(list_{d});\ i=i+1$)}
    \State $\pi_d^{tot} \gets \pi_d^{tot} + $R-SAMP$(list_{d}(i-1),list_{d}(i));$
\EndFor
\State $\left \{ \pi_d^{tot}, k_{c}^{opt} \right \} \gets $A-Optim$(list_{d},\pi_d^{tot}, k_{c}^{init});$
\State \textbf{return} $\left \{ \pi_d^{tot}, k_{c}^{opt}  \right \}$;
\end{algorithmic}
\end{algorithm}

The first step of the algorithm consists of generating robust trajectories ($\pi_d^i$) between successive desired states in the list (line 3 of Alg.~\ref{alg:RA-SAMP}) by means of a robust sensitivity-aware motion planner called R-SAMP (explained in Sect.~\ref{sec:RSAMP}) that uses our learning approach.
These trajectories are concatenated into a global one $\pi_d^{tot}$, connecting all the desired states of $list_{d}$ (lines 2-5 in Alg.~\ref{alg:RA-SAMP}). 

The trajectory from R-SAMP is locally modified by A-Optim (line 6 in Alg.~\ref{alg:RA-SAMP}), aiming at optimizing the accuracy at specific desired states along the trajectory. 
This algorithm iteratively samples both the trajectory from R-SAMP and the controller gains, adjusting the former to minimize uncertainty at these states. 
Indeed, as demonstrated in \cite{AliIROS}, optimizing both factors concurrently results in minimizing the uncertainty.
The algorithm produces two offline outputs: $(i)$ a robust desired trajectory $\pi_d^{tot}$ optimized for accuracy at the desired states, and $(ii)$ the optimized controller gains vector $k_{c}^{opt}$, considered constant throughout the trajectory.

\subsection{Robust Sensitivity-Aware Motion Planner}\label{sec:RSAMP}

This section explains how the learned uncertainty tubes can be incorporated into any sampling-based tree planner in order to obtain a robust sensitivity-aware motion planner (R-SAMP).
As highlighted before, a key challenge in computing such tubes for a given trajectory lies in the high computational cost of numerically integrating the dynamics of $\bPi(t)$ and $\bTheta(t)$.
Additionally, when extending the tree and computing these sensitivity matrices, various initial conditions (e.g. initial control input, $\bPi_0$, etc.) must be embedded in the tree nodes.

We solve this problem thanks to the GRU network, which naturally encodes  this information in its ``memory'' terms, i.e., the so-called hidden state (see \cite{cGRU}).
An interesting feature of the algorithm is to leverage this latent state to embed the initial conditions into each node. This enables the reuse of the updated initial conditions for predictions in future extensions.
Note that a hidden state $h$ is unique according to its parent.
Therefore, its use is only applicable to tree-based planners, where each node has a single parent.
We show in Alg.~\ref{alg:ExtensionExample} how to incorporate this hidden state and tube predictions for the case of a standard RRT planner \cite{cRRT} with the pseudocode of the R-SARRT algorithm, as a particular instance of an R-SAMP planner. 
Note that the use of this hidden state and tube predictions can be similarly applied to other tree-based planners. For instance, in the results presented in Sect.~\ref{sec:RobustPlanSimu}, we used a robust RRT$^*$ implementation denoted R-SARRT$^*$.

First, R-SARRT performs the standard RRT procedure (lines 1-5) that produces a local desired trajectory $\pi_d$ between a sampled state (${q}^{rand}$) and its nearest state (${q}^{near}$) in the tree. 
Then, as the tubes are only valid around the nominal trajectories, which may differ from the desired ones depending on the controller performance, the nominal trajectory ($\pi_{n}$) is computed by the SimulateExecution function (line 6).
It corresponds to the simulated tracking in closed-loop of the desired trajectory ($\pi_d$) to be robustly checked by the system under the nominal parameters (i.e. $\p = \p_c$).
Next, the starting hidden state (as for $h_{0}$ of the GRU) is recovered from the tree node (line 7).
Such initial condition is used together with the above-mentioned nominal trajectory by the GRU that returns all the radii and the control inputs profiles ($\Rq,\Ru,\u$), together with the final hidden state $h_{F}$ to be reused as initial condition in subsequent iterations (line 8).
Then, for each state of the nominal trajectory $\pi_{n}$, the function \emph{IsRobust} (line 9) performs a robust collision checking by using the uncertainty radii, and tests if the inputs are within the admissible bounds that the system can exert.
If the extension is valid, the final state of the desired trajectory is inserted in the tree as a new node, embedding at the same time the final hidden state $h_{F}$ to be reused as initial condition in next iterations (line 10-12).
Finally, the algorithm returns a global trajectory connecting ${q}^{init}$ and ${q}^{goal}$ if one exists in the tree (lines 15).

\begin{algorithm}[t]
\caption{R-SARRT [$q^{init}, q^{goal}$]}\label{alg:ExtensionExample}
\begin{algorithmic}[1]
\State $T \gets$ InitTree$({q^{init}});$
\While{\textbf{not} StopCondition$(T, {q^{goal}})$}  
\State ${q^{rand}} \gets $Sample()$;$
\State ${q^{near}} \gets$ Nearest$(T,{q^{rand}});$
\State $\pi_d \gets$ Steer$({q^{near}},{q^{rand}});$
\State $\pi_{n} \gets $SimulateExecution$(\pi_d);$
\State $h_{0} \gets $GetNodeConditions$({q^{near}});$
\State $\left \{\Rq,\Ru,\u, h_{F} \right \} \gets $GRU$(\pi_d,h_{0});$
\If {$IsRobust(\Rq,\Ru,\u, \pi_{n})$}
        \State SetNodeConditions$({q^{rand}}, h_{F});$
        \State AddNewNode$(T, {q^{rand}});$
        \State AddNewEdge$(T, {q^{near}}, {q^{rand}});$
\EndIf
\EndWhile
\State \textbf{return} GetTrajectory$(T, q^{init}, q^{goal})$;
\end{algorithmic}
\end{algorithm}

\subsection{Robust local accuracy optimization \emph{(A-Optim)} }\label{sec:AOptim}

\subsubsection{Cost function}

The application of a local optimization method at this level is justified by the cost function considered in order to optimize the accuracy at desired states. 
Indeed, the cost of a trajectory $\pi$ is defined as:
\begin{equation}\label{eq: cost}
    c(\pi) = w_1\mathbb{E}[L] + w_2\mathbb{V}[L], \quad L = \left[\lambda_{0} ... \lambda_n \right]
\end{equation}
with $\mathbb{E}[L]$ and $\mathbb{V}[L]$ the mean and the variance of $L$, where $\lambda_k$ is the p-norm of the radii of interest in the $k$-th state in the $list_{d}$ of Sect.\ref{sec:RASAMP}, and $w_1, w_2$ are user-defined weigths.
The variance is considered in this cost function so that the minimization of a radius at a given point does not lead to the growth of another radius at another waypoint.
This function is neither additive (i.e., considering two trajectories ($\pi_{1}, \pi_{2}$), the cost of their concatenation $c(\pi_{1}|\pi_{2}) \neq c(\pi_{1}) + c(\pi_{2})$), nor monotonic. Therefore, it is unsuitable for global optimization using sampling-based motion planners like \cite{cRRT,cRRTstar}, since they require additive and monotonic objective functions.
Given that we do not have the analytic expression of the cost function derivatives, the accuracy optimization of A-Optim has to be performed by a derivative-free method. In this work, we simply used a robust version of the random shortcut algorithm~\cite{cShortcut} that performs robust collision checks (as in Alg.~\ref{alg:ExtensionExample}) to maintain the robustness of the initial trajectory computed by the R-SAMP planner.

\subsubsection{Methods}

\begin{enumerate}
    \item Shortcut
    \item ExtendedShortcut
    \item STOMP
    \item NL COBYLA
\end{enumerate}

Mentionned that maybe consider only a subpart of the trajectory before the accuracy waypoints might be better, only few time steps impact it.
Sampling for ExtendedShortcut maybe adaptive according to the difficulty of the convergence.

%%%%%%%%%%%%%%%%%%%%%%%%%%%%%%%%%%%%%%%%%%%%%%%%%%%%%%%%%%%%%%%%%%%%%%%%%%%%%%%%%%%%%%%%%%%%%%%%%%%%
\section{Simulation results} \label{sec:SimuResults}

This section first presents results showing the quality of the learning-based uncertainty tubes prediction and its high efficiency when used for robust motion planning. 
Then, we show the ability of the proposed R-SAMP planning approach to generate robust 
and accurate trajectories for the two experimental scenarios illustrated in Fig.~\ref{fig: Missed and succes}.

%%%%%%%%%%%%%%%%%%%%%%%%%%%%%%%%%%%%%%%%%%%%%%%%%%%%%%%%%%%%%%%%%%%%%%%%%%%%%%%%%%%%%%%%%%%%%%%%%%%%
\subsection{Quadrotor setup} \label{sec:quad_setup}

Considering that the quadrotor hovering state is achieved for control inputs values around 5.000 $(rad.s^{-1})^2$, equivalent to propeller speed of approximately 70 $rad.s^{-1}$, the input limits are set to twice this values, which means propelelr speed of 140, which induced 20.000 $(rad.s^{-1})^2$.

%%%%%%%%%%%%%%%%%%%%%%%%%%%%%%%%%%%%%%%%%%%%%%%%%%%%%%%%%%%%%%%%%%%%%%%%%%%%%%%%%%%%%%%%%%%%%%%%%%%%
\subsection{Learning-based tube computation} \label{sec:NNresult}

The performance of the method presented in Chap.~\ref{chap:NN} and apply to a sampling based algorithm is illustrated in Fig.~\ref{fig: NN pred} that depicts the norm of predicted vector components for a 300-state trajectory part of the tree.
\todomarker[]{POlish: Note that the predictions are only valid for the parameter maximum range $\delta\p$ chosen during the generation of the training set, and that the model is trained for given values of the controller gains.
Also, learning optimal controller gains is left to future work, as it would require additional work on database generation and data annotation, the A-Optim method cannot benefit from it.}


\begin{figure} [t]
    \centering
    \includegraphics[width=0.8\linewidth]{figures/robust_accurate/PredNorm.png} 
    \caption{Example of GRU predictions along a trajectory (orange) against true values (back). $||\Rq||, ||\u||$ and $||\Ru||$ refer to the norm of their respective vector components. $||\Rq||$ is expressed in $m$, and control input associated values ($||\u||,||\Ru||)$ are squared propeller speeds [(rad/s)²].}%
    \label{fig: NN pred}%
\end{figure}

\begin{figure} [!t]
    \centering
    \subfloat[\centering RRT]{{\includegraphics[width=0.4\linewidth]{figures/robust_accurate/iterations_RRT_new.png} }}%
    \subfloat[\centering RRT$^*$]{{\includegraphics[width=0.4\linewidth]{figures/robust_accurate/iterations_RRTstar_new.png} }}%
    \caption{Number of (a) RRT / (b) RRT$^*$ iterations as a function of planning time in an obstacle-free environment using the standard (non-robust) RRT/RRT$^*$ implementation (blue), compared to robust versions using the GRU-based tube prediction (green) or the integration of the dynamics of $\bPi$ (red), as done in \cite{cSAMP}.}%
    \label{fig: NNTime}%
\end{figure}

\todomarker[]{ADD profiling ODEs vs learning}

Fig.~\ref{fig: NNTime} shows the significant performance improvement of using this learning-based prediction within a sampling-based tree planner for checking the robustness of the local tree expansions (see Alg.~\ref{alg:ExtensionExample}), against the previous version \cite{cSAMP} that directly integrates the $\bPi$ dynamics.  
Results provided for RRT and its RRT$^*$ near time-optimal variant compare the number of iterations of the main loop of the algorithm as a function of computing time in an obstacle-free environment, showing in both cases a significant time gain thanks to the proposed learning method compared with the method that integrates the dynamics of $\bPi$.
Note that in the case of RRT, this time gain is constant ($3$ times faster)  because the expansion benefits from the neural network only once per iteration.
In the case of RRT$^*$, the denser the tree, the more robust collision tests are required for the rewiring connection phase. 
Therefore, much more time is saved when using the learning method. 
The gain on the planning time can reach more than one order of magnitude for problems requiring a significant amount of iterations.

%%%%%%%%%%%%%%%%%%%%%%%%%%%%%%%%%%%%%%%%%%%%%%%%%%%%%%%%%%%%%%%%%%%%%%%%%%%%%%%%%%%%%%%%%%%%%%%%%%%%
\subsection{Robust planning} \label{sec:RobustPlanSimu}

We first demonstrate good efficiency and robustness of the R-SARRT planner (see Sect.~\ref{sec:RSAMP}) from comparative simulation results with a standard (non-robust) RRT, a RandUp-RRT \cite{cRandUpRRT} and SARRT standing for a RRT implementation of our former framework SAMP \cite{cSAMP}.
The robust collision checking of R-SARRT and SARRT is achieved by enlarging the robot shape to account for uncertainty. 
As for RandUp-RRT, it has been implemented with 20 ``RandUP particles'' to approximate the reachable set, and no $\epsilon$-padding is used.

We also compared an asymptotically optimal version of our algorithm (R-SARRT$^*$) to 
a classic RRT$^*$ to compute near time-optimal trajectories. 
Both algorithms ran until the solution cost converged below a threshold.
Note that we cannot perform a comparison with the RandUp-RRT since there is no optimal version of the algorithm. 
The comparison is based on their planning time and success rate on the scenario depicted in Fig.\ref{fig: fig1robust}, using an Intel i9 CPU@2.6GHz and a RTX A3000 GPU for the GRU predictions.
The same geometric controller that steers the robot toward a sampled desired state $\q_d$ was used for all planners.

Table~\ref{tab:Robust window} shows comparative results averaged for each planner over 20 trajectories and 30 simulations with uncertain parameters in the range $\delta\p$ of \todo{add ref to chap model}. 
First note that RandUp-RRT, SARRT and R-SARRT have a much stronger robustness than standard non-robust RRT. 
R-SARRT has a success rate of 100\% compared to 99,2\% for RanUp-RRT. 
Indeed, as mentioned in \cite{cRandUpRRT, cRandUP}, the computation of a conservative reachable set requires some additional padding step, which is set to zero in our experiment\footnote{The padding value is a user parameter that is difficult to find. Choosing the wrong padding value can result in set estimations that are too conservative. We chose zero as in some experiments of \cite{cRandUpRRT}.}. 
Also note the higher efficiency of R-SARRT which only uses one prediction per iteration whereas RandUp-RRT needs to perform a propagation per particle, yielding to a longer planning time. 
As for SARRT, it does not build a robust tree like R-SARRT. Instead, it only robustly checks the final solution, causing frequent disconnections and re-connections of non-robust nodes, which results in a higher planning time.
RRT, which does not account for robustness, remains faster but with a significantly lower success rate. 
Similar results are observed on the optimal versions. 
An example of trajectories planned by RRT$^*$ and a robust version R-SARRT$^*$ optimizing the trajectory duration, and their associated simulations, is illustrated in Fig.\ref{fig: simu window}.
It shows the effective robustness of the proposed algorithm as illustrated by the higher success rate indicated in Table \ref{tab:Robust window}.

We also experimentally demonstrate the window scenario on a real quadrotor.
Uncertainties are added to the system by randomly attaching a mass of up to 80g (not known by the controller) to the drone as depicted in Fig.\ref{fig: drone_mass}.

\begin{table*}[t!]
    \centering
    \begin{tabular}{l|lllll|}
    \cline{2-6} & \multicolumn{5}{c|}{Basic Planners} \\ \cline{2-6} 
        & \multicolumn{1}{c|}{RRT} & \multicolumn{1}{c|}{RandUp-RRT} & \multicolumn{1}{c|}{SARRT} & \multicolumn{1}{c|}{LazySARRT} & \multicolumn{1}{c|}{DeepSARRT} \\ \hline
    \multicolumn{1}{|c|}{Success (\%)} & \multicolumn{1}{c|}{61.8}   & \multicolumn{1}{c|}{99.2} & \multicolumn{1}{c|}{\textbf{100.0}}  &   \multicolumn{1}{c|}{\textbf{100.0}}  & \multicolumn{1}{c|}{\textbf{100.0}}\\ \hline
    \multicolumn{1}{|c|}{Plan time (s)} & \multicolumn{1}{c|}{\textbf{7.1} $\pm$ 7.6}  & \multicolumn{1}{c|}{57.8 $\pm$ 49.1}  & \multicolumn{1}{c|}{ 78.5 $\pm$ 55.1} & \multicolumn{1}{c|}{45.6 $\pm$ 33.4} &   \multicolumn{1}{c|}{ 22.3 $\pm$ 15.8} \\ \hline
    \end{tabular}

    \caption{
    \label{tab:Robust window}
    Quadrotor application: Average planning time and robust feasibility success rates of the simulated motions planned using standard non-robust RRT, RandUP-RRT, the SARRT variant (see Chapter~\ref{chap:samp}), LazySARRT (see Chapter~\ref{chap:samp}), and DeepSARRT.
    The results are averaged over 20 plans and 30 simulations per plan.}
\end{table*}

\todomarker[]{ADD RRTstar ODEs in the table}

\begin{figure} [h]
    \centering
    \subfloat[\centering RRT$^*$]{{\includegraphics[width=0.4\linewidth]{figures/robust_accurate/simu_RRTstar_notube.png} }}%
    \subfloat[\centering R-SARRT$^*$]{{\includegraphics[width=0.4\linewidth]{figures/robust_accurate/simu_SARRTstar.png} }}%
    \caption{Planned trajectory (black) produced by a (a) RRT$^*$ and our (b) R-SARRT$^*$. 
    Simulated trajectories under uncertainty are displayed in green in the case of success, and in red in the case of a crash.}%
    \label{fig: simu window}%
\end{figure}

\begin{figure} [h]
    \centering
    
    \subfloat[\centering ]{{\includegraphics[width=0.42\linewidth]{figures/robust_accurate/drone_window.png} }\label{fig: drone_mass}}%
    \subfloat[\centering ]{{\includegraphics[width=0.3\linewidth]{figures/robust_accurate/drone_perch.png}} \label{fig: drone_perch}}%
    
    \caption{Quadrotor setups for the two scenarios considered for the experimental validation. (a) a drone equipped with a random mass to perform a robust navigation through a window (b) a drone equipped with a perch to catch the rings.}%
    \label{fig: exp setup}%
\end{figure}

\begin{figure} [h]
    \centering
    \subfloat[\centering RRT$^{*}$]{{\includegraphics[width=0.4\linewidth]{figures/robust_accurate/exp_RRTstar.png} }}%
    \subfloat[\centering R-SARRT$^*$]{{\includegraphics[width=0.378\linewidth]{figures/robust_accurate/exp_SARRTstar.png} }}%
    \caption{Experimental execution by a quadrotor with uncertainty of trajectories planned by  RRT$^*$ (a) and  R-SARRT$^*$ (b). Both trajectories are executed with the same uncertainty and a virtual collision is found in the RRT$^*$ case while the R-SARRT$^*$ execution is robust.}%
    \label{fig: exp window}%
\end{figure}

In this experiment, a non-robust trajectory planned by RRT$^*$ and a robust one planned by R-SARRT$^*$ were executed ten times, using the same masses and attachment points between the two algorithms.
All trajectories were planned offline on a remote computer. To make the robot execute them, the geometric controller~\cite{cLee} ran online on the quadrotor's onboard computer, tracking the trajectories provided as input. The robot state was measured using a motion capture system with millimeter accuracy, ensuring that the only source of uncertainty was the attached unknown mass.
Fig.\ref{fig: exp window} illustrates the experimental execution of a non-robust RRT$^*$ trajectory and a robust R-SARRT$^*$ trajectory. 
The figure shows the recorded executions within a virtual environment to detect virtual collisions, thus mitigating the risk of real crashes and damages to the robot.
The experimental results confirm the simulation observations, providing an overall success rate of 100\% in the case of the robust trajectory computed with R-SARRT$^*$, against 40\% for the classic RRT$^*$.

%%%%%%%%%%%%%%%%%%%%%%%%%%%%%%%%%%%%%%%%%%%%%%%%%%%%%%%%%%%%%%%%%%%%%%%%%%%%%%%%%%%%%%%%%%%%%%%%%%%%
\subsection{Accuracy optimization} \label{sec:AccOptSimu}

\begin{figure} [t]
    \centering
    {\includegraphics[width=0.5\linewidth]{figures/robust_accurate/accuracy_opti.png} }%
    \caption{Example of uncertainty ellipsoid without optimization (red), with local trajectory optimization (yellow), with gains optimization (blue), and with local trajectory and gains optimization at the same time (green).}%
    \label{fig: Acc opti}%
\end{figure}

We implemented the A-Optim method of Sect.\ref{sec:AOptim} by using a robust version of the random shortcut algorithm \cite{cShortcut} and \eqref{eq: cost} as cost function to be optimized, where the radii of interest are the ones along the $\boldsymbol{x}$, $\boldsymbol{\rho}$ and $\boldsymbol{\omega}$ components of $\boldsymbol{q}$.
At each iteration of this method, a shortcut is attempted between two states of the input trajectory that are randomly sampled together with the controller gain values, sampled between 50\% and 150\% of their nominal values.
Fig.\ref{fig: Acc opti} motivates why we optimized both the trajectory and the controller gains at the same time in the A-Optim function in order to minimize uncertainty for a given point, as mentioned in Sec.\ref{sec:RASAMP}. In fact, these results corroborate the findings of \cite{AliIROS}, but this time by employing a sampling-based motion planner that considers obstacles in the environment.

\section{Experimental validation} \label{sec:Experimental}

This section provides an experimental validation of both the deep learning-based robust motion planning variants, and the accuracy optimization stage in two challenging scenarios depicted in Figure~\ref{fig: Missed and succes}.

\begin{figure} [h]
    \centering
    \subfloat[\centering Robust motion planning]{{\includegraphics[width=0.4\linewidth]{figures/robust_accurate/topFig_tube.png} }\label{fig: fig1robust}}%
    \subfloat[\centering Accuracy optimization]{{\includegraphics[width=0.4\linewidth]{figures/robust_accurate/ring_success.png} } \label{fig: fig1acc}}%
    \caption{
    Two scenarios considered for the experimental validation of the proposed method:
    (a) Robust navigation of a drone through a narrow window. (b) Precision in-flight `ring catching' task, where the uncertainty on the position of the perch end-effector is minimized to successfully accomplish the task.
    A video of the experiments is available at: \href{https://laas.hal.science/hal-04642257}{https://laas.hal.science/hal-04642257}}%
    \label{fig: Missed and succes}%
\end{figure}

\subsection{Robust planning} \label{sec:RobustPlanExp}

\subsection{Accuracy optimization} \label{sec:AccOptExp}

\begin{figure} [t]
    \begin{subfigure}{0.63\linewidth}
      \includegraphics[width=\linewidth]{figures/robust_accurate/ring_opti_v1.png}
    \end{subfigure}\hfill
    \begin{subfigure}{0.37\linewidth}
        \includegraphics[width=\linewidth]{figures/robust_accurate/Exp_ring_no_opti_full_zoom.png}
        \includegraphics[width=\linewidth]{figures/robust_accurate/Exp_ring_opti_full_zoom.png}
      \end{subfigure}\hfill
      
    \caption{Experimental validation of the ``ring catching'' scenario with a perch-equipped drone (left) with the position of the perch end-effector at the second ring location over 10 trajectories non accuracy optimized (top right) and accuracy optimized (bottom right).}
    \label{fig: exp ring}
  \end{figure}

We evaluated our complete framework with the accuracy optimization in a scenario that involves the in-flight retrieval of two 2cm radius rings in a cluttered environment using a drone equipped with a perch (see Fig.\ref{fig: drone_perch}) in a (near) time optimal way.
The experimental setup is shown in Fig.\ref{fig: exp ring}.
When the first ring is caught it becomes part of the drone and modifies the overall mass/inertia and center of mass of the system in an unmodeled way. 
A success is characterized by the recovery of both rings, otherwise we consider the execution as a failure.

We executed 10 trajectories using a vanilla (non-robust) RRT$^*$ planner and the RA-SARRT$^*$ algorithm, both of which optimize the trajectory time.
The RRT$^*$ does not use the A-Optim method to optimize accuracy while the RA-SARRT$^*$ does, in addition to guaranteeing the robustness.
The offline optimization in A-Optim aimed at minimizing the uncertainty at the location of the two rings.
Fig.~\ref{fig: exp ring} shows the perch end-effector position at the second ring location in the non-optimized case and in the optimized one. 
In the latter case, the perch tip is closer to the reference point in the middle of the ring than in the former case. This translates into a higher success rate of nine out of ten attempts to catch the ring with the optimized approach, against only three times out of ten for the non-optimized case.
However, given the chosen system and controller parameters, there is no guarantee that the computed tube will be enclosed in within the ring.
This explains why we still encounter one failure in the optimized case.
Overall, the experimental results show a success rate of 90\% for RA-SARRT$^*$ against only 30\% for RRT$^*$.

%%%%%%%%%%%%%%%%%%%%%%%%%%%%%%%%%%%%%%%%%%%%%%%%%%%%%%%%%%%%%%%%%%%%%%%%%%%%%%%%%%%%%%%%%%%%%%%%%%%%
\section{Conclusion} \label{sec:Conclusion}

We have presented a motion planner able to generate trajectories that are both robust and accurate in the presence of model uncertainties for a variety of robot/controller pair. The proposed planner leverages a GRU-based learning approach that quickly and accurately estimates the control inputs and the sensitivity-based uncertainty tubes of the state and of the inputs. 
The results on a quadrotor robot confirm the efficiency of the proposed learning method and highlight the benefit of its integration within a motion planner, resulting in a significant reduction of the planning times. 
Moreover, we showed that our framework is able to locally optimize the planned trajectory in order to minimize the size of the uncertainty tubes of the state at some desired locations, allowing the system to accurately perform a precision task. An experimental demonstration involving a quadrotor UAV in a ring-catching task allowed to validate the approach in real conditions. Future works will focus on considering uncertainties not only in the dynamic model, by extending the computation of the tubes for state estimation uncertainties. 
Furthermore, we aim to expand the capabilities of the neural network to learn the optimal controller gains.

% \cleardoublepage
% \newpage 
% \ % The empty page
% \newpage
\chapter{Conclusion}\label{chap:concl}
\markboth{Conclusion}{}% To set left/right header

\glsresetall

In this thesis, global robust control-aware motion planners have been presented to tackle the parametric uncertainties in any robot/controller models.
It has been shown that 

\section{Contributions}

Chapter~\ref{chap:samp} presented how to incorporate the closed-loop sensitivity concept and the resulting uncertainty tubes within a global sampling-based framework, whereas previous works primarily focused on local trajectory generation.
This approach enabled the generation of globally sensitivity-optimal trajectories. 
Furthermore, for the first time, sensitivity-based uncertainty tubes were used as robust constraints during the planning process, both on the state and control input spaces.
However, while this approach results in the generation of robust trajectories to parametric uncertainties and the sensitivity-based uncertainty tubes are obtained at a low computational cost, the overall cost remains too high when considering the thousands of computations required in a sampling-based application.
Thus, a decoupled approach has also been studied, based on a robust lazy strategy, which offers significant computational gains, especially as the system complexity increases.

Whereas Chapter~\ref{chap:samp} leverages a lazy robust collision check approach to address the computational challenges posed by the thousands of sensitivity-based uncertainty tube computations required to ensure robustness in standard tree-based applications, Chapter~\ref{chap:NN} exploits the structural similarity between the set of \myglsentry{odes} required for tube computation and \myglsentry{rnns}, directly correlating the desired trajectory to the uncertainty tubes and thereby eliminating the need to solve them.
According to the task at hand, the approach proposed a method for generating the dataset using a sampling-based approach.
Relying on a simple \myglsentry{gru} cell, the proposed architecture demonstrates an effective balance between prediction accuracy and inference time, achieving a time reduction of one order of magnitude compared to solving the set of ODEs.

This GRU-based architecture has been leveraged in Chapter~\ref{chap:sampNN} to propose an efficient alternative to the lazy approach presented in Chapter~\ref{chap:samp}.
Deep learning-based sensitivity-aware variants were proposed, demonstrating their efficiency in generating robust motions within a manageable planning time.
Furthermore, while in Chapter~\ref{chap:samp} the sensitivity was used to generate globally sensitivity-minimal trajectories throughout the entire motion, in this work it has been leveraged to achieve accuracy optimization, considering the uncertainty tubes only at specific locations along the motion, based on the task at hand.
An extensive simulation campaign was conducted to identify the most effective local optimizer for optimizing such punctual cost functions, by considering both the trajectory and the controller gains as optimization variables.
Finally, both the robust planners and the accuracy optimization stage were evaluated in two challenging scenarios: a quadrotor navigating through a narrow window and an in-flight ring-catching task requiring high precision. 
These tests validated the efficiency of the proposed methods in an indoor environment.
However, the proposed framework should be further validated on more complex system such as aerial manipulator.

\section{Future Works}

Although this work represents a step toward an efficient global robust control-aware motion planner framework for any robot/controller pair, it can be further improved both from a theoretical perspective and an algorithmic point of view.

\paragraph{Theoretical}

First, as mentioned in Chapter~\ref{chap:samp}, none of the proposed \gls{samp} variants are proven complete.
While the \gls{deepsamp} variants offer more practical implementations, they lack formal guarantees due to the unpredictable nature of learning models.
However, future work could focus on bounding the first-order sensitivity-based tube approximations to establish the robust feasibility of the \myglsentry{samp} solutions and, consequently, their completeness.

As discussed in Section~\ref{sec:sensi}, the computation of closed-loop sensitivity and the subsequent uncertainty tubes relies on the assumption of perfect knowledge of the robot state. 
Therefore, the current system does not account for uncertainty in state measurements, which is unavoidable due to sensor bias and noise.
Consequently, one could consider computing uncertainty tubes by also accounting for a priori known measurable sensor biases and noise, using the observability Gramian, which is a metric that quantifies how the robot state can be inferred from its outputs.

While this thesis focuses primarily on parametric uncertainties and presents indoor experiments, a step toward a more general framework applicable outdoors would involve accounting for external system disturbances, such as wind.
Since the closed-loop sensitivity of any function of the state can be computed, external disturbances might be incorporated by evaluating the closed-loop sensitivity of barrier functions, for instance.

\paragraph{Algorithmic}

From an algorithmic perspective, a new local optimizer called ExtendedShortcut was proposed in Chapter~\ref{chap:sampNN}. 
This algorithm aims to combine a broader exploration of the initial trajectory surroundings by using noisy state sampling within a shortcut process. 
Simulation results have shown promising outcomes. 
However, further investigation into a more effective noisy state generation process, based on adaptive sampling, could help reduce the algorithm sensitivity to its hyperparameters.


One can further improve the \myglsentry{lazysamp} variants to allow online robust replanning by investigating a more sofisticate robust reconnection procedure when not using sensitivity cost but only robust constraints as done in "R. Bohlin and L. E. Kavraki, "Path planning using lazy prm", Proceedings 2000 ICRA. Millennium Conference. IEEE International Conference on Robotics and Automation. Symposia Proceedings (Cat. No. 00CH37065), vol. 1, pp. 521-528, 2000.".
In brief, the colliding edge is removed from the tree and searching continues with adding samples near the discarded edge to connect the two sections of the tree. The lazy-collision checking alleviates one of the main computational bottlenecks of sampling-based planning by limiting collision checking to the edges of the solution. 
However, if number of collisions in the solution is large, then many efforts are required to repair the solution. 
Therefore, an efficient online replanning technic would consist on a simple low frequency "geometric" planning which robustly checks and reconnect in a lazy way and tacking advantage of the learning ! SUPER FAST BZZ BZZ 


% \cleardoublepage
% \newpage 
% \ % The empty page
% \newpage
\appendix
\chapter{Robust feasibility checking}\label{chap:appendixA}

% \begin{figure} [t]
%     \centering
%     \includegraphics[width=0.8\linewidth]{figures/appendix/CCellipse.png}
%     \caption{Collision checking given a uniform scaling of the robot AABB.}
%     \label{fig:appendix_A}
% \end{figure}

% Given a current robot AABB, this proof aims to show that the closest point on the robot to an obstacle remains the closest point after uniform scaling, and that the closest point of the obstacle to the robot also remains unchanged (see Figure~\ref{fig:appendix_A}).

% Let $A$ and $B$ be two convex sets. 
% This assumption holds for the collision test presented in Section~\ref{sec:robust_CC} as the robot AABB is a convex shape, and the environment is decomposed into convex shapes.
% Let \( p_1 \in A \) such that \( p_1 = \arg \min_{a \in A} \|a - b\| \, \forall b \in B \), and \( p_2 \in B \) such that \( p_2 = \arg \min_{b \in B} \|a - b\| \, \forall a \in A \).
% Also let $\lambda A, \lambda \geq 1$ be a uniform scaling of $A$.

% One want to show that:
% \begin{enumerate}
%     \item The projection of $p_1$ due to the scaling remains the closest point to the obstacle.
%     \item $p_2$ remains the closest point to the projection of $p_1$. 
% \end{enumerate}
% \[
%     \left\{
%     \begin{aligned}
%         \lambda p_1 &= \argmin_{a \in A} \|\lambda a - b\| \, \forall b \in B \\
%         p_2 &= \argmin_{b \in B} \|\lambda p_1 - b\|
%     \end{aligned}
%     \right.
% \]

% \paragraph{1.}
% Let \( a \in A \) and \( b \in B \).
% One can express the following relationship:
% \[
% \|\lambda a - b\| = \lambda \left\|a - \frac{b}{\lambda}\right\|,
% \]
% where \(\frac{b}{\lambda} \in B\) because \(B\) is convex. 
% Therefore, one gets:
% \[
% \arg \min_{a \in A} \left\|a - \frac{b}{\lambda}\right\| = \arg \min_{a \in A} \|a - b\| = p_1,
% \]
% which implies:
% \[
% \arg \min_{a \in A} \|\lambda a - b\| = \lambda p_1.
% \]

% \paragraph{2.}
% Let \( b \in B \), \( b \neq p_2 \) such that:

% \[
% \|\lambda p_1 - b\|^2 = \|\lambda p_1 - p_2\|^2 + \|p_2 - b\|^2,
% \]
% and since \( p_2 \neq b \), one gets \( \|p_2 - b\|^2 > 0 \). 
% Therefore, \( \|\lambda p_1 - b\| > \|\lambda p_1 - p_2\| \), which implies \( \arg \min_{b \in B} \|\lambda p_1 - b\| = p_2 \).

Sampling-based algorithms generate global trajectories by combining multiple continuous local trajectories.
During the process, each of these local trajectories are subject to collision checks to determine their feasibility.
However, in this thesis, the collision check is extended to include a more general feasibility test, which also accounts for the control input space. 
This extension ensures that the generated trajectories are not only free from obstacles but also prevent control inputs from reaching saturation. 
Additionally, it accounts for uncertainty in both the state and control input spaces, further enhancing the robustness of the trajectories.
The following appendix describes how these extended feasibility check is performed to ensure the robustness of each local trajectory in this context.
It is important to note that these test is not continuous; rather, it is performed along a discrete representation of the local trajectories, sampled at the controller's operating frequency.

\paragraph{Robust collision checking}

In this thesis, collision detection is performed using the widely used C implementation of PyBullet~\cite{cBullet}, which operates as follows: 
\begin{enumerate}
    \item It starts with a broad-phase collision detection using \myglsentry{AABBs} to quickly eliminate pairs of objects that are too far apart to collide, allowing the more computationally intensive collision checks to focus only on pairs that are potentially close to each other.
    \item Then, it performs a narrow-phase collision detection that, after potential collision pairs are identified, checks for each pair. 
    For each identified potential collision pair, this phase uses a more precise robot representation (as defined by the user) and specialized collision algorithms to detect actual intersections and determine contact points, normals, and penetration depths.
\end{enumerate}

Extending this procedure to account for robot state uncertainty aims to verify that the resulting extended bounding volume, which the robot may occupy due to the uncertainty, is clear of obstacles.
In this work, such bounding volume is computed by considering only the uncertainty in the position subspace for simplicity (i.e., the $\{x,y,z\}$-subspace for the quadrotor or the $\{x,y\}$-subspace for the differential drive robot).

This work employ a robust collision check generic to all free flying robots that operates as follows:
\begin{enumerate}
    \item As mentioned above, the first phase performs a broad collision detection considering the current robot AABB. 
    Therefore, in the robust collision detection of this work, this phase involves creating an extended AABB by scaling the current robot AABB in all directions according to their respective uncertainty radii as shown in Figure~\ref{fig:CCmethods}. 
    \item The narrow phase approach leverage a fine robot representation.
    However, generating the accurate extended collision mesh that incorporates uncertainty, required for the second phase, is challenging, as it involves deforming all the vertices of the robot mesh.
    To address this, this work approximate the extended collision mesh by sampling on the surface of the uncertainty ellipsoid bounding box defined by the uncertainty tube radii (see Figure~\ref{fig:CCmethods}). 
    While this method accurately approximates the true extended collision shape, it requires multiple calls to the collision-checking function for each robot state tested.
\end{enumerate}

\begin{figure}[htp]
    \centering
    % Row 1
    \begin{subfigure}{0.4\textwidth}
        \centering
        \includegraphics[width=\linewidth]{figures/samp/CC2.png}
        \caption{}
        \label{fig:CC1}
    \end{subfigure}
    % \hfill
    \begin{subfigure}{0.4\textwidth}
        \centering
        \includegraphics[width=\linewidth]{figures/samp/CC3.png}
        \caption{}
        \label{fig:CC2}
    \end{subfigure}
    
    % Row 2
    \begin{subfigure}{0.4\textwidth}
        \centering
        \includegraphics[width=\linewidth]{figures/appendix/CCdrone1.png}
        \caption{}
        \vspace{-0.3cm}
        \label{fig:CC3}
    \end{subfigure}
    % \hfill
    \begin{subfigure}{0.4\textwidth}
        \centering
        \includegraphics[width=\linewidth]{figures/appendix/CCdrone2.png}
        \caption{}
        \vspace{-0.3cm}
        \label{fig:CCall}
    \end{subfigure}

    \caption{Figure showing the resulting shapes tested for collisions: (a) differential drive robot extended AABB, 
    (b) sampling-based approximated uncertain mesh, for a differential drive robot with its associated uncertainty ellipsoid bounding box (green), (c) quadrotor extended AABB, and (d)
    sampling-based approximated uncertain mesh, for a quadrotor with its associated uncertainty ellipsoid bounding box (green).
    Note that not all the sampled configurations are displayed for clarity.}
    \label{fig:CCmethods}
\end{figure}

Although the methods employed in this thesis for robust collision checking rely on the uncertainty ellipsoid bounding box computed using Equation~\ref{eq:radius}, the tubes are represented by ellipsoids in the various figures of this manuscript for smoother visualization.

\paragraph{Robust saturation checking}
Then, in this manuscript, the feasibility check is not restricted to the aforementioned collision test, but it also verifies that the robot control inputs does not saturate.
This test is performed by checking that the tube associated with each control input remains in its feasibility domain.
An example of infeasible input for the quadrotor case is presented in Figure~\ref{fig:invalid_inputs}, where the tube (green) around the nominal control input of the first actuator (blue) exceeds the maximum allowed input (red).
Note that this simple test is less costly than the robust collision checking one, it is therefore performed first by mean of computational efficiency. 

\begin{figure} [htp]
    \centering
    \includegraphics[width=0.8\linewidth]{figures/samp/Invalid_Inputs.png} 
    \caption{Non-robust nominal control input profile for the first rotor of a quadrotor (blue) along a specified trajectory, depicted with its uncertainty tube (green) and the control input limits (red).}%
    \label{fig:invalid_inputs}%
\end{figure}
% \cleardoublepage
% \newpage 
% \ % The empty page
% \newpage
\chapter{Learning curves}\label{chap:appendixB}

\begin{figure} [h]
    \centering
    \vspace{-0.2cm}
    \includegraphics[width=0.95\linewidth]{figures/learning_unic/learning_curves.png}
    \caption{Learning curves showing the evolution of training loss (top) and corresponding validation loss (bottom) for the differential drive robot application.}
    \label{fig:appendix_B_unic}
\end{figure}

\begin{figure} [h]
    \centering
    \vspace{-1cm}
    \includegraphics[width=0.95\linewidth]{figures/learning_quadrotor/learning_curves.png}
    \caption{Learning curves showing the evolution of training loss (top) and corresponding validation loss (bottom) for the quadrotor application.}
    \label{fig:appendix_B_quad}
\end{figure}
% \cleardoublepage
% \newpage 
% \ % The empty page
% \newpage
\chapter{ExtendedShortcut}\label{chap:appendixC}

This appendix provides grid search results to determine the optimal hyperparameters for the \myglsentry{saextendedshct} algorithm.

\begin{figure} [htp]
    \centering
    \includegraphics[width=0.9\linewidth]{figures/appendix/uniform_radius_0_01.png} \\
    \includegraphics[width=0.9\linewidth]{figures/appendix/bplot_uniform_radius_0_01.png}
    \caption{Cost optimization results using a uniform sampling and a radius $\delta = 0.01$, averaged over 10 plans in a free space environment. 
    The standard deviation is represented as an envelope around the mean curves.
    The red curve corresponds to $K = 1$, the purple curve to $K = 2$, the blue one to $K = 3$, the black to $K = 5$, and the brown curve to $K = 10$.}%
    \label{fig:uniform_radius_0_01}%
\end{figure}

\begin{figure} [htp]
    \centering
    \includegraphics[width=0.9\linewidth]{figures/appendix/uniform_radius_0_1.png} \\
    \includegraphics[width=0.9\linewidth]{figures/appendix/bplot_uniform_radius_0_1.png}
    \caption{Cost optimization results using a uniform sampling and a radius $\delta = 0.1$, averaged over 10 plans in a free space environment. 
    The standard deviation is represented as an envelope around the mean curves.
    The red curve corresponds to $K = 1$, the purple curve to $K = 2$, the blue one to $K = 3$, the black to $K = 5$, and the brown curve to $K = 10$.}%
    \label{fig:uniform_radius_0_1}%
\end{figure}

\begin{figure} [htp]
    \centering
    \includegraphics[width=0.9\linewidth]{figures/appendix/gaussian_radius_0_01.png} \\
    \includegraphics[width=0.9\linewidth]{figures/appendix/bplot_gaussian_radius_0_01.png}
    \caption{Cost optimization results using a Gaussian sampling with a standard deviation $\delta = 0.01$, averaged over 10 plans in a free space environment. 
    The standard deviation is represented as an envelope around the mean curves.
    The red curve corresponds to $K = 1$, the purple curve to $K = 2$, the blue one to $K = 3$, the black to $K = 5$, and the brown curve to $K = 10$.}%
    \label{fig:gaussian_radius_0_01}%
\end{figure}

\begin{figure} [htp]
    \centering
    \includegraphics[width=0.9\linewidth]{figures/appendix/gaussian_radius_0_1.png} \\
    \includegraphics[width=0.9\linewidth]{figures/appendix/bplot_gaussian_radius_0_1.png}
    \caption{Cost optimization results using a Gaussian sampling with a standard deviation $\delta = 0.1$, averaged over 10 plans in a free space environment. 
    The standard deviation is represented as an envelope around the mean curves.
    The red curve corresponds to $K = 1$, the purple curve to $K = 2$, the blue one to $K = 3$, the black to $K = 5$, and the brown curve to $K = 10$.}%
    \label{fig:gaussian_radius_0_1}%
\end{figure}

\begin{figure} [htp]
    \centering
    \includegraphics[width=0.9\linewidth]{figures/appendix/comp.png} \\
    \includegraphics[width=0.9\linewidth]{figures/appendix/bplot_comp.png}
    \caption{Comparison of the best cases found in their respective classes: The red curve corresponds to a Gaussian sampling with $K = 1$ and $\delta = 0.1$, the blue curve is also associated to a Gaussian sampling with $K = 1$ and $\delta = 0.01$.
    The cases using uniform sampling are represented by the black and purple curves, both achieved with $K = 1$, and $\delta = 0.01$ and $\delta = 0.1$ respectively.}%
    \label{fig:comp}%
\end{figure}
% \cleardoublepage
% \newpage 
% \ % The empty page
% \newpage
\chapter{Résumé en Français}
\label{app:fr_small}

La planification du mouvement des robots est essentielle pour garantir un comportement robotique à la fois efficace et sûr.
Cependant, les robots opérant dans des environnements réels font inévitablement face à des incertitudes, qu'il s'agisse de perturbations externes (e.g. le vent), d'inexactitudes dans les modèles, ou d'erreurs d'estimation d'état.
Une approche efficace pour gérer la complexité d'évoluer and présences de ces incertitudes repose sur le paradigme de la "prévision/rétroaction" ou "planification/contrôle". 
Ce processus se déroule en deux étapes principales :
\begin{enumerate}
    \item Phase de Planification (Prévision) : Une trajectoire de référence pour les états et commandes du robot est planifiée en se basant sur les informations disponibles, telles que les modèles du robot et de son environnement. 
    Cette étape, généralement réalisée hors ligne, intègre des contraintes (e.g. évitement de collisions, limitations des actionneurs) et optimise des métriques comme la longueur de la trajectoire ou l'efficacité énergétique. 
    Cependant, une exécution en boucle ouverte de cette trajectoire planifiée échoue souvent en pratique en raison des incertitudes qui affectent les références prévues.
    \item Phase de Contrôle (Rétroaction) : Pour garantir une exécution robuste, un contrôleur de mouvement est employé pour fermer la boucle entre le mouvement planifié et le mouvement réel. 
    Ce contrôleur compense les effets imprévus et les incertitudes qui n'ont pas été prises en compte lors de la planification.
\end{enumerate} 

Bien que cette approche séquentielle séparant planification et contrôle soit efficace, elle présente des limitations significatives :

\begin{itemize}
    \item Les planificateurs modernes excellent à générer des trajectoires réalisables et globalement optimales pour des systèmes à haute dimension et des contraintes complexes. 
    Cependant, ils ignorent généralement le rôle du contrôleur en temps réel, ce qui entraîne deux problèmes majeurs : (1) le contrôleur doit s’écarter de la trajectoire planifiée pour gérer les incertitudes et perturbations, compromettant ainsi rapidement la faisabilité et l’optimalité ; 
    (2) le planificateur ne tient pas compte de la robustesse intrinsèque offerte par le contrôleur, manquant ainsi des opportunités pour produire des plans de mouvement plus robustes.
    \item De nombreuses méthodes de contrôle adaptatif ou robuste (par exemple, H-infini, méthodes LPV) ont été conçues pour gérer efficacement les incertitudes et perturbations, mais elles sont souvent locales vis à vis de la trajectoire de référence. 
    Ces méthodes peinent à relever des défis plus larges comme la faisabilité sous contraintes, l’optimalité globale et les performances, qui sont mieux gérés par des approches de planification globale.
\end{itemize}
    
Pour combler l’écart entre ces deux communautés, plusieurs approches ont été introduites, notamment des contrôleurs plus globaux tels que le \gls{mpc}~\cite{cTMPC}. 
De plus, la dernière décennie a vu émerger le concept de "planification de mouvement avec rétroaction" (feedback motion planning)~\cite{cTognon, cContractThMP, cContractThOnlineMP, cMajundarLibrary, cFaSTrack, cRandUpRRT, cRandUP}. 
Cependant, ces méthodes continuent de rencontrer des défis en termes de généralisabilité, d'efficacité en temps de calcul et de dépendance à des modèles potentiellement inexacts de la pair robot/contrôleur en raison des incertitudes sur les paramètres du modèle.

Ainsi, ce travail, basé sur le paradigme de "planification de mouvement avec rétroaction" (ou "planification de mouvement consciente du contrôle"), exploite le concept de sensibilité en boucle fermée~\cite{cPi,cTh} (une extension de la notion de sensibilité qui incorpore le comportement du contrôleur vis-à-vis des incertitudes paramétriques) afin de créer des planificateurs robustes et conscients du contrôle pour une large classe de systèmes et de contrôleurs, tout en abordant les incertitudes dans leurs représentations modélisées.

Cette thèse débute par une revue de la littérature au Chapitre~\ref{chap:related_work}, qui fournit un aperçu de l’état de l’art. 
Elle se concentre d’abord sur les approches découplées pour la planification robuste de mouvement, en commençant par les méthodes de recherche de chemin et en progressant vers la planification kinodynamique de trajectoire associée à des stratégies de contrôle robuste. 
La revue se poursuit avec des approches unifiées, présentant des méthodes de planification de mouvement robustes prenant en compte les incertitudes, ainsi que des techniques de planification consciente du contrôle. 
Le chapitre se termine par une mise en avant des approches unifiées et robustes conscientes du contrôle, qui constituent le sujet principal de cette thèse.

Les chapitres suivants explorent en détail les contributions principales de cette thèse, depuis l’introduction des concepts de sensibilité en boucle fermée jusqu’à leur application pratique via des planificateurs conscients du contrôle optimisés par apprentissage profond, démontrant leur efficacité dans des scénarios exigeant robustesse et précision élevées.

Le chapitre~\ref{chap:samp} a présenté l'intégration du concept de sensibilité en boucle fermée et des tubes d'incertitude correspondants dans un cadre de planification global basé sur l'échantillonnage, répondant ainsi aux limitations des travaux antérieurs qui se concentraient principalement sur la génération locale de trajectoires. 
Cette méthodologie a permis de générer des trajectoires globalement optimales en termes de sensibilité. 
De plus, pour la première fois, les tubes d'incertitude basés sur la sensibilité ont été utilisés comme contraintes robustes dans le processus de planification, à la fois dans les espaces d'état et d'entrée de contrôle.

Bien que le calcul de ces tubes d'incertitude soit peu coûteux, la charge de calcul globale reste élevée en raison du grand nombre de calculs nécessaires dans les applications basées sur l'échantillonnage. 
Pour remédier à cela, une approche découplée s'appuyant sur une stratégie robuste et paresseuse a été explorée, réduisant considérablement les besoins en calcul, en particulier pour des systèmes de complexité croissante.

Dans cette continuité, le chapitre~\ref{chap:NN} a abordé davantage les défis computationnels en exploitant la similarité structurelle entre le système d'\gls{odes} nécessaire au calcul des tubes et les \myglsentry{rnns}. 
Le réseau de neuronne proposé a permis d'établire une corrélation directe entre la trajectoire planifiée et les tubes d'incertitude, éliminant ainsi le besoin de résoudre les ODEs. 
La méthode a impliqué la génération d'un jeu de données basé sur une approche par échantillonnage adaptée à la tâche. 
En utilisant une cellule simple de type \myglsentry{gru}, l'architecture proposée a atteint un compromis efficace entre la précision des prédictions et la vitesse d'inférence, réduisant le temps de calcul d'un ordre de grandeur par rapport aux solveurs d'ODEs traditionnels.

Dans le chapitre~\ref{chap:sampNN}, cette architecture basée sur les GRU a été appliquée pour développer une alternative efficace à la stratégie ponctuelle de vérification robuste des collisions présentée dans le chapitre~\ref{chap:samp}. 
Des variantes de planification conscient de la sensibilité basées sur l'apprentissage profond ont été proposées, démontrant leur efficacité pour générer des trajectoires robustes dans des temps de planification acceptables. 
De plus, alors que le chapitre~\ref{chap:samp} se concentrait principalement sur la minimisation globale de la sensibilité tout au long du mouvement, les méthodes de ce chapitre se sont orientées vers l'optimisation de la précision. 
Les tubes d'incertitude ont été appliqués de manière sélective à des segments spécifiques du mouvement en fonction des exigences de la tâche.

Une campagne de simulation approfondie a permis d'identifier le meilleur optimiseur local pour cette fonction de coût spécifique, en optimisant à la fois les trajectoires et les gains du contrôleur. 
Les méthodes proposées ont été validées expérimentalement dans deux scénarios exigeants: la navigation d'un quadrirotor à travers une fenêtre étroite et la capture en vol d'anneaux nécessitant une grande précision. 
Ces tests ont mis en évidence l'efficacité des méthodes proposés dans un environnement intérieur. 
Cependant, une validation supplémentaire est nécessaire pour évaluer son applicabilité à des systèmes plus complexes, tels que les manipulateurs aériens.
% \cleardoublepage
% \newpage 
% \ % The empty page
% \newpage

\backmatter

\printbibliography[heading=bibintoc,title={References}]
\cleardoublepage


\end{document}
